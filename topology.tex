%!TEX root = rootfile.tex

%-------------------------------------------------------------------%
%-------------------------------------------------------------------%
\section{Elements of general topology}
%-------------------------------------------------------------------%
%-------------------------------------------------------------------%

%-------------------------------------------------------------------%
\subsection{Ultrafilters and compacta}
%-------------------------------------------------------------------%

\begin{ntn}
	Write $ \Set $ for the category of tiny finite sets.
	Write $ \Fin \subset \Set $ for the full subcategory of finite sets,
	and write $ i $ for the inclusion $ \Fin \inclusion \Set$.
\end{ntn}

\begin{dfn}
	For any tiny set $ S $, write $ h^S $ for the functor $ \Fin \to \Set $ given by $ I  \mapsto \Map(S, I) $.
	An \defn{ultrafilter} $ \mu $ on $ S $ is a natural transformation
	\[
		\int_S (\cdot) \ d\mu \colon h^S \to i \comma
	\]
	which for any finite set $I$ gives a map
	\[
		\begin{tikzcd}[column sep={1ex}, row sep={0ex}]
			\Map(S, I) \ar[r] & I \\
			f \ar[r, mapsto] & \int_S f \ d \mu
		\end{tikzcd}
	\]

	Write $ \beta (S) $ for the set of ultrafilters on $ S $.
	For any set $ S $, the set $ \beta(S) $ is the set
	\[
		\beta(S) = \lim_{I \in \Fin_{S/}} I \period
	\]
	The functor
	\[
		\beta \colon \Set \to \Set
	\]
	is thus the right Kan extension of the inclusion $ \Fin \inclusion \Set $ along itself.
\end{dfn}

\begin{exm}
	For any set $ S $ and any element $ s \in S $, there is a \defn{principal ultrafilter} $ \delta_s $, which is defined so that
	\[
		\int_S f \ d \delta_s = f(s) \period
	\]
\end{exm}

Every ultrafilter on a finite set is principal,
but infinite sets have ultrafilters that are not principal.
To prove the existence of these, let us look at a more traditional way of defining an ultrafilter on a set.

\begin{dfn}
	Let $ S $ be a set, $ T \subseteq S$, and $ \mu $ an ultrafilter on $ S $.
	There is a unique \defn{characteristic map} $ \chi_T \colon S \to \{ 0,1 \}$ such that $ \chi_T(s) = 1 $ if and only if $ s \in T $.
	Let us write
	\[
		\mu(T) \coloneq \int_S \chi_T \ d \mu \period
	\]
	
	We say that \defn{$ T $ is $ \mu $-thick} if and only if $\mu(T) = 1$.
	Otherwise (that is, if $ \mu(T) = 0 $), then we say that $ T $ is \defn{$ \mu $-thin}.

	For any $ s \in S$, the principal ultrafilter $ \delta_s $ is the unique ultrafilter relative to which $ \{ s \} $ is thick.
\end{dfn}

\begin{nul}
	If $ S $ is a set and $ \mu $ is an ultrafilter on $ S $, then we can observe the following facts about the collection of thick and thin subsets (relative to $ \mu $):
	\begin{enumerate}[(1)]
		\item The empty set is thin.
		\item Complements of thick sets are thin.
		\item Every subset is either thick or thin.
		\item Subsets of thin sets are thin.
		\item The intersection of two thick sets is thick.
	\end{enumerate}
	In other words, if $ S $ is a set, then an ultrafilter on $ S $ is tantamount to a Boolean algebra homomorphism $ \PP(S) \to \{0,1\} $.

	It is possible to define ultrafilters on more general posets, and if $ P $ is a Boolean algebra, then an ultrafilter is precisely a Boolean algebra homomorphism $ P \to \{0, 1\} $.
\end{nul}

\begin{nul}
	Ultrafilters are functorial in maps of sets.
	Let $ \phi \colon S \to T $ be a map, and let $ \mu $ be an ultrafilter on $ S $.
	The ultrafilter $ \phi_{\ast}\mu $ on $ T $ given by
	\[
		\int_T f \ d (\phi_{\ast}\mu) = \int_S (f \circ \phi) \ d \mu \period
	\]
	For any $ U \subseteq T$, one has in particular
	\[
		(\phi_{\ast} \mu)(U) = \mu (\phi^{-1}(U)) \period
	\]
	Thus $ U $ is $ \phi_{\ast} \mu $-thick if and only if $ \phi^{-1} U $ is $ \mu $-thick.
\end{nul}

\begin{dfn}
	A \defn{system of thick subsets} of $ S $ is a collection $ F \subseteq \PP(S) $ such that for any finite set $ I $ and any partition
	\[
		S = \coprod_{ i \in I } S_i \comma
	\]
	there is a unique $ i \in I $ such that $ S_i \in F $.
\end{dfn}

\begin{cnstr}
	We have seen that an ultrafilter $ \mu $ specifies the system $ F_{\mu} $ of $ \mu $-thick subsets.
	In the other direction, attached to any system $F$ of thick subsets is an ultrafilter $\mu_F$: for any finite set $ I $ and any map $ f \colon S \to I $, the element $ i = \int_S f \ d \mu \in I $ is the unique one such that $ S_i \in F$.

	The assignments $ \mu \mapsto F_{\mu} $ and $ F \mapsto \mu_F $ together define a bijection between ultrafilters on $S$ and systems of thick subsets.
\end{cnstr}

\begin{dfn}
	If $ S $ is a set, and if $ G \subseteq \PP(S) $, then an ultrafilter $ \mu $ is said to be \defn{supported on $ G $} if and only if every element of $G$ is $ \mu $-thick, that is, $ G \subseteq F_{\mu} $.
\end{dfn}

\begin{lem} \label{generateultrafilters}
	Let $ S $ be a set, and let $ G \subseteq \PP(S) $.
	Assume that no finite intersection of elements of $ G $ is empty.
	Then there exists an ultrafilter $ \mu $ on $ S $ supported on $ G $.
\end{lem}

\begin{proof}
	Consider all the families $ A \subseteq \PP(S) $ with the following properties:
	\begin{enumerate}[(1)]
		\item $ A $ contains $ G $;
		\item \label{FIP} no finite intersection of elements of $ A $ is empty.
	\end{enumerate}
	By Zorn's lemma there is a maximal such family, $ F $.

	We claim that $ F $ is a system of thick subsets.
	For this, let $ S = \coprod_{i \in I} S_i $ be a finite partition of $ S $.
	Condition \ref{FIP} ensures that at most one of the summands $ S_i $ can lie in $ F $.
	Now suppose that none of the summands $ S_i $ lies in $ F $.
	Consider, for each $ i \in I $, the family $ F \cup \{ S_i \} \subseteq \PP(S) $;
	the maximality of $ F $ implies that none of these families can satisfy Condition \ref{FIP}.
	Thus for each $ i \in I $, there is an empty finite intersection
	$S_i \cap \bigcap_{j = 1}^{n_i} T_{ij} = \varnothing $.
	But this implies that the intersection $ \bigcap_{i \in I}\bigcap_{j = 1}^{n_i} T_{ij} $ is empty, contradicting Condition \ref{FIP} for $ F $ itself.
	Hence at least one -- and thus exactly one -- of the summands $ S_i $ lies in $ F $.
	Thus $ F $ is a system of thick subsets of $ S $.
\end{proof}

\begin{nul}
	It is not quite accurate to say that the Axiom of Choice is \emph{necessary} to produce nonprincipal ultrafilters, but it is true that their existence is independent of Zermelo--Fraenkel set theory.
\end{nul}

\begin{nul}
	If $ \phi $ is a functor $ \Set \to \Set $, then a natural transformation $ \phi \to \beta $ is the same thing as a natural transformation $ \phi \circ i \to i $.
	Please observe that we have a canonical identification $ \beta \circ i = i $.

	It follows readily that the functor $ \beta $ is a monad: the unit $ \delta \colon \id \to \beta $ corresponds to the identification $ {\id} \circ i = i $, and the multiplication $ \mu \colon \beta^2 \to \beta $ corresponds to the identification $ \beta^2 \circ i = i $.

	The unit for the monad $\beta$ structure is the assignment $ s \mapsto \delta_s $ that picks out the principal ultrafilter at a point.

	To describe the multiplication $ \tau \mapsto \mu_{\tau} $, let us write $ T^{\dag} $ for the set of ultrafilters supported on $\{T\}$.
	Now if $ \tau $ is an ultrafilter on $ \beta(S) $, then $ \mu_{\tau} $ is the ultrafilter on $S$ such that
	\[
		\mu_{\tau} ( T ) = \tau ( T^{\dag} ) \period
	\]
\end{nul}

\begin{cnstr}
	Let $ \categ{Top} $ denote the category of tiny topological spaces.
	If $ S $ is a set, we can introduce a topology on $ \beta(S) $ simply by forming the inverse limit $ \lim_{I \in \Fin_{S/}} I $ in $ \categ{Top} $.
	That is, we endow $ \beta(S) $ with the coarsest topology such that all the projections $ \beta(S) \to I $ are continuous.
	We call this the \defn{Stone topology} on $\beta(S)$.
	By Tychonoff, this limit is a compact Hausdorff topological space.
	This lifts $ \beta $ to a functor $ \Set \to \Top $.
\end{cnstr}

\begin{nul}
	Let's be more explicit about the topology on $ \beta(S) $.
	The topology on $ \beta(S) $ is generated by the sets $ T^{\dag} $ (for $ T \subseteq S $).
	In fact, since the sets $ T^{\dag} $ are stable under finite intersections, they form a base for the Stone topology on $ \beta(S) $.
	Additionally, since the sets $ T^{\dag} $ are stable under the formation of complements, they even form a base of clopens of $ \beta(S) $.
\end{nul}

\begin{dfn} \label{compactaasbetaalgebras}
	A \defn{compactum} is an algebra for the monad $ \beta $.
	Hence a compactum consists of a set $ K $ and a map $ \lambda_K \colon \beta(K) \to K $, which is required to satisfy the usual identities:
	\[
		\lambda_K(\lambda_{K,\ast}\tau) = \lambda_K(\mu_{\tau}) \andeq{} \lambda_K(\delta_s) = s \comma
	\]
	for any ultrafilter $ \tau $ on $ \beta(S) $ and any point $ s \in S $.
	The image $ \lambda_K(\mu) $ will be called the \defn{limit} of the ultrafilter $\mu$.
	We write $ \Comp $ for the category of compacta.
\end{dfn}

\begin{cnstr} \label{turnacompactumintoatopspace}
	If $K$ is a compactum, then we use the limit map $ \lambda_K \colon \beta(K) \to K $ to topologise $ K $ as follows.
	For any subset $ T \subseteq K $, we define the closure of $ T $ as the image $ \lambda_K(T^{\dag}) $.

	A subset $ Z \subseteq K $ is thus closed if and only if the limit of any ultrafilter relative to which $ Z $ is thick lies in $ Z $.
	Dually, a subset $ U \subseteq K $ is open if and only if it is thick with respect to any ultrafilter whose limit lies in $ U $.

	We denote the resulting topological space $ K^{\textit{top}} $.
	The assignment $ K \mapsto K^{\textit{top}} $ defines a lift $ \Alg(\beta) \to \Top $ of the forgetful functor $ \Alg(\beta) \to \Set $.
\end{cnstr}

\begin{prp} \label{compactaarecompacta}
	The functor $ K \mapsto K^{\textit{top}} $ identifies the category of compacta with the category of compact Hausdorff topological spaces. 
\end{prp}

We will spend the remainder of this section proving this claim.
Please observe first that $ K \mapsto K^{\textit{top}} $ is faithful.
What we will do now is prove:
\begin{enumerate}[(1)]
	\item that for any compactum $ K $, the topological space $ K^{\textit{top}} $ is compact Hausdorff;
	\item that for any compact Hausdorff topological space $ X $, there is a $ \beta $-algebra structure $ K $ on the underlying set of $ X $ such that $ X \cong K^{\textit{top}} $; and
	\item that for any compacta $ K $ and $ L $, any continuous map $ K^{\textit{top}} \to L^{\textit{top}} $ lifts to a $ \beta $-algebra homomorphism $ K \to L$.
\end{enumerate}
To do this, it is convenient to describe a related idea: that of \emph{convergence} of ultrafilters on topological spaces.

\begin{dfn} \label{limitpointofultrafilter}
	Let $ X $ be a topological space, and let $ x \in X $.
	We say that $ x $ is a \defn{limit point} of an ultrafilter $ \mu $ on (the underlying set of) $ X $ if and only if every open neighbourhood of $ x $ is $ \mu $-thick.
	In other words, $ x $ is a limit point of $ \mu $ if and only if, for every open neighbourhood $ U $ of $ x $, one has $ \mu \in U^{\dag} $.
\end{dfn}

\begin{lem} \label{opensetsarethickwrtultrafilters}
	Let $ X $ be a topological space, and let $ U \subseteq X$ be a subset.
	Then $ U $ is open if and only if it is thick with respect to any ultrafilter with limit point in $ U $.
\end{lem}

\begin{proof}
	If $ U $ is open, then $ U $ is by definition thick with respect to any ultrafilter with limit point in $ U $.

	Conversely, assume that $ U $ is thick with respect to any ultrafilter with limit point in $ U $.
	Let $ u \in U $.
	Consider the set $ G \coloneq N(u) \cup \{ X \smallsetminus U \} $, where $N(u)$ is the collection of open neighbourhoods of $u$.
	If $ U $ does not contain any open neighbourhood of $u$, then no finite intersection of elements of $ G $ is empty.
	By \Cref{generateultrafilters} there is an ultrafilter $ \mu $ supported on the $ N(u) \cup \{ X \smallsetminus U \} $, whence $ u $ is a limit point of $ \mu $, but $ U $ is not $ \mu $-thick.
	This contradicts our assumption, and so we deduce that $ U $ contains an open neighbourhood of $ u $.
\end{proof}

\begin{lem} \label{continuityviaultrafilters}
	Let $ X $ and $ Y $ be topological spaces, and let $ \phi \colon X \to Y $ be a map.
	Then $ \phi $ is continuous if and only if, for any ultrafilter $ \mu $ on $ X $ with limit point $ x \in X $, the point $ \phi(x) $ is a limit point of $ \phi_{\ast}\mu $.
\end{lem}

\begin{proof}
	Assume that $ \phi $ is continuous, and let $ \mu $ be an ultrafilter on $ X $, and assume that $ x \in X $ is a limit point of  $ \mu $.
	Now assume that $ V $ is an open neighbourhood of $ \phi(x) $.
	Since $ \phi^{-1}V $ is an open neighbourhood of $ x $, so it is $ \mu $-thick, whence $ V $ is $\phi_{\ast}\mu$-thick.
	Thus $ \phi(x) $ is a limit point of $ \phi_{\ast}\mu $.

	Assume now that if $ x \in X $ is a limit point of an ultrafilter $ \mu $, then $ \phi(x) $ is a limit point of $ \phi_{\ast}\mu $.
	Let $ V \subseteq Y $ be an open set.
	Let $ x \in \phi^{-1}(V) $, and let $ \mu $ be an ultrafilter on $ X $ with limit point $ x $.
	Then $ \phi(x) $ is a limit point of $ \phi_{\ast}\mu $, so $V$ is $ \phi_{\ast}\mu $-thick, whence $ \phi^{-1}(V) $ is $ \mu $-thick.
	It follows from \Cref{opensetsarethickwrtultrafilters} that $\phi^{-1}(V)$ is open.
\end{proof}

\begin{lem} \label{quasicompactiffeveryultrafilterhasalimitpoint}
	Let $ X $ be a topological space.
	Then $ X $ is quasicompact if and only if every ultrafilter on $ X $ has at least one limit point.
\end{lem}

\begin{proof}
	Assume first that $ X $ is quasicompact.
	Let $ \mu $ be an ultrafilter on $ X $, and assume that $ \mu $ has no limit point.
	Select, for every point $ x \in X $, an open neighbourhood $ U_x $ thereof that is not $ \mu $-thick.
	Quasicompactness implies that there is a finite collection $ x_1, \dots, x_n \in X $ such that $ \{ U_{x_1}, \dots, U_{x_n} \} $ covers $ X $.
	But at least one of $ U_{x_1}, \dots, U_{x_n} $ must be $ \mu $-thick.
	This is a contradiction.

	Now assume that $ X $ is not quasicompact.
	Then there exists a collection $ G \subseteq \PP(X) $ of closed subsets of $ X $ such that the intersection all the elements of $ G $ is empty, but no finite intersection of elements of $ G $ is empty.
	In light of \Cref{generateultrafilters}, there is an ultrafilter $ \mu $ with the property that every element of $ G $ is thick.
	For any $ x \in X $, there is an element $ Z \in G $ such that $ x \in X \smallsetminus Z $.
	Since $ Z $ is $ \mu $-thick, $ X \smallsetminus Z $ is not.
	Thus $ \mu $ has no limit points.
\end{proof}

\begin{lem} \label{hausdorffiffeveryultrafilterhasatmostonelimitpoint}
	Let $ X $ be a topological space.
	Then $ X $ is Hausdorff if and only if every ultrafilter on $ X $ has at most one limit point.
\end{lem}

\begin{proof}
	Assume that $ \mu $ is an ultrafilter with two distinct limit points $ x_1 $ and $ x_2 $.
	Choose open neighbourhoods $ U_1 $ of $ x_1 $ and $ U_2 $ of $ x_2 $.
	Since they are both $ \mu $-thick, they cannot be disjoint;
	hence $ X $ is not Hausdorff.

	Conversely, assume that $ X $ is not Hausdorff.
	Select two points $ x_1 $ and $ x_2 $ such that every open neighbourhoods $ U_1 $ of $ x_1 $ and $ U_2 $ of $ x_2 $ intersect.
	Now the set $ G $ consisting of open neighbourhoods of either $ x_1 $ \emph{or} $ x_2 $ has the property that no finite intersection of elements of $ G $ is empty.
	In light of \Cref{generateultrafilters}, there is an ultrafilter $ \mu $ with the property that every element of $ G $ is thick.
	Thus $ x_1 $ and $ x_2 $ are limit points of $ \mu $.
\end{proof}

Let us now return to our functor $ K \mapsto K^{\textit{top}} $.

\begin{lem} \label{limitsarelimits}
	Let $ K $ be a compactum, and let $ \mu $ be an ultrafilter on $ K $.
	Then a point of $ K^{\textit{top}} $ is a limit point of $ \mu $ in the sense of \Cref{limitpointofultrafilter} if and only if it is the limit of $ \mu $ in the sense of \Cref{compactaasbetaalgebras}.
\end{lem}

\begin{proof}
	Let $ x \coloneq \lambda_K(\mu) $.
	The open neighbourhoods $ U $ of $ x $ are by definition thick (relative to $ \mu $), so certainly $ x $ is a limit point of $ \mu $.

	Now assume that $ y \in K^{\textit{top}} $ is a limit point of $ \mu $.
	To prove that the limit of $ \mu $ is $ y $, we shall build an ultrafilter $ \tau $ on $ \beta(K) $ with the following properties:
	\begin{enumerate}[(1)]
		\item under the multiplication $ \beta^2 \to \beta $, the ultrafilter $ \tau $ is sent to $ \mu $; and
		\item under the map $ \lambda_{\ast} \colon \beta^2 \to \beta  $, the ultrafilter $ \tau $ is sent to $\delta_y$.
	\end{enumerate}
	Once we have succeeded, it will follow that
	\[
		\lambda_K( \mu ) = \lambda_K( \mu_{\tau} ) = \lambda_K(\lambda_{K,\ast}\tau) = \lambda_K(\delta_y) = y \comma
	\]
	and the proof will be complete.

	Consider the family $ G' $ of subsets of $ \beta(K) $ of the form $ T^{\dag} $ for a $ \mu $-thick subset $ T \subseteq S $;
	since these are all nonempty and they are stable under finite intersections, it follows that no finite intersection of elements of $ G' $ is empty.

	Now consider the set $ G \coloneq G' \cup \{ \lambda_K^{-1}\{y\} \}$.
	If $ T $ is $ \mu $-thick, then we claim that there is an ultrafilter $ \nu \in \lambda_K^{-1}\{y\} \cap T^{\dag} $.
	Indeed, consider the set $ N(y) \cup \{T\} $, where $ N(y) $ is the collection of open neighbourhoods of $ y $.
	Since every open neighbourhood of $ y $ is $ \mu $-thick, no intersection of an open neighbourhood of $ y $ with $ T $ is empty.
	By \Cref{generateultrafilters} there is an ultrafilter supported on $ N(y) \cup \{T\} $, which implies that no finite intersection of elements of $ G $ is empty.

	Applying \Cref{generateultrafilters} again, we see that $ G $ supports an ultrafilter $ \tau $ on $ \beta(K) $.
	For any $ T \subseteq K $,
	\[
		\mu_{\tau}(T) = \tau(T^{\dag}) \comma
	\]
	so since $ \tau $ is supported on $ G' $, it follows that $ \mu_{\tau} = \mu $.
	At the same time, since $ \tau $ is supported on $ \{\lambda_K^{-1}\{y\}\} $, it follows that $ \{y\} $ is thick relative to $ \lambda_{K,\ast}\tau $, whence $ \lambda_{K,\ast}\tau = \delta_y $.
\end{proof}

\begin{proof}[Proof of \Cref{compactaarecompacta}]
	Let $ K $ be a compactum.
	Combine \Cref{hausdorffiffeveryultrafilterhasatmostonelimitpoint,quasicompactiffeveryultrafilterhasalimitpoint,limitsarelimits} to conclude that $ K^{\textit{top}} $ is a compact Hausdorff topological space.

	Let $ X $ be a compact Hausdorff topological space with underlying set $ K $.
	Define a map $ \lambda_K \colon \beta(K) \to K $ by carrying an ultrafilter $ \mu $ to its unique limit point in $ X $.
	This is a $ \beta $-algebra structure on $ X $, and it follows from \Cref{limitsarelimits} and the definition of the topology together imply that $ X \cong K^{\textit{top}}$.

	Finally, let $ K $ and $ L $ be compacta, and let $ \phi \colon K^{\textit{top}} \to L^{\textit{top}} $ be a continuous map.
	To prove that $ \phi $ is a $ \beta $-algebra homomorphism, it suffices to confirm that if $ \mu $ is an ultrafilter on $ K $, then
	\[
		\lambda_L (\phi_{\ast} \mu) = \phi (\lambda_K(\mu)) \comma
	\]
	but this follows exactly from \Cref{continuityviaultrafilters}.
\end{proof}

\begin{nul}
	We opted in \Cref{turnacompactumintoatopspace} to define the topology on a compactum $ K $ in very explicit terms, but note that the map $ \lambda_K \colon \beta(K) \to K^{\textit{top}} $ is a continuous surjection between compact Hausdorff topological spaces.
	Thus $ K^{\textit{top}} $ is endowed with the quotient topology relative to $ \lambda_K $.
\end{nul}

%-------------------------------------------------------------------%
\subsection{Stone spaces and projective compacta}
%-------------------------------------------------------------------%

\begin{dfn}
	Let $ X $ be a topological space.
	One says that $ X $ is \defn{totally separated} if and only if, for any two distinct points $ x, y \in X $, there exists a clopen subset $ V \subseteq X $ that contains $ x $ but not $ y $.
\end{dfn}

\begin{lem}
	A compactum $ K $ is totally separated if and only if it admits a base consisting of clopen sets.
\end{lem}

\begin{proof}
	Assume that $ K $ is totally separated.
	Let $ U \subseteq K $ be an open subset.
	It suffices to show that for any point $ x \in U $, there is a clopen neighbourhood of $ x $ that is contained in $ U $.
	For any $ y \notin U $, let $ V_y \subseteq X $ be a clopen that contains $ y $ but not $ x $;
	now $ \{ V_y \}_{y \in X \smallsetminus U} $ covers $ X \smallsetminus U $.
	Since $ X \smallsetminus U $ is a closed subset of a compactum, it too is compact, whence there exist finitely many points $ y_1, \dots, y_n \in X \smallsetminus U $ such that $ \{ V_{y_1}, \dots, V_{y_n} \} $ cover $ X \smallsetminus U $.
	Now the complement 
	\[
		X \smallsetminus (V_{y_1} \cup \dots \cup V_{y_n})
	\]
	is a clopen neighbourhood of $ x $ contained in $ U $.

	Conversely, assume that $ X $ admits a base of clopen subsets, and let $ x, y \in X $ be distinct points of $ X $.
	By Hausdorffness, there exists an open neighbourhood $ U $ of $ x $ that does not contain $ y $. 
	Since $ X $ admits a base of clopen subsets, there is a clopen neighbourhood of $ x $ that is contained in $ U $, which therefore does not contain $ y $.
\end{proof}

\begin{dfn}
	A compactum is a \defn{Stone space} if and only if it is totally separated.
	Let us write $ \Stone \subseteq \Comp $ for the full subcategory spanned by the Stone spaces.
\end{dfn}

\begin{exm}
	Clearly any finite set is a Stone space.

	More generally, let $ I \colon A^{\op} \to \Fin $ be a diagram of finite sets.
	If we form the limit $ K = \lim_{\alpha \in A^{\op}} I_{\alpha}^{\disc}$ in $ \Top $ or $ \Comp $, then $ K $ is a Stone space.
	Indeed, $ K $ is clearly Hausdorff and compact by Tychonoff;
	since it admits a base consisting of the inverse images of opens from the discrete spaces $ I_{\alpha}^{\disc} $, it follows that it admits a base of clopens.

	In particular, if $ S $ is any set, then $ \beta(S) = \lim_{I \in \Fin_{S/}} I $ is a Stone space.
\end{exm}

\begin{lem}
	Any Stone space is the inverse limit of its finite discrete quotients.
\end{lem}

\begin{proof}
	Let $ K $ be a Stone space.
	The category $ \Fin_{K/} $ of finite sets to which $ K $ maps (in $ \Comp $ is an \defn{ inverse } category -- i.e., the opposite of a filtered category.
	Limit-cofinal in $ \Fin_{K/} $ is the full subcategory spanned by the finite discrete quotients.
	Hence we aim to show that the natural continuous map
	\[
		p \colon K \to \lim_{I \in \Fin_{K/}} I
	\]
	is a homeomorphism.
	Since both source and target are compact Hausdorff topological spaces, it suffices to prove that $ p $ is a bijection.
	For this, let $ x = \{ x_I \}_{I \in \Fin_{K/}} $ be a point of the limit.
	For any finite discrete quotient $p_I \colon K \to I $, let $ W_I $ be the clopen set $ p_I^{-1}(x_I) $;
	each of these is clopen, and the claim now is that the intersection
	\[
		W \coloneq \bigcap_I W_I
	\]
	consists of exactly one point of $ K $.
	Since $ K $ is quasicompact, it follows that $ W $ is nonempty.
	Since $ K $ is totally disconnected and Hausdorff, it follows that if $ x \neq y $, there exists a continuous map to $\{ 0,1 \}^{\disc}$ such that $ x \mapsto 0 $ and $ y \mapsto 1 $;
	hence $ W $ contains at most one point.
\end{proof}

\begin{nul}
	In particular, $ \Stone $ is the smallest full subcategory of $ \Top $ that contains $ \Fin $ and is closed under inverse limits.
\end{nul}

Inverse limits of compacta are exceptionally well behaved.
A key lemma that demonstrates this is the following.

\begin{lem} \label{emptyinverselimitofcompacta}
	Let $ \{ K_{\alpha} \}_{\alpha \in A^{\op}} $ be an inverse system of compacta, and assume that the inverse limit is empty.
	Then one of the $ K_{\alpha} $ is empty as well.
\end{lem}

\begin{proof}
	Let $ K $ be the product $\prod_{\alpha \in A} K_{\alpha}$;
	by Tychonoff it is compact.
	For any $ \beta \in A $, consider the subset
	\[
		Z_{\beta} \coloneq \left\{ (x_{\alpha})_{\alpha \in A} \in K : \forall \beta \to \alpha,\ \phi_{\alpha\beta} (x_{\alpha}) = x_{\beta} \right\} \period
	\]
	The subsets $Z_{\beta} \subseteq X$ are closed by Hausdorffness, and the intersection $\bigcap_{\beta \in A} Z_{\beta}$ is the limit of the $ K_{\alpha} $, which is empty.
	By compactness and the filteredness of $ A $, there exists an index $ \beta $ for which $ Z_{\beta} $ is empty.

	On the other hand, $Z_{\beta}$ is in bijection with $ K_{\beta} \times L_{\beta} $, where $ L_{\beta} $ is the product of $ K_{\gamma} $ over those $ \gamma \in A $ such that $ \gamma $ does not receive a map from $ \beta $.
	Thus one of these copies of $ K_{\alpha} $ is empty.
\end{proof}

\begin{exm}
	The compactness condition is necessary in the previous lemma.
	For instance, consider the inverse system
	\begin{equation*}
		\begin{tikzcd}[sep=1.5em]
			\cdots \arrow[r, "s", hooked] & \NN^{\disc} \arrow[r, "s", hooked] &  \NN^{\disc} \arrow[r, "s", hooked] & \NN^{\disc}
		\end{tikzcd}
	\end{equation*}
	where $ s \colon \NN \inclusion \NN $ is the successor function.
	Its limit is empty.
\end{exm}

\begin{lem}
	Any finite discrete set is cocompact as an object of $ \Comp $.
	Consequently, the fully faithful functor $ \Fin \inclusion \Comp $ extends to a limit-preserving fully faithful functor $ \Pro(\Fin) \inclusion \Comp $ whose essential image is $ \Stone $.
\end{lem}

\begin{proof}
	Let $ \{ K_{\alpha} \}_{\alpha \in A^{\op}} $ be an inverse system of compacta, and let $ I $ be a finite set.
	Write $ K \coloneq \lim_{\alpha \in A^{\op}} K_{\alpha} $;
	the claim is that the map $ \colim_{\alpha \in A^{\op}} \Map(K_{\alpha}, I^{\disc}) \to \Map(K, I^{\disc}) $ is a bijection.

	For any topological space $ X $, a continuous map $ X \to I^{\disc} $ is the same thing as a partition of $ X $ into clopens indexed by the elements of $ I $.
	Hence by induction, it suffices to show:
	\begin{enumerate}[(1)]
		\item that every clopen $ V \subseteq K $ into two complementary clopens is the inverse image of some clopen of $ V_{\alpha} \subseteq K_{\alpha} $, and
		\item that if clopens $ V_{\alpha} \subseteq K_{\alpha} $ and $ V_{\beta} \subseteq V_{\beta} $ pull back to the same $ V \subseteq K $, then there are maps $ \alpha \to \gamma $ and $ \beta \to \gamma $ in $ A $ such that $ V_{\alpha} $ and $ V_{\beta} $ pull back to the same subset of $ K_{\gamma} $.
	\end{enumerate}

	For the first claim, consider a clopen $ V \subseteq K $.
	Since $ K $ has the inverse limit topology, $ V $ is a union of open sets of the form $ V_{\gamma} $, where $ V_{\gamma} $ is pulled back from an open $ K_{\gamma} $.
	But since $ V $ is also closed, it is quasicompact, and therefore by the filteredness of $ A $ there is a single $ \gamma \in A $ such that $ V $ is pulled back from $ V_{\gamma} $.
	The same analysis of the complement of $ V $ exhibits it as the pullback from some $ K_{\beta} $;
	now letting $ \alpha \in A $ be an object that receives maps from both $ \beta $ and $ \gamma $ completes the proof.

	For the second claim, suppose that clopens $ V_{\alpha} \subseteq K_{\alpha} $ and $ V_{\beta} \subseteq V_{\beta} $ pull back to the same $ V \subseteq K $.
	For any object $ \gamma \in A $ that receives morphisms from both $ \alpha $ and $ \beta $, let $ V_{\alpha\gamma} \subseteq K_{\gamma} $ denote the inverse image of $ V_{\alpha} $, and let $ V_{\beta\gamma} \subseteq K_{\gamma} $ denote the inverse image of $ V_{\beta} $.
	Let $D_{\gamma} \subseteq K_{\gamma} $ be the symmetric difference of $ V_{\alpha\gamma} $ and $ V_{\beta\gamma} $;
	its inverse image in $ K $ is empty.
	\Cref{emptyinverselimitofcompacta} now implies that for some index $\gamma$, the set $ D_{\gamma} $ is empty, whence $ V_{\alpha\gamma} = V_{\beta\gamma}$.
\end{proof}

\begin{dfn}
	By a \defn{projective compactum} we mean a projective object in compacta.
	That is, a compactum $ K $ is projective if and only if, for any surjection $ Y \surjection Z $, the map $ \Map(K, Y) \to \Map(K, Z) $ is also a surjection.
\end{dfn}

\begin{lem}
	The following are equivalent for a compactum $ K $.
	\begin{itemize}
		\item $ K $ is a retract of a free compactum $ \beta(S) $.
		\item $ K $ is projective.
	\end{itemize}
\end{lem}

\begin{proof}
	Since a set $ S $ is a projective object of $ \Set $, it follows that the free compactum $ \beta(S) $ is a projective compactum,
	and since surjections are stable under retracts, it follows that projective compacta are stable under retracts.

	Conversely, let $ K $ be a projective compactum, and let $ \lambda_K \colon \beta(K) \surjection K $ be the structure map.
	Since $ K $ is projective, $ \lambda_K $ admits a section, whence $ K $ is a retract of a free compactum.
\end{proof}

\begin{nul}
	In particular, please note that any projective compactum is a Stone space.
\end{nul}

\begin{dfn}
	A topological space $ X $ is \defn{extremally disconnected} if and only if any closure of an open subset is open.
\end{dfn}

\begin{nul}
	Taking complements, we see that a topological space $ X $ is extremally disconnected if and only if the interior of a closed subset is closed.
\end{nul}

\begin{nul}
	It is quite difficult to construct interesting examples of extremally disconnected topological spaces.
	Any metric space that is extremally disconnected is in fact discrete.
	The next lemma provides the main source of these examples.
\end{nul}

\begin{lem}
	If $ X $ is an extremally disconnected topological space, then if $ \{ Z_1, \dots, Z_n \} $ is a finite family of closed subsets that cover $ X $, then the interiors $ \iota(Z_1), \dots, \iota(Z_n) $ cover $ X $ as well.
\end{lem}

\begin{proof}
	Assume that $ 1 \leq i \leq n $ and that $ \{ \iota(Z_1), \dots, \iota(Z_{i-1}), Z_i, \dots, Z_n \} $ cover $ X $.
	Then since $ \iota(Z_1) \cup \dots \cup \iota(Z_{i-1}) \cup Z_{i+1} \cup \dots \cup Z_n $ is closed, it follows that $ \{ \iota(Z_1), \dots, \iota(Z_i), Z_{i+1}, \dots, Z_n \} $ cover $ X $ as well.
\end{proof}

\begin{prp}
	The following are equivalent for a compactum $ K $.
	\begin{itemize}
		\item $ K $ is projective.
		\item $ K $ is extremally disconnected as a topological space.
	\end{itemize}
\end{prp}

\begin{proof}
	Assume that $ K $ is projective, and let $ U \subseteq K $ be an open subset.
	Let $ Z $ be the complement of $ U $, and let $ V $ be its closure.
	The composite $ \phi $ of the inclusion $ Z \sqcup V \inclusion K \sqcup K $ followed by the fold map $ \nabla \colon K \sqcup K \to K $ is a surjection, so since $ K $ is projective, it admits a section $ \sigma \colon K \to Z \sqcup V $.
	For any $ x \in U $, one has $ \sigma(x) = x $, and by continuity the same holds for any $ x \in V $.
	Thus $ \sigma^{-1}(V) = V $, so since $ V $ is open in $ Z \sqcup V $, it follows that $ V $ is open in $ K $.

	Conversely, assume that $ K $ is extremally disconnected, assume that $ X \surjection Y $ is a surjection between compacta, and assume that $ f \colon K \to Y $ is a continuous map.
	A lift of $ f $ is the same thing as a section of the projection $ p \colon P \coloneq X \times_Y K \surjection K $.
	In other words, it suffices to prove the existence of a closed subset $ W \subseteq P $ such that $ p $ restricts to a homeomorphism $ W \equivalence K $.
	Consider the set of closed subsets $ W' \subseteq P $ such that $ p(W') = K $;
	Zorn's lemma ensures that this collection contains a minimal element $ W $.
	To show that $ p $ restricts to a homeomorphism on $ W $, it suffices to show that $ p $ restricts to an injection.

	Let $ x \neq y $ be distinct points of $ W $.
	Choose closed subsets $ E $ and $ F $ that cover $ W $ such that $ x \notin F $, and $ y \notin E $.
	The sets $ p(E) $ and $ p(F) $ cover $ K $.
	Since $ K $ is extremally disconnected, it follows that the interiors $ \iota(p(E)) $ and $ \iota(p(F))) $ also cover $ K $.

	So to prove that $ p(x) \neq p(y) $, we shall show that $ p(x) \notin \iota(p(F)) $, and that $ p(y) \notin \iota(p(E)) $.
	Without loss of generality it suffices to prove the first claim.
	Suppose that $ B \subseteq K $ is a closed subset such that $ B \cup p(F) = K $;
	we aim to show that $ p(x) \in B $.
	Indeed, $ p(p^{-1}(B) \cup F) = K $, so the minimality of $ W $ implies that $ p^{-1}(B) \cup F = W $, whence $ x \in p^{-1}(B) $.
\end{proof}
	

%-------------------------------------------------------------------%
\subsection{Compactly generated topological spaces}
%-------------------------------------------------------------------%

