%-------------------------------------------------------------------%
%-------------------------------------------------------------------%
\section{Pyknotic abelian groups}
%-------------------------------------------------------------------%
%-------------------------------------------------------------------%

%-------------------------------------------------------------------%
\subsection{Basic structures}
%-------------------------------------------------------------------%

\begin{dfn}
	Let $ C $ be a category with all finite products.
	A \defn{pyknotic object} of $ C $ is a functor
	\[
		X \colon \Compproj^{\op} \to  C
	\]
	that carry finite coproducts of projective compacta to products in $ C $.
	Write $ \Pyk(C) $ for the category of pyknotic objects of $ C $.
\end{dfn}

\begin{nul}
	A \emph{pyknotic abelian group} can be regarded as a sheaf of abelian groups on $ \Comp $ or $ \CompStone $.
	Equally, it can be regarded as an abelian group object of pyknotic sets.
	In any case, it follows formally that the category $ \Pyk(\Ab) $ of pyknotic abelian groups is an abelian category with the following properties:
	\begin{itemize}
		\item $ \Pyk(\Ab) $ is presentable;
		\item all colimits and limits exist (AB$3$ and AB$3^{\ast}$);
		\item coproducts, products, and filtered colimits are all exact (AB$4$, AB$4^{\ast}$, and AB$5$);
		\item products commute with filtered colimits (AB$6$); and
		\item $ \Pyk(\Ab) $ has enough injectives.
	\end{itemize}
\end{nul}

\begin{exm}
	Let $ A $ be an (abstract) abelian group.
	Then $ A^{\disc} $ and $ A^{\indisc} $ are each pyknotic abelian groups.
	The forgetful functor $ \Pyk(\Ab) \to \Ab $ given by $ B \mapsto B^{\und} $ thus admits both a left adjoint $ A \mapsto A^{\disc} $ and a right adjoint $ A \mapsto A^{\indisc} $.
\end{exm}

\begin{prp}
	Limits and colimits in $ \Pyk(\Ab) $ are computed objectwise when viewed as a functor $ \Compproj^{\op} \to \Ab $.
\end{prp}

\begin{proof}
	This is true for limits since $ \Pyk(\Ab) $ is a category of sheaves.
	For colimits, it suffices to show that any colimit in $ \Fun(\Compproj^{\op}, \Ab) $ of a diagram of pyknotic abelian groups is again a pyknotic abelian group.
	Since sifted colimits commute with products (in $ \Ab $), it follows that our claim is true for sifted colimits.
	At the same time, finite coproducts in $ \Fun(\Compproj^{\op}, \Ab) $ are finite products, and the same is true in $ \Pyk(\Ab) $.
	That completes the proof.
\end{proof}

\begin{cor}
	A pyknotic homomorphism $ B \to A $ is an epimorphism if and only if, for any projective compactum $ P $, the homomorphism $ B(P) \to A(P) $ is an epimorphism.
\end{cor}

\begin{exm}
	If $ Y $ is a pyknotic set, then we define the free pyknotic abelian group $ \ZZ \{ Y \} $ as the sheafifcation of the assignment $ P \mapsto \ZZ \{ Y \}(P) $, where $ \ZZ \{ Y(P) \} $ is the free abelian group generated by the set $ Y(P) $.

	This has the following universal property: for any pyknotic abelian group $ A $, one has a natural isomorphism
	\[
		\Hom(\ZZ \{ Y \}, A) \cong \Map(Y, A) \period
	\]
	In other words, $ Y \mapsto \ZZ \{ Y \} $ is left adjoint to the forgetful functor $ \Pyk(\Ab) \to \Pyk(\Set) $.

	In particular, for any compactum $ K $, the abelian group of pyknotic homomorphisms $ \ZZ \{ K \} \to A $ is isomorphic to the value $ A(K) $.
	For discrete pyknotic sets, one has $ \ZZ \{ S^{\disc} \} \cong {\ZZ \{ S \}}^{\disc}$. 
\end{exm}

\begin{exm}
	Let $ P $ be a projective compactum.
	For any epimorphism $ B \surjection A $ of pyknotic abelian groups, the map
	\[
		\Hom(\ZZ \{ P \}, B) \cong B(P) \surjection A(P) \cong \Hom(\ZZ \{ P \}, A)
	\]
	is an epimorphism as well.
	Thus it follows that $ \ZZ \{ P \} $ is a projective object of $ \Pyk(\Ab) $.
\end{exm}

\begin{prp}
	The abelian category $ \Pyk(\Ab) $ has enough projectives.
\end{prp}

\begin{proof}
	Any pyknotic abelian group $ A $ admits a cover $ \{ P_i \} $ by projective compacta in $ \Pyk(\Set) $.
	Hence there exists an epimorphism
	\[
		\bigoplus_{i \in I} \ZZ \{ P_i \} \surjection A \period \qedhere
	\]
\end{proof}

\begin{dfn}
	A \defn{topological group} is a group object in $ \Top $.
	A \defn{compactly generated group} is a group object in the category $ \Topcg $.
	We write $ \Top(\Ab) $ and $ \Topcg(\Ab) $ for the category of topological abelian groups and that of compactly generated abelian groups, respectively.
\end{dfn}

\begin{wrn}
	A topological group $ G $ whose underlying topological space is compactly generated is always a compactly generated group:
	indeed, the multiplication $ G \times^0 G \to G $ can be composed with the natural continuous map $ G \times G \to G \times^0 G $.

	A compactly generated group $ G $ need not be a topological group, since nothing ensures that the multiplication $ G \times G \to G $ factors through $ G \times^0 G $.
	For example, let $ A \coloneq \Map(\RR^{\disc}, \RR) $, a compactly generated group under the pointwise product;
	it is not the case that the multiplication $A \times^0 A \to A $ is continuous relative to the product topology.
	
	If $ G $ is a topological group, the group $ kG $ is  a compactly generated group, but there are compactly generated groups that are not of this form.
	
	If $ G $ is a compactly generated abelian group whose topological space is locally compact, first countable, or locally $ k_{\omega} $, then $ G $ is a topological group as well.
\end{wrn}

\begin{cnstr}
	The Yoneda embedding $ \Topcg \inclusion \Pyk(\Set) $ preserves limits, whence it lifts to a fully faithful, limit-preserving functor $ \Topcg(\Ab) \inclusion \Pyk(\Ab) $.
\end{cnstr}

\begin{exm}
	Let $ A = \{ F_{\alpha} \} $ be a profinite abelian group, exhibited as a limit of finite abelian groups.
	Then we may form the limit
	\[
		A = \lim_{\alpha \in A^{\op}} F_{\alpha}
	\]
	in $ \Pyk(\Ab) $, which is not discrete.
	This agrees with what happens when we regard $ A $ as a compactly generated group, and take its corresponding pyknotic abelian group.
\end{exm}

\begin{cnstr}
	The category $ \Pyk(\Ab) $ is, by general principles, a symmetric monoidal closed category.
	The unit for the tensor product is the discrete pyknotic abelian group $ \ZZ^{\disc} $.
	The tensor product of two pyknotic abelian groups $ A $ and $ B $ is given as the sheafification of the assignment $ P \mapsto A(P) \otimes B(P) $.
	Thus a pyknotic homomorphism $ A \otimes B \to C $ is a pyknotic map $ A \times B \to C $ such that, for each projective compactum, the map $ A(P) \times B(P) \to C(P) $ is bilinear.

	The internal Hom $ \Hom(A, B) $ admits a couple of helpful descriptions.
	First, one can think of it as a subsheaf of $ \Map(A, B)$ that carries $ P $ to the abelian group $ \Hom(A|P, B|P) $ of morphisms of sheaves of abelian groups on $ \Compproj_{/P} $.
	Equivalently, $ \Hom(A, B) $ is the assignment 
	\[
		P \mapsto \Hom(A \otimes \ZZ \{ P \}, B) \period
	\]
\end{cnstr}

\begin{nul}
	Compactly generated abelian groups also have a kind of internal Hom:
	if $ A $ and $ B $ are compactly generated abelian groups, the subset $ \Hom(A, B) \subseteq \Map(A, B)$ can be endowed with its kaonised subspace topology.
	This has the following characterisation: a continuous map $ P \to \Map(A, B) $ `is' a continuous map $ \phi \colon P \times A \to B $ such that for any point $ p \in P $, the continuous map $ \phi(p, -) \colon A \to B $ is a homomorphism.
\end{nul}

Of course, we now have a potential conflict of notation, but everything is ok in the end:

\begin{prp}
	Let $ A $ and $ B $ be compactly generated abelian groups.
	Then the pyknotic abelian group $ \Hom(A, B) $ is the one attached to the compactly generated abelian group $ \Hom(A, B) $.
\end{prp}

\begin{proof}
	The statement is that the following are equivalent:
	\begin{itemize}
		\item for any projective compactum $ P $ and any continuous map $ \phi \colon P \times  A \to B $, the corresponding morphism $ A|P \to B|P $ of sheaves on $ \Compproj_{/P} $ is a homomorphism  of abelian groups;
		\item for any projective compactum $ P $ and for any point $ p \in P $, the continuous map $ \phi(p, -) \colon A \to B $ is a group homomorphism. \qedhere
	\end{itemize}
\end{proof}

%-------------------------------------------------------------------%
\subsection{Some cohomology computations}
%-------------------------------------------------------------------%

Cohomology of pyknotic sets is now defined as the usual cohomology of objects of a topos:

\begin{dfn}
	Let $ Y $ be a pyknotic set, and let $ A $ be a pyknotic abelian group.
	Then the \defn{cohomology} of $ Y $ with coefficients in $ A $ is given by
	\[
		H^i(Y, A) \coloneq \Ext^i_{\Pyk(\Ab)}( \ZZ\{ Y \}, A) \period
	\]
\end{dfn}

\begin{exm}
	$ H^0(Y, A) $ is the usual abelian group $ \Map(Y, A) $ of `functions' on $ Y $.
\end{exm}

\begin{exm}
	Let $ I $ be a finite set.
	Any effective epimorphism $ J \to I $ is split, whence the Cech complex
	\[
		H^0(I, A^{\disc}) \to H^0(J, A^{\disc}) \to H^0(J \times_{I} J, A^{disc}) \to \cdots
	\]
	is exact.

	Now if $ K $ is a Stone space, then any effective epimorphism $ L \surjection K $ from another Stone space can be presented as an inverse limit of surjections $ \{ L_{\alpha} \surjection K_{\alpha} \}_{\alpha \in \Lambda^{\op}} $ of finite sets.
	Now for each $ \alpha \in \Lambda $, the Cech complex
	\[
		H^0(K_{\alpha}, A^{disc}) \to H^0(L_{\alpha}, A^{\disc}) \to H^0(L_{\alpha} \times_{K_{\alpha}} L_{\alpha}, A^{\disc}) \to \cdots
	\]
	is exact, and so its filtered colimit
	\[
		H^0(K, A^{\disc}) \to H^0(L, A^{\disc}) \to H^0(L \times_{K} L, A^{\disc}) \to \cdots
	\]
	is exact as well.
	This proves that $ H^i(K, A^{\disc}) = 0 $ unless $ i = 0 $.
\end{exm}



%-------------------------------------------------------------------%
\subsection{The derived categories of pyknotic abelian groups}
%-------------------------------------------------------------------%

\begin{cnstr}
	We are entitled to construct an array of derived \categories\ of pyknotic abelian groups.
	Happily, they all may be expressed as pyknotic objects of the various derived \categories\ of abelian groups:
	\[
		D_{\pyk}^?(\Ab) = \Pyk(D^?(\Ab)) 
	\]
	where $ ? \in \{ {\geq m}, {\leq n}, {[m, n]}, {b}, {+}, {-}, {\textit{all}} \} {[a, b]}, {+}, {-}, {\textit{all}} \} $.
	These stable \categories\ all admit t-structures with the boundedness/completeness conditions suggested by the notation. 
\end{cnstr}











