%!TEX root = rootfile.tex

%-------------------------------------------------------------------%
%-------------------------------------------------------------------%
\section*{Introduction}\addcontentsline{toc}{section}{Introduction}
%-------------------------------------------------------------------%
%-------------------------------------------------------------------%

These are notes for a series of talks at MSRI about the \emph{pyknotic formalism}.

The pyknotic formalism offers a way to coordinate `topological' and `derived' structures.
This formalism is only partially developed, but it’s already clear that there is a lot to explore, and a lot of interesting possible applications.
Many different points of view on pyknotic objects will be necessary to develop these applications.

\subsection*{What we're trying to achieve}

\begin{exm*}
	The study of field theories necessitates often the use of infinite-dimensional vector spaces, such as the space of distributions on a smooth manifold.
	Of course, these are locally convex topological spaces, and the functional analysis literature is full of techniques to deal with these structures.
	At the same time, modern tools like the Batalin--Vilkovisky formalism require the deployment of homological algebra.
	So a natural question arises: how does one \emph{do} homological algebra with the kinds of topological vector spaces that arise in these situations?

	The literature is full of examples in which one manages to work with complexes of locally convex topological vector spaces by contemplating the derived \category of a quasiabelian category.
	But for more sophisticated applications, one has to go further:
\end{exm*}