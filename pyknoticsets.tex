%!TEX root = rootfile.tex

%-------------------------------------------------------------------%
%-------------------------------------------------------------------%
\section{Pyknotic sets}
%-------------------------------------------------------------------%
%-------------------------------------------------------------------%

%-------------------------------------------------------------------%
\subsection{Definitions}
%-------------------------------------------------------------------%

\begin{cnstr}
	A continuous map of compacta $ K \to L $ is a surjection if and only if
	it is a quotient map.
	It is in that case an \emph{effective epimorphism};
	that is, the sequence
	\[
		K \times_L K \parallelto K \to L
	\]
	is a coequaliser.

	A pullback of a surjection in $ \Comp $ is again a surjection.
	The coproduct of a finite collection of surjections in $ \Comp $ is a surjection.
	Consequently, there is a Grothendieck topology on $ \Comp $ in which the covering sieves are families generated by surjections $ K \to L $.
	We call this the \defn{effective epimorphism topology}.
\end{cnstr}

\begin{nul}
	A sheaf (of sets) for the effective epimorphism topology is a functor $ F \colon \Comp^{\op} \to \Set $ such that for any surjection $ K \to L $, the sequence
	\[
		F(L) \to F(K) \parallelto F(K \times_L K)
	\]
	is an equaliser.
\end{nul}

\begin{wrn}
	As written, the object sets of all these categories is a proper class.
	Consequently, to deal honestly with categories of sheaves on these sites requires a little set-theoretic care.

	We work within \textsc{zfcu} -- Zermelo--Fraenkel set theory along with Grothendieck's \defn{Axiom of Universes}.
	This is the requirement that any cardinal is dominated by a strongly inaccessible cardinal (which we always take to be uncountable).
	For any strongly inaccessible cardinal $ \delta $, a set that is in bijection with one whose rank is strictly less than $ \delta $ is said to be \defn{$ \delta $-small}.
	More generally, we shall be slightly indolent in our usage, and we shall say that an abelian group, space, spectrum, category, \category, \emph{etc}., is \defn{$\delta $-small} if and only if it is \emph{equivalent} (in whatever sense is appropriate) to one whose underlying set is $ \delta $-small.

	The strongly inaccessible cardinals are linearly ordered:
	\[
		\delta_0 < \delta_1 < \cdots \period
	\]
	We will say that a set is \defn{tiny} if and only if it is in bijection with a $ \delta_0 $-small set;
	we will say that it is \defn{small} if and only if it is $ \delta_1 $-small;
	and we will say that it is \defn{large} if and only if it is $ \delta_2 $-small.%
	\footnote{This terminology isn't really ideal;
	if someone else has an alternative proposal, please tell me! --- CB}

	We will also introduce some shorthand notations:
	\begin{itemize}
		\item $ \Comp $, $ \CompStone $, $ \Compproj $, and $ \Compfree $ will be taken to mean the categories of tiny compacta, Stone spaces, projective compact, and free compacta (respectively).
		\item $ \Set $ will be taken to mean the category of small sets.
		\item $ \Top $, $ \Topcg $, and $ \Topcgwh $ will be taken to mean the categories of small topological spaces, compactly generated topological spaces, and compactly generated weak Hausdorff topological spaces (respectively).
	\end{itemize}
\end{wrn}

\begin{dfn}
	A \defn{pyknotic set} is a sheaf $ F \colon \Comp^{\op} \to \Set $ for the effective epimorphism topology.
	Thus the category $ \Pyk(\Set) $ of pyknotic sets is the full subcategory of $ \Fun(\Comp^{\op}, \Set) $ spanned by the sheaves for the effective epimorphism topology.
	Observe that $ \Pyk(\Set) $ is a $ \delta_1 $-presentable category.
\end{dfn}

\begin{lem}
	The full subcategories
	\[
		\Compfree \subseteq \Compproj \subseteq \CompStone \subseteq \Comp
	\]
	are bases for the effective epimorphism topology.
\end{lem}

\begin{proof}
	This follows from the fact that any compactum $ K $ is a quotient of the free compactum $ \beta(K^{\disc}) $.
\end{proof}

\begin{nul}
	Thus a pyknotic set may be described as a sheaf for the effective epimorphism topology on $ \CompStone $, $ \Compproj $, or $ \Compfree $.
	This fact has a number of useful consequences.

	First, the identification $ \CompStone \simeq \Pro(\Fin) $ identifies the effective epimorphism site of $ \CompStone $ with Bhatt and Scholze's proétale topology on a geometric point.
	Thus the category of pyknotic sets is the proétale topos of a geometric point.

	In $ \Compproj $, every surjection is split.
	Consequently, a functor $ F \colon \Compproj^{\op} \to \Set $ is a sheaf for the effective epimorphism topology if and only if it carries finite coproducts of projective compacta to products:
	\[
		F( K \sqcup L ) \equivalence F ( K ) \times F ( L ) \period
	\]
	The projective compacta are thus a collection of compact projective generators for the category $ \Pyk(\Set) $.

	Finally, $ \Compfree $ is the Kleisli category for the ultrafilter monad $ \beta $.
	Consequently, a pyknotic set $ F \colon \Compfree^{\op} \to \Set $ can be regarded as an algebra $ F_{\beta} $ for the `infinitary Lawvere theory' defined by $ \beta $:
	it carries any tiny set to a set $ F_{\beta}(S) $, and it carries any map $ S \to \beta(T) $ to a map $ F_{\beta}(T) \to F_{\beta}(S) $, all in such a way as to ensure that $ F_{\beta}(S \sqcup T) \cong F_{\beta}(S) \times F_{\beta}(T) $.
\end{nul}

\begin{exm}
	Any topological space $ X $ defines, via Yoneda, a pykontic set, which we will also denote $ X $:
	\[
		X(K) \coloneq \Map(K, X) \comma
	\]
	for any compactum $ K $.
\end{exm}

\begin{prp}
	The assignment above defines a fully faithful functor $ \Topcg \inclusion \Pyk(Set) $.
\end{prp}

\begin{proof}
	That the functor is fully faithful is the very definition of compact generation.
\end{proof}

\begin{cnstr}
	Let $ Y $ be a pyknotic set.
	Then $ Y $ can be exhibited as the canonical colimit of the compacta that map to it:
	\[
		Y \simeq \colim_{K \in \Comp_{/Y}} K \period
	\]
	We define a topological space $ Y^{\textit{top}} $ by forming this same colimit in the category $ \Top $.
	Since it is a colimit of compacta, $ Y^{\textit{top}} $ is a compactly generated topological space.
	The topological space $ Y^{\textit{top}} $ is thus the set $ Y(\ast) $ equipped with the quotient topology from the surjection
	\[
		\coprod_{\beta(S) \in \Compfree_{/Y}} \beta(S) \surjection Y(\ast)
	\]
	This colimit defines a left adjoint $ Y \mapsto Y^{\textit{top}} $ to the fully faithful functor $ \Topcg \inclusion \Pyk(Set) $.

	Thus compactly generated topological spaces form a locaisation of the category of pyknotic sets.
	Consequently, limits formed in $ \Topcg $ are the same as those formed in $ \Pyk(\Set) $.
	
	The inclusion $ \Topcgwh \inclusion \Pyk(\Set) $ also admits a left adjoint that carries $ Y \mapsto h(Y^{\textit{top}}) $.
	Thus limits formed in $ \Topcgwh $ are also the same as those formed in $ \Pyk(\Set) $.
\end{cnstr}

\begin{nul}
	If $ X \colon A \to \Pyk(\Set) $ is a filtered diagram of pyknotic sets, then the colimit is given objectwise:
	\[
		(\colim_{\alpha \in A} X_{\alpha})(K) \cong \colim_{\alpha \in A} X_{\alpha}(K) \period
	\]
	Note that the filtered colimit of compactly generated topological spaces is not preserved by the inclusion into pyknotic sets;
	in particular, compacta are compact objects in $ \Pyk(\Set) $, but not in $ \Topcg $.
\end{nul}

\begin{exm}
	For any set $ S $, there is both a discrete and an indiscrete pyknotic set attached to $ S $.
	The indiscrete pyknotic set attached to $ S $ is given by the formula
	\[
		S^{\indisc}(K) \cong \Map(K, S^{\indisc}) \cong \prod_{k \in |K|} S \comma
	\]
	where $ |K|$ denotes the set underlying $ K $. 
	
	The discrete pyknotic set attached to $ S $ is given by the formula
	\[
		S^{\disc}(K) \cong S^K \coloneq \colim_{\alpha \in A} \prod_{k \in K_{\alpha}} S \comma
	\]
	where $ K = \lim_{\alpha \in A^{\op}} K_{\alpha} $ is a Stone space, exhibited as an inverse limit of finite sets.
	Please observe that this is the right Kan extension along $ \{ \ast \} \inclusion \Fin^{\op} $ followed by a left Kan extension along $ \Fin^{\op} \inclusion \CompStone^{\op} $.
	Thus the discrete pyknotic sets $ Y $ are exactly those that are left Kan extended from a functor $ \Fin^{\op} \to \Set $ that is itself right Kan extended from the point.
\end{exm}

\begin{dfn}[Clausen \& Scholze]
	A pyknotic set $ X $ is a \defn{condensed set} if its values are all tiny, and the functor
	\[
		X \colon \Ind(\Fin^{\op}) \simeq \Pro(\Fin)^{\op} \to \Set_{\delta_0}
	\]
	is accessible (relative to the tiny universe).
\end{dfn}

\begin{exm}
	The Sierpinski space $ \{s, \eta \} $, in which $ \{\eta\} $ is open but $ \{ s \} $ is not, is not a condensed set.
	In a similar vein, indiscrete pyknotic sets on sets with more than two points are never condensed.
\end{exm}

\begin{cnstr}
	Since $ \Pyk (\Set) $ is a topos, it follows readily that there is an internal Hom between two pyknotic sets;
	that is, the functor $ - \times Y $ admits a right adjoint $ \Map( Y, - ) $, where $ \Map(Y, Z) $ is given by the assignment
	\[
		P \mapsto \Map(Y \times P, Z) \cong \Map(Y|P, Z|P) \comma
	\]
	where $ Y|P $ and $ Z|P $ are the restrictions of $ Y $ and $ Z $ to $ \Compproj_{/P} $.
\end{cnstr}

For compactly generated topological spaces, this presents a potential conflict of notation, but in fact all is well:
\begin{prp}
	Let $ X $ and $ Y $ be compactly generated topological spaces.
	Then the pyknotic set $ \Map(X, Y) $ is the one attached to the compactly generated topological space $ \Map(X, Y) $.
\end{prp}

%-------------------------------------------------------------------%
\subsection{Quasicompact and quasiseparated}
%-------------------------------------------------------------------%

\begin{nul}
	The category $ \Comp $ is a pretopos, and the topos $ \Pyk(\Set) $ is the corresponding topos.
	In particular, that means that compacta are exactly the same as the \emph{coherent} pyknotic sets -- that is, those that are quasicompact and quasiseparated.
	Though this is a complete proof, it is actually advantageous to be a little more direct about this.
\end{nul}

\begin{nul}
	In a topos $ \XX $, a \defn{covering} $ {\{U_i\}}_{i \in I} $ of an object $ X $ is a family of objects of $ \XX_{/X} $ such that the morphism
	\[
		\coprod_{i \in I} U_i \surjection X
	\]
	is an effective epimorphism.
	We say that $ {\{U_i\}}_{i \in I} $ \defn{covers} $ X $.

	The object $ X $ is \defn{quasicompact} if and only if, for every covering $ {\{ U_i \}}_{i \in I} $, there exists a finite subset $ I_0 \subseteq I $ such that $ {\{ U_i \}}_{i \in I_0} $ covers as well.

	A morphism $ X \to Y $ is \defn{quasicompact} if and only if, for any quasicompact object $ L $ any any morphism $ L \to Y $, the pullback $ X \times_Y L $ is quasicompact as well.

	The object $ X $ is \defn{quasiseparated} if and only if, for any two quasicompact objects $ K $ and $ K' $, the fibre product $ K \times_X K' $ is quasicompact as well.

	Finally, we say that $ X $ is \defn{coherent} if and only if it is both quasicompact and quasiseparated.
\end{nul}

\begin{nul}
	The 

	If $ K $ is quasicompact, and if $ K \surjection L $ is an effective epimorphism, then $ L $ is quasicompact as well.
	The proof familiar from topology adapts readily to this situation.
\end{nul}

\begin{lem}
	A pyknotic set $ Y $ is quasicompact if and only if, for any covering $ {\{ P_i \}}_{i \in I} $ of $ Y $ by projective compacta, there exists a finite subset $ I_0 \subseteq I $ such that $ {\{P_i\}}_{i \in I_0} $ covers as well.
\end{lem}

\begin{proof}
	It is clear that any quasicompact object enjoys this property.
	Conversely, suppose that $ Y $ enjoys this property, and suppose that $ {\{ U_i \}}_{i \in I} $ is a covering of $ Y $. 
	Choose, for each $ i \in I $, a covering $ {\{ P_j \}}_{j \in J_i} $ comprised of projective compacta.
	
	Write $ J = \coprod_{i \in I} J_i $, and let $ f \colon J \to I $ be the map that carries $ j $ to the element $ i \in I $ such that $ j \in J_i $.
	Hence the objects $ {\{ P_j \}}_{j \in J} $ cover $ Y $, 
	and by assumption, there exists a finite subset $ J_0 \subseteq J $ such that $ {\{ P_j \}}_{j \in J_0} $ cover $ Y $. 
	Now let $ I_0 \coloneq f(J_0) $;
	then $ {\{ U_i \}}_{i \in I_0} $ covers $ Y $, and the proof is complete.
\end{proof}

\begin{cor}
	Any compactum is quasicompact as a pyknotic set.
\end{cor}

\begin{lem}
	A pyknotic set $ Y $ is quasiseparated if and only if, for any projective compacta $ P $ and $ P' $ and any morphism $ P \to Y $ and $ P' \to Y $, the pullback $ P \times_Y P' $ is quasicompact.
\end{lem}

\begin{proof}
	It is clear that any quasiseparated pyknotic set enjoys this property.
	Conversely, assume that $ Y $ enjoys this property, let $ K $ and $ K' $ be quasicompact objects, and let $ K \to Y $ and $ K' \to Y $ be morphisms.
	Using quasicompacness, we may select effective epimorphisms $ P \surjection K $ and $ P' \surjection K' $.
	By assumption, $ P \times_Y P' $ is quasicompact, and the map $ P \times_Y P' \surjection K \times_Y K' $ is an effective epimorphism, so $ K \times_Y K' $ is quasicompact as well.
\end{proof}

\begin{cor}
	Any compactum is quasiseparated as a pyknotic set.
\end{cor}

\begin{prp}
	Any coherent pyknotic set is a compactum.
\end{prp}

\begin{proof}
	Let $ Y $ be a coherent pyknotic set.
	By quasicompacness there exists an effective epimorphism $ K \surjection Y $, and by quasiseparatedness the fibre product $ K \times_Y K $ is quasicompact.
	By the definition of the effective epimorphism topology, if $ K \times_Y K $ is represented by a closed subspace of $ K \times K $, then the quotient $ Y $ will be the quotient of $ K $ as computed in $ \Comp $.
	We therefore aim to show that $ K \times_Y K $ is represented by a closed subspace of $ K \times K $.

\end{proof}


