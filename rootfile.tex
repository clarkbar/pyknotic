% !TeX program = xelatex
%-*- program: xelatex -*-
%-*- encoding: utf-8 -*-
% \pdfoutput=1 % for the arXiv
\documentclass[draft=false, leqno]{article}
\usepackage{amsmath,amsthm,mathrsfs}
\usepackage[minionmath]{boilerart1}
\hypersetup{
    urlcolor = slate,
}

%-------------------------------------------------------------------%
% Typesetting                                                       %
%-------------------------------------------------------------------%


\newcommand{\invert}{\operatorname{H}}
\newcommand{\respectively}[1]{\text{\qquad(respectively,\quad}{#1}\text{\quad)}}
\newcommand{\resp}[1]{\text{\qquad(resp.,\quad}{#1}\text{\quad)}}
\newcommand{\andeq}{\text{\qquad and\qquad}}
\newcommand{\period}{\rlap{\ .}}
\newcommand{\comma}{\rlap{\ ,}}

\usepackage[style=british]{csquotes}
\newcommand{\defn}[1]{\emph{#1}}
\newcommand{\upperth}{\textsuperscript{th}\,}

%-------------------------------------------------------------------%
% Terms with an '∞' (or not)                                        %
%-------------------------------------------------------------------%

\newcommand{\category}{{$\infty$-cat\-e\-go\-ry}\xspace}
\newcommand{\categories}{{$\infty$-cat\-e\-gories}\xspace}
\newcommand{\categorical}{{$\infty$-cat\-e\-gor\-i\-cal}\xspace}
\newcommand{\Categorical}{{$\infty$-Cat\-e\-gor\-i\-cal}\xspace}
\newcommand{\acategory}{{an $\infty$-cat\-e\-go\-ry}\xspace}
\newcommand{\Acategory}{{An $\infty$-cat\-e\-go\-ry}\xspace}
\newcommand{\acategorical}{{an $\infty$-cat\-e\-gor\-i\-cal}\xspace}
\newcommand{\Acategorical}{{An $\infty$-cat\-e\-gor\-i\-cal}\xspace}

\newcommand{\groupoid}{{$\infty$-group\-oid}\xspace}
\newcommand{\groupoids}{{$\infty$-group\-oids}\xspace}
\newcommand{\agroupoid}{{an $\infty$-group\-oid}\xspace}
\newcommand{\Agroupoid}{{An $\infty$-group\-oid}\xspace}

\newcommand{\topos}{{$\infty$-topos}\xspace}
\newcommand{\topoi}{{$\infty$-topoi}\xspace}
\newcommand{\toposic}{{$\infty$-topos\-ic}\xspace}
\newcommand{\atopos}{{an $\infty$-topos}\xspace}
\newcommand{\Atopos}{{An $\infty$-topos}\xspace}
\newcommand{\Topos}{{$\infty$-Topos}\xspace}
\newcommand{\Topoi}{{$\infty$-Topoi}\xspace}

\newcommand{\pretopos}{{$\infty$-pre\-topos}\xspace}
\newcommand{\pretopoi}{{$\infty$-pre\-topoi}\xspace}
\newcommand{\pretoposic}{{$\infty$-pre\-toposic}\xspace}
\newcommand{\apretopos}{{an $\infty$-pre\-topos}\xspace}
\newcommand{\Apretopos}{{An $\infty$-pre\-topos}\xspace}

\newcommand{\site}{{$\infty$-site}\xspace}
\newcommand{\sites}{{$\infty$-sites}\xspace}
\newcommand{\asite}{{an $\infty$-site}\xspace}
\newcommand{\Asite}{{An $\infty$-site}\xspace}

\newcommand{\presite}{{$\infty$-pre\-site}\xspace}
\newcommand{\presites}{{$\infty$-pre\-sites}\xspace}
\newcommand{\apresite}{{an $\infty$-pre\-site}\xspace}
\newcommand{\Apresite}{{An $\infty$-pre\-site}\xspace}

%-------------------------------------------------------------------%
% Letters                                                           %
%-------------------------------------------------------------------%

\newcommand{\elbar}{\bar{\el}}

\newcommand{\Cech}{\check{C}}

\newcommand{\elowerstar}{e_{\ast}}
\newcommand{\eupperstar}{e^{\ast}}
\newcommand{\ilowerstar}{i_{\ast}}
\newcommand{\iupperstar}{i^{\ast}}
\newcommand{\iuppershriek}{i^{!}}
\newcommand{\jlowerstar}{j_{\ast}}
\newcommand{\jupperstar}{j^{\ast}}
\newcommand{\jlowershriek}{j_{!}}
\newcommand{\fbar}{\bar{f}}
\newcommand{\flowerstar}{f_{\ast}}
\newcommand{\fupperstar}{f^{\ast}}
\newcommand{\fuppershriek}{f^{!}}
\newcommand{\flowershriek}{f_{!}}
\newcommand{\gbar}{\bar{g}}
\newcommand{\glowerstar}{g_{\ast}}
\newcommand{\gupperstar}{g^{\ast}}
\newcommand{\guppershriek}{g^{!}}
\newcommand{\hlowerstar}{h_{\ast}}
\newcommand{\hupperstar}{h^{\ast}}
\newcommand{\huppershriek}{h^{!}}
\newcommand{\plowerstar}{p_{\ast}}
\newcommand{\pupperstar}{p^{\ast}}
\newcommand{\puppershriek}{p^{!}}
\newcommand{\plowershriek}{p_{!}}
\newcommand{\prupperstar}{\pr^{\ast}}
\newcommand{\qlowerstar}{q_{\ast}}
\newcommand{\tupperstar}{t^{\ast}}
\newcommand{\tlowerstar}{t_{\ast}}
\newcommand{\qupperstar}{q^{\ast}}
\newcommand{\xlowerstar}{x_{\ast}}
\newcommand{\xupperstar}{x^{\ast}}
\newcommand{\xuppershriek}{x^{!}}
\newcommand{\ylowerstar}{y_{\ast}}
\newcommand{\yupperstar}{y^{\ast}}
\newcommand{\yuppershriek}{y^{!}}
\newcommand{\wlowerstar}{w_{\ast}}
\newcommand{\wupperstar}{w^{\ast}}
\newcommand{\wuppershriek}{w^{!}}
\newcommand{\zlowerstar}{z_{\ast}}
\newcommand{\zupperstar}{z^{\ast}}
\newcommand{\zuppershriek}{z^{!}}
\newcommand{\philowerstar}{\phi_{\ast}}
\newcommand{\pihat}{\hat{\pi}}
\newcommand{\piupperstar}{\pi^{\ast}}
\newcommand{\pilowerstar}{\pi_{\ast}}
\newcommand{\xiupperstar}{\xi^{\ast}}
\newcommand{\xilowerstar}{\xi_{\ast}}
\newcommand{\phiupperstar}{\phi^{\ast}}
\newcommand{\betabar}{\bar{\beta}}
\newcommand{\Omegalowerstar}{\Omega_{\ast}}
\newcommand{\Omegaupperstar}{\Omega^{\ast}}
\newcommand{\Psilowerstar}{\Psi_{\ast}}
\newcommand{\Psiupperstar}{\Psi^{\ast}}
\newcommand{\tauhat}{\hat{\tau}}
\newcommand{\Ntilde}{\widetilde{N}}
\newcommand{\Xtilde}{\widetilde{X}}
\newcommand{\Ztilde}{\widetilde{Z}}

\newcommand{\Gammalowerstar}{\Gammaup_{\ast}}
\newcommand{\Gammaupperstar}{\Gammaup^{\ast}}
\newcommand{\Gammauppershriek}{\Gammaup^{!}}
\newcommand{\Gammalowershriek}{\Gammaup_{!}}

%-------------------------------------------------------------------%
% Italic text                                                       %
%-------------------------------------------------------------------%

\newcommand{\Nis}{\ensuremath{\textit{nis}}}
\renewcommand{\fin}{\,\ensuremath{\textit{fin}}}
\newcommand{\bcc}{\ensuremath{\textit{bcc}}}
\newcommand{\cart}{\ensuremath{\textit{cart}}}
\newcommand{\cocart}{\ensuremath{\textit{cocart}}}
\newcommand{\leftfib}{\ensuremath{\textit{left}}}
\newcommand{\rightfib}{\ensuremath{\textit{right}}}
\newcommand{\constr}{\ensuremath{\textit{cons}}}
\newcommand{\fconstr}{\ensuremath{\textit{fcons}}}
\newcommand{\lc}{\ensuremath{\textit{lc}}}
\newcommand{\pro}{\ensuremath{\textit{pro}}}
\newcommand{\lft}{\ensuremath{\textit{lft}}}
\newcommand{\an}{\ensuremath{\textit{an}}}
\newcommand{\proet}{\ensuremath{\textit{proét}}}
\newcommand{\perf}{\ensuremath{\textit{perf}}}
\newcommand{\red}{\ensuremath{\textit{red}}}
\newcommand{\zar}{\ensuremath{\textit{zar}}}
\newcommand{\hyp}{\ensuremath{\textit{hyp}}}
\newcommand{\post}{\ensuremath{\textit{post}}}
\newcommand{\bdd}{\ensuremath{\textit{b}}}
\newcommand{\spec}{\ensuremath{\textit{spec}}}
\newcommand{\disc}{\ensuremath{\textit{disc}}}
\newcommand{\indisc}{\ensuremath{\textit{indisc}}}
\newcommand{\loc}{\ensuremath{\textit{loc}}}
\newcommand{\qc}{\ensuremath{\textit{qc}}}
\newcommand{\vop}{\ensuremath{\textit{vop}}}
\newcommand{\trig}{\ensuremath{\textit{trig}}}
\newcommand{\all}{\ensuremath{\textit{all}}}
\newcommand{\coh}{\ensuremath{\textit{coh}}}
\newcommand{\ncoh}{n\ensuremath{\textit{-coh}}}
\newcommand{\bc}{\ensuremath{\textit{bc}}}
\newcommand{\lex}{\ensuremath{\textit{lex}}}
\newcommand{\ft}{\ensuremath{\textit{ft}}}
\newcommand{\inv}{\ensuremath{\textit{inv}}}
\newcommand{\cts}{\ensuremath{\textit{cts}}}
\newcommand{\ext}{\ensuremath{\textit{ext}}}
\newcommand{\noeth}{\ensuremath{\textit{noeth}}}
\newcommand{\sober}{\ensuremath{\textit{sober}}}
\newcommand{\sh}{\ensuremath{\textit{sh}}}
\newcommand{\hens}{\ensuremath{\textit{h}}}
\newcommand{\dec}{\ensuremath{\textit{déc}}}
\newcommand{\predec}{\ensuremath{\textit{predéc}}}
\newcommand{\pre}{\ensuremath{\textit{pre}}}
\newcommand{\toptextit}{\ensuremath{\textit{top}}}
\newcommand{\abs}{\ensuremath{\textit{abs}}}
\newcommand{\norm}{\ensuremath{\textit{norm}}}
\newcommand{\po}{\ensuremath{\textit{po}}}
\newcommand{\stextit}{\ensuremath{\textit{s}}}
\newcommand{\cc}{\ensuremath{\textit{cc}}}
\newcommand{\adm}{\ensuremath{\textit{adm}}}
\newcommand{\und}{\ensuremath{\textit{und}}}
\newcommand{\pyk}{\ensuremath{\textit{pyk}}}
\newcommand{\cg}{\ensuremath{\textit{cg}}}
\newcommand{\proj}{\ensuremath{\textit{proj}}}

\newcommand{\pt}{\ast}

%-------------------------------------------------------------------%
% Binary operators & symbols                                        %
%-------------------------------------------------------------------%

\newcommand{\cross}{\times}
\newcommand{\crosslimits}{\operatornamewithlimits{\cross}} 
\newcommand{\leftadjoint}{\shortlefttack}
\newcommand{\unionlimits}{\operatornamewithlimits{\cup}} 
\newcommand{\isomorphic}{\cong}
\newcommand{\orientedcup}{\mathbin{\overleftarrow{\cup}}}
\newcommand{\orientedcupbc}{\orientedcup_{\bc}}
\newcommand{\orientedtimes}{\mathbin{\overleftarrow{\times}}}
\newcommand{\of}{\circ}
\newcommand{\equivalent}{\simeq}
\newcommand{\paren}[1]{\left(#1\right)}
\newcommand{\leftnat}[1]{\vphantom{#1}_{\natural}\mskip-1mu{#1}}
\newcommand{\rightnat}[1]{{#1}^{\natural}}
\renewcommand{\rhd}{\smalltriangleright}
\newcommand{\angs}[1]{\langle #1 \rangle}
\newcommand{\ultimes}{\,\underline{\times}\,}
\newcommand{\horn}[2]{\upLambda_{#1}^{#2}}
\def\rnatural{{\mathpalette\reflectedop\natural}}
\def\reflectedop#1#2{\mathop{\reflectbox{$#1{#2}$}}}
\newcommand{\cosv}[1]{{\uparrow}{#1}}
\newcommand{\sv}[1]{{\downarrow}{#1}}
\newcommand{\llbracket}{⟦}
\newcommand{\rrbracket}{⟧}

%-------------------------------------------------------------------%
% Categories                                                        %
%-------------------------------------------------------------------%

\newcommand{\Deltaop}{\DDelta^{\op}}

\newcommand{\poSet}{\po\Set}
\newcommand{\poSetfin}{\poSet^{\fin}}
\newcommand{\Setfin}{\Set^{\fin}}
\newcommand{\Preord}{\categ{Preord}}

\newcommand{\Comp}{\categ{Comp}}
\newcommand{\Stn}{\categ{Stn}}
\newcommand{\Compproj}{\categ{Proj}}
\newcommand{\Compprojop}{\Compproj^{\op}}
\newcommand{\Compfree}{\categ{Free}}
\newcommand{\CompStone}{\categ{Stone}}

\newcommand{\Topcg}{\categ{CG}}
\newcommand{\Topcgwh}{\categ{CGWH}}

\newcommand{\FEt}{\textbf{\textup{FÉt}}}
\newcommand{\Fib}{\categ{Fib}}
\newcommand{\Sch}{\categ{Sch}}
\newcommand{\Schft}{\Sch^{\,\ft}}

\newcommand{\Op}{\categ{Op}}
\newcommand{\Orbital}{\categ{Orb}}
\newcommand{\PreOp}{\categ{PreOp}}

\DeclareMathOperator{\Pyk}{Pyk}
\DeclareMathOperator{\CS}{CS}
\DeclareMathOperator{\CO}{CO}

\DeclareMathOperator{\LS}{LS}
\DeclareMathOperator{\PSh}{PSh}
\DeclareMathOperator{\Sh}{Sh}
\newcommand{\Shhyp}{\Sh^{\!\hyp}}
\newcommand{\Sheff}[1]{\Sh_{\eff}(#1)}
\newcommand{\Sheffhyp}[1]{\Sh_{\eff}^{\!\hyp}(#1)}

\newcommand{\Dperf}{\categ{Perf}}
\renewcommand{\Rep}{\operatorname{Rep}}

\DeclareMathOperator{\FC}{FC}
\DeclareMathOperator{\UH}{UH}

\DeclareMathOperator{\Pup}{P}

%-------------------------------------------------------------------%
% Tildes                                                            %
%-------------------------------------------------------------------%

\newcommand{\xtilde}{\widetilde{x}}
\newcommand{\stilde}{\tilde{s}}
\newcommand{\Stilde}{\widetilde{S}}
\newcommand{\Ptilde}{\widetilde{P}}
\newcommand{\Mtilde}{\widetilde{M}}
\newcommand{\Utilde}{\widetilde{U}}
\newcommand{\Sigmatilde}{\widetilde{\Sigma}}

%-------------------------------------------------------------------%
% Math operators                                                    %
%-------------------------------------------------------------------%

\DeclareMathOperator{\Isom}{Isom}
\DeclareMathOperator{\Equival}{Equiv}
\DeclareMathOperator{\Flat}{Flat}
\DeclareMathOperator{\Gal}{Gal}
\newcommand{\GalDelta}{\Gal^{\upDelta}}
\DeclareMathOperator{\Lin}{Lin}
\DeclareMathOperator{\mat}{mat}
\DeclareMathOperator{\Pairing}{Pair}
\DeclareMathOperator{\Pro}{Pro}
\DeclareMathOperator{\rk}{rk}
\DeclareMathOperator{\Sing}{Sing}
\DeclareMathOperator{\specpos}{sp}
\DeclareMathOperator{\Tot}{Tot}
\DeclareMathOperator{\Pt}{Pt}
\DeclareMathOperator{\Locop}{Loc}
\DeclareMathOperator{\Nbd}{Nbd}
\DeclareMathOperator{\Ran}{Ran}
\newcommand{\MOR}{\ensuremath{\textup{\textsc{Mor}}}}
\newcommand{\Funlowerstar}{\Fun_{\ast}}
\newcommand{\Funupperstar}{\Fun^{\ast}}
\newcommand{\sdop}{\sd^{\op}}
\DeclareMathOperator{\rank}{rk}
\DeclareMathOperator{\ran}{ran}
\newcommand{\Functs}{\Fun^{\cts}}
\newcommand{\Funpyk}{\Functs}

\newcommand{\Funcross}{\Fun^{\cross}}

\DeclareMathOperator{\Frac}{Frac}

\newcommand{\restrict}[2]{#1|_{#2}}
\newcommand{\Union}{\bigcup}
\DeclareMathOperator{\Nerve}{N}
\newcommand{\NNerve}{\NNup}
\renewcommand{\Bar}{\operatorname{Bar}}
\DeclareMathOperator{\Hup}{H}
\DeclareMathOperator{\Oup}{O}

\DeclareMathOperator{\Frob}{Frob}
\DeclareMathOperator{\Mon}{Mon}

\DeclareMathOperator{\BC}{BC}


%-------------------------------------------------------------------%
% Coherent objects                                                  %
%-------------------------------------------------------------------%

\newcommand{\WWcoh}{\WW^{\coh}}
\newcommand{\XXcoh}{\XX^{\coh}}
\newcommand{\YYcoh}{\YY^{\coh}}
\newcommand{\ZZcoh}{\ZZ^{\coh}}
\newcommand{\WWcohbdd}{\WWcoh_{<\infty}}
\newcommand{\XXcohbdd}{\XXcoh_{<\infty}}
\newcommand{\YYcohbdd}{\YYcoh_{<\infty}}
\newcommand{\ZZcohbdd}{\ZZcoh_{<\infty}}

%-------------------------------------------------------------------%
% Other topoi                                                       %
%-------------------------------------------------------------------%

\newcommand{\XXbar}{\overline{\XX}}
\newcommand{\XXhyp}{\XX^{\hyp}}
\newcommand{\YYhyp}{\YY^{\hyp}}
\DeclareMathOperator{\Pathop}{Path}
\newcommand{\Path}[1]{\Pathop(#1)}

%-------------------------------------------------------------------%
% Lisse & constructible objects                                     %
%-------------------------------------------------------------------%

\newcommand{\lisse}{\ensuremath{\textit{lisse}}}
\newcommand{\locsys}{\ensuremath{\textit{locsys}}}
\newcommand{\oneconstr}{[1]\ensuremath{\textit{-cons}}}
\newcommand{\Pconstr}{P\ensuremath{\textit{-cons}}}
\newcommand{\Pprimeconstr}{P'\ensuremath{\textit{-cons}}}
\newcommand{\Sconstr}{S\ensuremath{\textit{-cons}}}
\newcommand{\Sprimeconstr}{S'\ensuremath{\textit{-cons}}}
\newcommand{\Mconstr}{M\ensuremath{\textit{-cons}}}
\newcommand{\Uconstr}{U\ensuremath{\textit{-cons}}}
\newcommand{\Pfconstr}{P\ensuremath{\textit{-fcons}}}
\newcommand{\Sfconstr}{S\ensuremath{\textit{-fcons}}}
\newcommand{\Pspec}{P\ensuremath{\textit{-spec}}}
\newcommand{\Sspec}{S\ensuremath{\textit{-spec}}}
\newcommand{\nspec}{[n]\ensuremath{\textit{-spec}}}

%-------------------------------------------------------------------%
% Décollages                                                        %
%-------------------------------------------------------------------%

\DeclareMathOperator{\Dec}{Déc}
\newcommand{\DecTop}[1]{\Dec_{#1}(\Topbc)}
\newcommand{\DecTopStone}[1]{\Dec_{#1}(\TopStone)}
\newcommand{\DecSpace}[1]{\Dec_{#1}(\Space)}
\newcommand{\DecSpacefin}[1]{\Dec_{#1}(\Spacefin)}
\newcommand{\DecSpaceprofin}[1]{\Dec_{#1}(\Spaceprofin)}

%-------------------------------------------------------------------%
% Oriented fibre products and pushouts                              %
%-------------------------------------------------------------------%

\newcommand{\orientedpull}[3]{#1 \orientedtimes_{#2} #3}
\newcommand{\orientedpush}[3]{#1 \orientedcup^{#2} #3}
\newcommand{\commacat}[3]{#1 \mathbin{\downarrow}_{#2} #3}
\newcommand{\Loc}[2]{{#1}_{(#2)}}

\newcommand{\near}{\upPsi}
\newcommand{\conear}{\upPsi}
\newcommand{\conearlowerstar}{\conear_{\ast}}

%-------------------------------------------------------------------%
% Rings                                                             %
%-------------------------------------------------------------------%

\newcommand{\DDelta}{\mathbfup{\Delta}}
\newcommand{\AAup}{\mathbfup{A}}
\newcommand{\CCup}{\mathbfup{C}}
\newcommand{\FFup}{\mathbfup{F}}
\newcommand{\NNup}{\mathbfup{N}}
\newcommand{\PPup}{\mathbfup{P}}
\newcommand{\QQup}{\mathbfup{Q}}
\newcommand{\UUup}{\mathbfup{U}}
\newcommand{\ZZup}{\mathbfup{Z}}
\newcommand{\FFp}{\FFup_{\!p}}
\newcommand{\FFq}{\FFup_{\!q}}
\newcommand{\QQell}{\QQup_{\el}}
\newcommand{\QQellbar}{\overline{\QQup}_{\el}}
\newcommand{\ZZell}{\ZZup_{\el}}
\newcommand{\ZZhat}{\widehat{\ZZup}}
\DeclareMathOperator{\BG}{BG}
\DeclareMathOperator{\BD}{BD}
\DeclareMathOperator{\BI}{BI}
\DeclareMathOperator{\Gup}{G}
\DeclareMathOperator{\Iup}{I}
\DeclareMathOperator{\Kup}{K}
\DeclareMathOperator{\Sup}{S}
\DeclareMathOperator{\Bup}{B\!}
\newcommand{\BZZ}{{\Bup\ZZup}}
\newcommand{\BZZhat}{{\Bup\ZZhat}}

%-------------------------------------------------------------------%
% Ideals                                                            %
%-------------------------------------------------------------------%

\newcommand{\mfrak}{\mathfrak{m}}
\newcommand{\pfrak}{\mathfrak{p}}

%-------------------------------------------------------------------%
% Shapes                                                            %
%-------------------------------------------------------------------%

\newcommand{\Shape}{\upPi_{\infty}}
\newcommand{\Shapen}{\upPi_{n}}
\newcommand{\Shapetrun}{\upPi_{<\infty}}
\newcommand{\Shapeprofin}{\widehat{\upPi}_{\infty}}
\newcommand{\Shapeet}{\Shape^{\et}}
\newcommand{\Shapeetprofin}{\Shapeprofin^{\et}}
\newcommand{\Shapeetprofinone}{\widehat{\upPi}_{1}^{\et}}
\newcommand{\Shapeethyp}{\Shape^{\et,\hyp}}
\newcommand{\Shapeethypprofin}{\Shapeprofin^{\et,\hyp}}

\newcommand{\StrShape}{\widehat{\upPi}_{(\infty,1)}}
\newcommand{\StrShapen}{\widehat{\upPi}_{(n,1)}}
\newcommand{\StrShapeet}{\StrShape^{\et}}


%-------------------------------------------------------------------%
% Todos                                                             %
%-------------------------------------------------------------------%

\usepackage{todonotes}
\usepackage{xargs}  % To use more than one optional parameter in a new command

\usepackage{longtable}
\usepackage{booktabs}

%-------------------------------------------------------------------%
%-------------------------------------------------------------------%
% Document Macros                                                   %
%-------------------------------------------------------------------%
%-------------------------------------------------------------------%

\hypersetup{
    urlcolor = slate,
}

%-------------------------------------------------------------------%
% Indexing                                                          %
%-------------------------------------------------------------------%

\indexsetup{toclevel=part} % Adds indices to table of contents as 'parts'
\makeindex[name=terminology,title={Glossary of Terminology},columns=2,intoc]
\makeindex[name=notation,title={Index of Notation},columns=2,intoc]



%-------------------------------------------------------------------%
%-------------------------------------------------------------------%
% Referencing                                                       %
%-------------------------------------------------------------------%
%-------------------------------------------------------------------%

% \newcommand{\enumref}[2]{(\ref{#1}.\ref{#1.#2})} % Allows for easy referencing of enumerated lists
\newcommand{\enumref}[2]{(\hyperref[#1.#2]{\ref*{#1}.\ref*{#1.#2}})}

\setcounter{secnumdepth}{2}
\numberwithin{equation}{subsection}

\hyphenation{mon-oid-al}
\hyphenation{Grun-dlehren}
\hyphenation{Birkhäus-er} % Comment out when using PDFLaTeX

% Does sectioning and addition of references to table of contents correctly.
\defbibheading{references}[\refname]{%
	\section*{#1}%
	\addcontentsline{toc}{part}{References} %
	\markboth{#1}{#1}
	}

\addbibresource{universal.bib} % To add the bibliography resource

%-------------------------------------------------------------------%
%  Title data                                                       %
%-------------------------------------------------------------------%

\title{General pyknosis}

\author{MSRI}

\date{Spring 2020}

%-------------------------------------------------------------------%
%-------------------------------------------------------------------%
%-------------------------------------------------------------------%
%-------------------------------------------------------------------%

\begin{document}

\maketitle

\bookmark[page=1,level=1]{Contents}
\setcounter{tocdepth}{2}
\tableofcontents

\setcounter{section}{-1}

%!TEX root = rootfile.tex

%-------------------------------------------------------------------%
%-------------------------------------------------------------------%
\section*{Introduction}\addcontentsline{toc}{section}{Introduction}
%-------------------------------------------------------------------%
%-------------------------------------------------------------------%

These are notes for a series of talks at MSRI about the \emph{pyknotic formalism}.

The pyknotic formalism offers a way to coordinate `topological' and `derived' structures.
This formalism is only partially developed, but it’s already clear that there is a lot to explore, and a lot of interesting possible applications.
Many different points of view on pyknotic objects will be necessary to develop these applications.

\subsection*{What we're trying to achieve}

\begin{exm*}
	The study of field theories necessitates often the use of infinite-dimensional vector spaces, such as the space of distributions on a smooth manifold.
	Of course, these are locally convex topological spaces, and the functional analysis literature is full of techniques to deal with these structures.
	At the same time, modern tools like the Batalin--Vilkovisky formalism require the deployment of homological algebra.
	So a natural question arises: how does one \emph{do} homological algebra with the kinds of topological vector spaces that arise in these situations?

	The literature is full of examples in which one manages to work with complexes of locally convex topological vector spaces by contemplating the derived \category of a quasiabelian category.
	But for more sophisticated applications, one has to go further:
\end{exm*}

%!TEX root = rootfile.tex

\section{Elements of general topology}

\subsection{Ultrafilters and compacta}

\begin{ntn}
	Write $ \Set $ for the category of tiny finite sets.
	Write $ \Fin \subset \Set $ for the full subcategory of finite sets,
	and write $ i $ for the inclusion $ \Fin \inclusion \Set$.
\end{ntn}

\begin{dfn}
	For any tiny set $ S $, write $ h^S $ for the functor $ \Fin \to \Set $ given by $ I  \mapsto \Map(S, I) $.
	An \defn{ultrafilter} $ \mu $ on $ S $ is a natural transformation
	\[
		\int_S (\cdot) \ d\mu \colon h^S \to i \comma
	\]
	which for any finite set $I$ gives a map
	\[
		\begin{tikzcd}[column sep={1ex}, row sep={0ex}]
			\Map(S, I) \ar[r] & I \\
			f \ar[r, mapsto] & \int_S f \ d \mu
		\end{tikzcd}
	\]

	Write $ \beta (S) $ for the set of ultrafilters on $ S $.
	For any set $ S $, the set $ \beta(S) $ is the set
	\[
		\beta(S) = \lim_{I \in \Fin_{S/}} I \period
	\]
	The functor
	\[
		\beta \colon \Set \to \Set
	\]
	is thus the right Kan extension of the inclusion $ \Fin \inclusion \Set $ along itself.
\end{dfn}

\begin{exm}
	For any set $ S $ and any element $ s \in S $, there is a \defn{principal ultrafilter} $ \delta_s $, which is defined so that
	\[
		\int_S f \ d \delta_s = f(s) \period
	\]
\end{exm}

Every ultrafilter on a finite set is principal,
but infinite sets have ultrafilters that are not principal.
To prove the existence of these, let us look at a more traditional way of defining an ultrafilter on a set.

\begin{dfn}
	Let $ S $ be a set, $ T \subseteq S$, and $ \mu $ an ultrafilter on $ S $.
	There is a unique \defn{characteristic map} $ \chi_T \colon S \to \{ 0,1 \}$ such that $ \chi_T(s) = 1 $ if and only if $ s \in T $.
	Let us write
	\[
		\mu(T) \coloneq \int_S \chi_T \ d \mu \period
	\]
	
	We say that \defn{$ T $ is $ \mu $-thick} if and only if $\mu(T) = 1$.
	Otherwise (that is, if $ \mu(T) = 0 $), then we say that $ T $ is \defn{$ \mu $-thin}.

	For any $ s \in S$, the principal ultrafilter $ \delta_s $ is the unique ultrafilter relative to which $ \{ s \} $ is thick.
\end{dfn}

\begin{nul}
	If $ S $ is a set and $ \mu $ is an ultrafilter on $ S $, then we can observe the following facts about the collection of thick and thin subsets (relative to $ \mu $):
	\begin{enumerate}[(1)]
		\item The empty set is thin.
		\item Complements of thick sets are thin.
		\item Every subset is either thick or thin.
		\item Subsets of thin sets are thin.
		\item The intersection of two thick sets is thick.
	\end{enumerate}
\end{nul}

\begin{nul}
	Ultrafilters are functorial in maps of sets.
	Let $ \phi \colon S \to T $ be a map, and let $ \mu $ be an ultrafilter on $ S $.
	The ultrafilter $ \phi_{\ast}\mu $ is the ultrafilter on $ T $ given by
	\[
		\int_T f \ d (\phi_{\ast}\mu) = \int_S (f \circ \phi) \ d \mu \period
	\]
	For any $ U \subseteq T$, one has in particular
	\[
		(\phi_{\ast} \mu)(U) = \mu (\phi^{-1}(U)) \period
	\]
	Thus $ U $ is $ \phi_{\ast} \mu $-thick if and only if $ \phi^{-1} U $ is $ \mu $-thick.
\end{nul}

\begin{dfn}
	A \defn{system of thick subsets} of $ S $ is a collection $ F \subseteq \PP(S) $ such that for any finite set $ I $ and any partition
	\[
		S = \coprod_{ i \in I } S_i \comma
	\]
	there is a unique $ i \in I $ such that $ S_i \in F $.
\end{dfn}

\begin{cnstr}
	We have seen that an ultrafilter $ \mu $ specifies the system $ F_{\mu} $ of $ \mu $-thick subsets.
	In the other direction, attached to any system $F$ of thick subsets is an ultrafilter $\mu_F$: for any finite set $ I $ and any map $ f \colon S \to I $, the element $ i = \int_S f \ d \mu \in I $ is the unique one such that $ S_i \in F$.

	The assignments $ \mu \mapsto F_{\mu} $ and $ F \mapsto \mu_F $ together define a bijection between ultrafilters on $S$ and systems of thick subsets.
\end{cnstr}

\begin{dfn}
	If $ S $ is a set, and if $ G \subseteq \PP(S) $, then an ultrafilter $ \mu $ is said to be \defn{supported on $ G $} if and only if every element of $G$ is $ \mu $-thick, that is, $ G \subseteq F_{\mu} $.
\end{dfn}

\begin{lem} \label{generateultrafilters}
	Let $ S $ be a set, and let $ G \subseteq \PP(S) $.
	Assume that no finite intersection of elements of $ G $ is empty.
	Then there exists an ultrafilter $ \mu $ on $ S $ supported on $ G $.
\end{lem}

\begin{proof}
	Consider all the families $ A \subseteq \PP(S) $ with the following properties:
	\begin{enumerate}[(1)]
		\item $ A $ contains $ G $;
		\item \label{FIP} no finite intersection of elements of $ A $ is empty.
	\end{enumerate}
	By Zorn's lemma there is a maximal such family, $ F $.

	We claim that $ F $ is a system of thick subsets.
	For this, let $ S = \coprod_{i \in I} S_i $ be a finite partition of $ S $.
	Condition \ref{FIP} ensures that at most one of the summands $ S_i $ can lie in $ F $.
	Now suppose that none of the summands $ S_i $ lies in $ F $.
	Consider, for each $ i \in I $, the family $ F \cup \{ S_i \} \subseteq \PP(S) $;
	the maximality of $ F $ implies that none of these families can satisfy Condition \ref{FIP}.
	Thus for each $ i \in I $, there is an empty finite intersection
	$S_i \cap \bigcap_{j = 1}^{n_i} T_{ij} = \varnothing $.
	But this implies that the intersection $ \bigcap_{i \in I}\bigcap_{j = 1}^{n_i} T_{ij} $ is empty, contradicting Condition \ref{FIP} for $ F $ itself.
	Hence at least one -- and thus exactly one -- of the summands $ S_i $ lies in $ F $.
	Thus $ F $ is a system of thick subsets of $ S $.
\end{proof}

\begin{nul}
	It is not quite accurate to say that the Axiom of Choice is \emph{necessary} to produce nonprincipal ultrafilters, but it is true that their existence is independent of Zermelo--Fraenkel set theory.
\end{nul}

\begin{nul}
	If $ \phi $ is a functor $ \Set \to \Set $, then a natural transformation $ \phi \to \beta $ is the same thing as a natural transformation $ \phi \circ i \to i $.
	Please observe that we have a canonical identification $ \beta \circ i = i $.

	It follows readily that the functor $ \beta $ is a monad: the unit $ \id \to \beta $ corresponds to the identification $ {\id} \circ i = i $, and the multiplication $ \beta^2 \to \beta $ corresponds to the identification $ \beta^2 \circ i = i $.

	The unit for the monad $\beta$ structure is the assignment $ s \mapsto \delta_s $ that picks out the principal ultrafilter at a point.

	To describe the multiplication $ \tau \mapsto \mu(\tau) $, let us write $ T^{\dag} $ for the set of ultrafilters supported on $\{T\}$.
	Now if $ \tau $ is an ultrafilter on $ \beta(S) $, then $ \mu(\tau) $ is the ultrafilter on $S$ such that
	\[
		\mu(\tau) ( T ) = \tau ( T^{\dag} ) \period
	\]
\end{nul}

\begin{cnstr}
	Let $ \categ{Top} $ denote the category of tiny topological spaces.
	If $ S $ is a set, we can introduce a topology on $ \beta(S) $ simply by forming the inverse limit $ \lim_{I \in \Fin_{S/}} I $ in $ \categ{Top} $.
	That is, we endow $ \beta(S) $ with the coarsest topology such that all the projections $ \beta(S) \to I $ are continuous.
	We call this the \defn{Stone topology} on $\beta(S)$.
	By Tychonoff, this limit is a compact Hausdorff topological space.
	This lifts $ \beta $ to a functor $ \Set \to \Top $.
\end{cnstr}

\begin{nul}
	Let's be more explicit about the topology on $ \beta(S) $.
	The topology on $ \beta(S) $ is generated by the sets $ T^{\dag} $ (for $ T \subseteq S $).
	In fact, since the sets $ T^{\dag} $ are stable under finite intersections, they form a base for the Stone topology on $ \beta(S) $.
	Additionally, since the sets $ T^{\dag} $ are stable under the formation of complements, they even form a base of clopens of $ \beta(S) $.
\end{nul}

\begin{dfn} \label{compactaasbetaalgebras}
	A \defn{compactum} is an algebra for the monad $ \beta $.
	Hence a compactum consists of a set $ K $ and a map $ \lambda_K \colon \beta(K) \to K $, which is required to satisfy the usual identities:
	\[
		\lambda_K(\lambda_{K,\ast}\tau) = \lambda_K(\mu(\tau)) \andeq{} \lambda_K(\delta_s) = s \comma
	\]
	for any ultrafilter $ \tau $ on $ \beta(S) $ and any point $ s \in S $.
	The image $ \lambda_K(\mu) $ will be called the \defn{limit} of the ultrafilter $\mu$.
	The category $ \categ{Comp} $ is the category of compacta.
\end{dfn}

\begin{cnstr}
	If $K$ is a compactum, then we use the limit map $ \lambda_K \colon \beta(K) \to K $ to topologise $ K $ as follows.
	For any subset $ T \subseteq K $, we define the closure of $ T $ as the image $ \lambda_K(T^{\dag}) $.

	A subset $ Z \subseteq K $ is thus closed if and only if the limit of any ultrafilter relative to which $ Z $ is thick lies in $ Z $.
	Dually, a subset $ U \subseteq K $ is open if and only if it is thick with respect to any ultrafilter whose limit lies in $ U $.

	We denote the resulting topological space $ K^{\textit{top}} $.
	The assignment $ K \mapsto K^{\textit{top}} $ defines a lift $ \Alg(\beta) \to \Top $ of the forgetful functor $ \Alg(\beta) \to \Set $.
\end{cnstr}

\begin{prp} \label{compactaarecompacta}
	The functor $ K \mapsto K^{\textit{top}} $ identifies the category of compacta with the category of compact Hausdorff topological spaces. 
\end{prp}

We will spend the remainder of this section proving this claim.
Please observe first that $ K \mapsto K^{\textit{top}} $ is faithful.
What we will do now is prove:
\begin{enumerate}[(1)]
	\item that for any compactum $ K $, the topological space $ K^{\textit{top}} $ is compact Hausdorff;
	\item that for any compact Hausdorff topological space $ X $, there is a $ \beta $-algebra structure $ K $ on the underlying set of $ X $ such that $ X \cong K^{\textit{top}} $; and
	\item that for any compacta $ K $ and $ L $, any continuous map $ K^{\textit{top}} \to L^{\textit{top}} $ lifts to a $ \beta $-algebra homomorphism $ K \to L$.
\end{enumerate}
To do this, it is convenient to describe a related idea: that of \emph{convergence} of ultrafilters on topological spaces.

\begin{dfn} \label{limitpointofultrafilter}
	Let $ X $ be a topological space, and let $ x \in X $.
	We say that $ x $ is a \defn{limit point} of an ultrafilter $ \mu $ on (the underlying set of) $ X $ if and only if every open neighbourhood of $ x $ is $ \mu $-thick.
	In other words, $ x $ is a limit point of $ \mu $ if and only if, for every open neighbourhood $ U $ of $ x $, one has $ \mu \in U^{\dag} $.
\end{dfn}

\begin{lem} \label{opensetsarethickwrtultrafilters}
	Let $ X $ be a topological space, and let $ U \subseteq X$ be a subset.
	Then $ U $ is open if and only if it is thick with respect to any ultrafilter with limit point in $ U $.
\end{lem}

\begin{proof}
	If $ U $ is open, then $ U $ is by definition thick with respect to any ultrafilter with limit point in $ U $.

	Conversely, assume that $ U $ is thick with respect to any ultrafilter with limit point in $ U $.
	Let $ u \in U $.
	Consider the set $ G \coloneq N(u) \cup \{ X \smallsetminus U \} $, where $N(u)$ is the collection of open neighbourhoods of $u$.
	If $ U $ does not contain any open neighbourhood of $u$, then no finite intersection of elements of $ G $ is empty.
	By \Cref{generateultrafilters} there is an ultrafilter $ \mu $ supported on the $ N(u) \cup \{ X \smallsetminus U \} $, whence $ u $ is a limit point of $ \mu $, but $ U $ is not $ \mu $-thick.
	This contradicts our assumption, and so we deduce that $ U $ contains an open neighbourhood of $ u $.
\end{proof}

\begin{lem} \label{continuityviaultrafilters}
	Let $ X $ and $ Y $ be topological spaces, and let $ \phi \colon X \to Y $ be a map.
	Then $ \phi $ is continuous if and only if, for any ultrafilter $ \mu $ on $ X $ with limit point $ x \in X $, the point $ \phi(x) $ is a limit point of $ \phi_{\ast}\mu $.
\end{lem}

\begin{proof}
	Assume that $ \phi $ is continuous, and let $ \mu $ be an ultrafilter on $ X $, and assume that $ x \in X $ is a limit point of  $ \mu $.
	Now assume that $ V $ is an open neighbourhood of $ \phi(x) $.
	Since $ \phi^{-1}V $ is an open neighbourhood of $ x $, so it is $ \mu $-thick, whence $ V $ is $\phi_{\ast}\mu$-thick.
	Thus $ \phi(x) $ is a limit point of $ \phi_{\ast}\mu $.

	Assume now that if $ x \in X $ is a limit point of an ultrafilter $ \mu $, then $ \phi(x) $ is a limit point of $ \phi_{\ast}\mu $.
	Let $ V \subseteq Y $ be an open set.
	Let $ x \in \phi^{-1}(V) $, and let $ \mu $ be an ultrafilter on $ X $ with limit point $ x $.
	Then $ \phi(x) $ is a limit point of $ \phi_{\ast}\mu $, so $V$ is $ \phi_{\ast}\mu $-thick, whence $ \phi^{-1}(V) $ is $ \mu $-thick.
	It follows from \Cref{opensetsarethickwrtultrafilters} that $\phi^{-1}(V)$ is open.
\end{proof}

\begin{lem} \label{quasicompactiffeveryultrafilterhasalimitpoint}
	Let $ X $ be a topological space.
	Then $ X $ is quasicompact if and only if every ultrafilter on $ X $ has at least one limit point.
\end{lem}

\begin{proof}
	Assume first that $ X $ is quasicompact.
	Let $ \mu $ be an ultrafilter on $ X $, and assume that $ \mu $ has no limit point.
	Select, for every point $ x \in X $, an open neighbourhood $ U_x $ thereof that is not $ \mu $-thick.
	Quasicompactness implies that there is a finite collection $ x_1, \dots, x_n \in X $ such that $ \{ U_{x_1}, \dots, U_{x_n} \} $ covers $ X $.
	But at least one of $ U_{x_1}, \dots, U_{x_n} $ must be $ \mu $-thick.
	This is a contradiction.

	Now assume that $ X $ is not quasicompact.
	Then there exists a collection $ G \subseteq \PP(X) $ of closed subsets of $ X $ such that the intersection all the elements of $ G $ is empty, but no finite intersection of elements of $ G $ is empty.
	In light of \Cref{generateultrafilters}, there is an ultrafilter $ \mu $ with the property that every element of $ G $ is thick.
	For any $ x \in X $, there is an element $ Z \in G $ such that $ x \in X \smallsetminus Z $.
	Since $ Z $ is $ \mu $-thick, $ X \smallsetminus Z $ is not.
	Thus $ \mu $ has no limit points.
\end{proof}

\begin{lem} \label{hausdorffiffeveryultrafilterhasatmostonelimitpoint}
	Let $ X $ be a topological space.
	Then $ X $ is Hausdorff if and only if every ultrafilter on $ X $ has at most one limit point.
\end{lem}

\begin{proof}
	Assume that $ \mu $ is an ultrafilter with two distinct limit points $ x_1 $ and $ x_2 $.
	Choose open neighbourhoods $ U_1 $ of $ x_1 $ and $ U_2 $ of $ x_2 $.
	Since they are both $ \mu $-thick, they cannot be disjoint;
	hence $ X $ is not Hausdorff.

	Conversely, assume that $ X $ is not Hausdorff.
	Select two points $ x_1 $ and $ x_2 $ such that every open neighbourhoods $ U_1 $ of $ x_1 $ and $ U_2 $ of $ x_2 $ intersect.
	Now the set $ G $ consisting of open neighbourhoods of either $ x_1 $ \emph{or} $ x_2 $ has the property that no finite intersection of elements of $ G $ is empty.
	In light of \Cref{generateultrafilters}, there is an ultrafilter $ \mu $ with the property that every element of $ G $ is thick.
	Thus $ x_1 $ and $ x_2 $ are limit points of $ \mu $.
\end{proof}

Let us now return to our functor $ K \mapsto K^{\textit{top}} $.

\begin{lem} \label{limitsarelimits}
	Let $ K $ be a compactum, and let $ \mu $ be an ultrafilter on $ K $.
	Then a point of $ K^{\textit{top}} $ is a limit point of $ \mu $ in the sense of \Cref{limitpointofultrafilter} if and only if it is the limit of $ \mu $ in the sense of \Cref{compactaasbetaalgebras}.
\end{lem}

\begin{proof}
	Let $ x \coloneq \lambda_K(\mu) $.
	The open neighbourhoods $ U $ of $ x $ are by definition thick (relative to $ \mu $), so certainly $ x $ is a limit point of $ \mu $.

	Now assume that $ y \in K^{\textit{top}} $ is a limit point of $ \mu $.
	To prove that the limit of $ \mu $ is $ y $, we shall build an ultrafilter $ \tau $ on $ \beta(K) $ with the following properties:
	\begin{enumerate}[(1)]
		\item under the multiplication $ \beta^2 \to \beta $, the ultrafilter $ \tau $ is sent to $ \mu $; and
		\item under the map $ \lambda_{\ast} \colon \beta^2 \to \beta  $, the ultrafilter $ \tau $ is sent to $\delta_y$.
	\end{enumerate}
	Once we have succeeded, it will follow that
	\[
		\lambda_K( \mu ) = \lambda_K( \mu_{\tau} ) = \lambda_K(\lambda_{K,\ast}\tau) = \lambda_K(\delta_y) = y \comma
	\]
	and the proof will be complete.

	Consider the family $ G' $ of subsets of $ \beta(K) $ of the form $ T^{\dag} $ for a $ \mu $-thick subset $ T \subseteq S $;
	since these are all nonempty and they are stable under finite intersections, it follows that no finite intersection of elements of $ G' $ is empty.

	Now consider the set $ G \coloneq G' \cup \{ \lambda_K^{-1}\{y\} \}$.
	If $ T $ is $ \mu $-thick, then we claim that there is an ultrafilter $ \nu \in \lambda_K^{-1}\{y\} \cap T^{\dag} $.
	Indeed, consider the set $ N(y) \cup \{T\} $, where $ N(y) $ is the collection of open neighbourhoods of $ y $.
	Since every open neighbourhood of $ y $ is $ \mu $-thick, no intersection of an open neighbourhood of $ y $ with $ T $ is empty.
	By \Cref{generateultrafilters} there is an ultrafilter supported on $ N(y) \cup \{T\} $, which implies that no finite intersection of elements of $ G $ is empty.

	Applying \Cref{generateultrafilters} again, we see that $ G $ supports an ultrafilter $ \tau $ on $ \beta(K) $.
	For any $ T \subseteq K $,
	\[
		\mu_{\tau}(T) = \tau(T^{\dag}) \comma
	\]
	so since $ \tau $ is supported on $ G' $, it follows that $ \mu_{\tau} = \mu $.
	At the same time, since $ \tau $ is supported on $ \{\lambda_K^{-1}\{y\}\} $, it follows that $ \{y\} $ is thick relative to $ \lambda_{K,\ast}\tau $, whence $ \lambda_{K,\ast}\tau = \delta_y $.
\end{proof}

\begin{proof}[Proof of \Cref{compactaarecompacta}]
	Let $ K $ be a compactum.
	Combine \Cref{hausdorffiffeveryultrafilterhasatmostonelimitpoint,quasicompactiffeveryultrafilterhasalimitpoint,limitsarelimits} to conclude that $ K^{\textit{top}} $ is a compact Hausdorff topological space.

	Let $ X $ be a compact Hausdorff topological space with underlying set $ K $.
	Define a map $ \lambda_K \colon \beta(K) \to K $ by carrying an ultrafilter $ \mu $ to its unique limit point in $ X $.
	This is a $ \beta $-algebra structure on $ X $, and it follows from \Cref{limitsarelimits} and the definition of the topology together imply that $ X \cong K^{\textit{top}}$.

	Finally, let $ K $ and $ L $ be compacta, and let $ \phi \colon K^{\textit{top}} \to L^{\textit{top}} $ be a continuous map.
	To prove that $ \phi $ is a $ \beta $-algebra homomorphism, it suffices to confirm that if $ \mu $ is an ultrafilter on $ K $, then
	\[
		\lambda_L (\phi_{\ast} \mu) = \phi (\lambda_K(\mu)) \comma
	\]
	but this follows exactly from \Cref{continuityviaultrafilters}.
\end{proof}


%!TEX root = rootfile.tex

%-------------------------------------------------------------------%
%-------------------------------------------------------------------%
\section{Pyknotic sets}
%-------------------------------------------------------------------%
%-------------------------------------------------------------------%

%-------------------------------------------------------------------%
\subsection{The effective epimorphism topology}
%-------------------------------------------------------------------%

\begin{cnstr}
	A continuous map of compacta $ K \to L $ is a surjection if and only if it is a quotient map.
	It is in that case an \emph{effective epimorphism};
	that is, the sequence
	\[
		K \times_L K \parallelto K \to L
	\]
	is a coequaliser.
\end{cnstr}






%-------------------------------------------------------------------%
%-------------------------------------------------------------------%
%-------------------------------------------------------------------%
%  References                                                       %
%-------------------------------------------------------------------%
%-------------------------------------------------------------------%
%-------------------------------------------------------------------%

\newpage

% Separates bibliography into two parts: the specially labeled references followed by the numbered references. The specially labeled references appear with just the label; numbered references appear as "n."
\DeclareFieldFormat{labelnumberwidth}{#1}
\printbibliography[keyword=alph, heading=references]
% \addcontentsline{toc}{part}{References} % Adds References to table of contents
\DeclareFieldFormat{labelnumberwidth}{{#1\adddot\midsentence}}
\printbibliography[heading=none, notkeyword=alph]

%-------------------------------------------------------------------%
%-------------------------------------------------------------------%
%-------------------------------------------------------------------%
%  Indices                                                          %
%-------------------------------------------------------------------%
%-------------------------------------------------------------------%
%-------------------------------------------------------------------%

\newpage
% \addcontentsline{toc}{part}{Index of Notation}
\printindex[notation]

\newpage
% \addcontentsline{toc}{part}{Glossary of Terminology}
\printindex[terminology]

\end{document}