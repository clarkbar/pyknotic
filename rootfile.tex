% !TeX program = xelatex
%-*- program: xelatex -*-
%-*- encoding: utf-8 -*-
% \pdfoutput=1 % for the arXiv
\documentclass[draft=false, leqno]{article}
\usepackage{amsmath,amsthm,mathrsfs}
\usepackage[minionmath]{boilerart1}
\usepackage{longtable}
\usepackage{booktabs}

%-------------------------------------------------------------------%
%-------------------------------------------------------------------%
% Document Macros                                                   %
%-------------------------------------------------------------------%
%-------------------------------------------------------------------%

\hypersetup{
    urlcolor = slate,
}

%-------------------------------------------------------------------%
% Indexing                                                          %
%-------------------------------------------------------------------%

\indexsetup{toclevel=part} % Adds indices to table of contents as 'parts'
\makeindex[name=terminology,title={Glossary of Terminology},columns=2,intoc]
\makeindex[name=notation,title={Index of Notation},columns=2,intoc]

%-------------------------------------------------------------------%
% Typesetting                                                       %
%-------------------------------------------------------------------%

\theoremstyle{definition}
\newtheorem{rmk}[equation]{Remark}

\newcommand{\invert}{H}
\newcommand{\respectively}[1]{\text{\qquad(respectively,\quad}{#1}\text{\quad)}}
\newcommand{\resp}[1]{\text{\qquad(resp.,\quad}{#1}\text{\quad)}}
\newcommand{\andeq}{\text{\qquad and\qquad}}
\newcommand{\period}{\rlap{\ .}}
\newcommand{\comma}{\rlap{\ ,}}

\usepackage[style=british]{csquotes}
\newcommand{\defn}[1]{\emph{#1}}
\newcommand{\upperth}{\textsuperscript{th}\,}

%-------------------------------------------------------------------%
% Terms with an '∞' (or not)                                        %
%-------------------------------------------------------------------%

\newcommand{\category}{{$\infty$-cat\-e\-go\-ry}\xspace}
\newcommand{\categories}{{$\infty$-cat\-e\-gories}\xspace}
\newcommand{\categorical}{{$\infty$-cat\-e\-gor\-i\-cal}\xspace}
\newcommand{\Categorical}{{$\infty$-Cat\-e\-gor\-i\-cal}\xspace}
\newcommand{\acategory}{{an $\infty$-cat\-e\-go\-ry}\xspace}
\newcommand{\Acategory}{{An $\infty$-cat\-e\-go\-ry}\xspace}
\newcommand{\acategorical}{{an $\infty$-cat\-e\-gor\-i\-cal}\xspace}
\newcommand{\Acategorical}{{An $\infty$-cat\-e\-gor\-i\-cal}\xspace}

\newcommand{\groupoid}{{$\infty$-group\-oid}\xspace}
\newcommand{\groupoids}{{$\infty$-group\-oids}\xspace}
\newcommand{\agroupoid}{{an $\infty$-group\-oid}\xspace}
\newcommand{\Agroupoid}{{An $\infty$-group\-oid}\xspace}

\newcommand{\topos}{{$\infty$-topos}\xspace}
\newcommand{\topoi}{{$\infty$-topoi}\xspace}
\newcommand{\toposic}{{$\infty$-topos\-ic}\xspace}
\newcommand{\atopos}{{an $\infty$-topos}\xspace}
\newcommand{\Atopos}{{An $\infty$-topos}\xspace}
\newcommand{\Topos}{{$\infty$-Topos}\xspace}
\newcommand{\Topoi}{{$\infty$-Topoi}\xspace}

\newcommand{\pretopos}{{$\infty$-pre\-topos}\xspace}
\newcommand{\pretopoi}{{$\infty$-pre\-topoi}\xspace}
\newcommand{\pretoposic}{{$\infty$-pre\-toposic}\xspace}
\newcommand{\apretopos}{{an $\infty$-pre\-topos}\xspace}
\newcommand{\Apretopos}{{An $\infty$-pre\-topos}\xspace}

\newcommand{\site}{{$\infty$-site}\xspace}
\newcommand{\sites}{{$\infty$-sites}\xspace}
\newcommand{\asite}{{an $\infty$-site}\xspace}
\newcommand{\Asite}{{An $\infty$-site}\xspace}

\newcommand{\presite}{{$\infty$-pre\-site}\xspace}
\newcommand{\presites}{{$\infty$-pre\-sites}\xspace}
\newcommand{\apresite}{{an $\infty$-pre\-site}\xspace}
\newcommand{\Apresite}{{An $\infty$-pre\-site}\xspace}

%-------------------------------------------------------------------%
% Letters                                                           %
%-------------------------------------------------------------------%

\newcommand{\elbar}{\bar{\el}}

\newcommand{\Cech}{\check{C}}

\newcommand{\elowerstar}{e_{\ast}}
\newcommand{\eupperstar}{e^{\ast}}
\newcommand{\ilowerstar}{i_{\ast}}
\newcommand{\iupperstar}{i^{\ast}}
\newcommand{\jlowerstar}{j_{\ast}}
\newcommand{\jupperstar}{j^{\ast}}
\newcommand{\jlowershriek}{j_{!}}
\newcommand{\fbar}{\bar{f}}
\newcommand{\flowerstar}{f_{\ast}}
\newcommand{\fupperstar}{f^{\ast}}
\newcommand{\fuppershriek}{f^{!}}
\newcommand{\flowershriek}{f_{!}}
\newcommand{\gbar}{\bar{g}}
\newcommand{\glowerstar}{g_{\ast}}
\newcommand{\gupperstar}{g^{\ast}}
\newcommand{\guppershriek}{g^{!}}
\newcommand{\hlowerstar}{h_{\ast}}
\newcommand{\hupperstar}{h^{\ast}}
\newcommand{\huppershriek}{h^{!}}
\newcommand{\plowerstar}{p_{\ast}}
\newcommand{\pupperstar}{p^{\ast}}
\newcommand{\puppershriek}{p^{!}}
\newcommand{\plowershriek}{p_{!}}
\newcommand{\prupperstar}{\pr^{\ast}}
\newcommand{\qlowerstar}{q_{\ast}}
\newcommand{\tupperstar}{t^{\ast}}
\newcommand{\tlowerstar}{t_{\ast}}
\newcommand{\qupperstar}{q^{\ast}}
\newcommand{\xlowerstar}{x_{\ast}}
\newcommand{\xupperstar}{x^{\ast}}
\newcommand{\xuppershriek}{x^{!}}
\newcommand{\ylowerstar}{y_{\ast}}
\newcommand{\yupperstar}{y^{\ast}}
\newcommand{\yuppershriek}{y^{!}}
\newcommand{\wlowerstar}{w_{\ast}}
\newcommand{\wupperstar}{w^{\ast}}
\newcommand{\wuppershriek}{w^{!}}
\newcommand{\zlowerstar}{z_{\ast}}
\newcommand{\zupperstar}{z^{\ast}}
\newcommand{\zuppershriek}{z^{!}}
\newcommand{\philowerstar}{\phi_{\ast}}
\newcommand{\pihat}{\hat{\pi}}
\newcommand{\piupperstar}{\pi^{\ast}}
\newcommand{\pilowerstar}{\pi_{\ast}}
\newcommand{\xiupperstar}{\xi^{\ast}}
\newcommand{\xilowerstar}{\xi_{\ast}}
\newcommand{\phiupperstar}{\phi^{\ast}}
\newcommand{\betabar}{\bar{\beta}}
\newcommand{\Gammalowerstar}{\Gamma_{\ast}}
\newcommand{\Gammaupperstar}{\Gamma^{\ast}}
\newcommand{\Omegalowerstar}{\Omega_{\ast}}
\newcommand{\Omegaupperstar}{\Omega^{\ast}}
\newcommand{\Psilowerstar}{\Psi_{\ast}}
\newcommand{\Psiupperstar}{\Psi^{\ast}}
\newcommand{\Gammauppershriek}{\Gamma^{!}}
\newcommand{\Gammalowershriek}{\Gamma_{!}}
\newcommand{\tauhat}{\hat{\tau}}
\newcommand{\Ntilde}{\widetilde{N}}
\newcommand{\Xtilde}{\widetilde{X}}
\newcommand{\Ztilde}{\widetilde{Z}}

%-------------------------------------------------------------------%
% Italic text                                                       %
%-------------------------------------------------------------------%

\newcommand{\Nis}{\ensuremath{\textit{nis}}}
\renewcommand{\fin}{\,\ensuremath{\textit{fin}}}
\newcommand{\bcc}{\ensuremath{\textit{bcc}}}
\newcommand{\cart}{\ensuremath{\textit{cart}}}
\newcommand{\cocart}{\ensuremath{\textit{cocart}}}
\newcommand{\leftfib}{\ensuremath{\textit{left}}}
\newcommand{\rightfib}{\ensuremath{\textit{right}}}
\newcommand{\constr}{\ensuremath{\textit{constr}}}
\newcommand{\fconstr}{\ensuremath{\textit{fconstr}}}
\newcommand{\lc}{\ensuremath{\textit{lc}}}
\newcommand{\pro}{\ensuremath{\textit{pro}}}
\newcommand{\lft}{\ensuremath{\textit{lft}}}
\newcommand{\an}{\ensuremath{\textit{an}}}
\newcommand{\proet}{\ensuremath{\textit{proét}}}
\newcommand{\perf}{\ensuremath{\textit{perf}}}
\newcommand{\red}{\ensuremath{\textit{red}}}
\newcommand{\zar}{\ensuremath{\textit{zar}}}
\newcommand{\hyp}{\ensuremath{\textit{hyp}}}
\newcommand{\post}{\ensuremath{\textit{post}}}
\newcommand{\bdd}{\ensuremath{\textit{b}}}
\newcommand{\spec}{\ensuremath{\textit{spec}}}
\newcommand{\disc}{\ensuremath{\textit{disc}}}
\newcommand{\loc}{\ensuremath{\textit{loc}}}
\newcommand{\qc}{\ensuremath{\textit{qc}}}
\newcommand{\vop}{\ensuremath{\textit{vop}}}
\newcommand{\trig}{\ensuremath{\textit{trig}}}
\newcommand{\all}{\ensuremath{\textit{all}}}
\newcommand{\coh}{\ensuremath{\textit{coh}}}
\newcommand{\ncoh}{n\ensuremath{\textit{-coh}}}
\newcommand{\bc}{\ensuremath{\textit{bc}}}
\newcommand{\lex}{\ensuremath{\textit{lex}}}
\newcommand{\ft}{\ensuremath{\textit{ft}}}
\newcommand{\inv}{\ensuremath{\textit{inv}}}
\newcommand{\cts}{\ensuremath{\textit{cts}}}
\newcommand{\ext}{\ensuremath{\textit{ext}}}
\newcommand{\noeth}{\ensuremath{\textit{noeth}}}
\newcommand{\sober}{\ensuremath{\textit{sober}}}
\newcommand{\sh}{\ensuremath{\textit{sh}}}
\newcommand{\hens}{\ensuremath{\textit{h}}}
\newcommand{\dec}{\ensuremath{\textit{déc}}}
\newcommand{\predec}{\ensuremath{\textit{predéc}}}
\newcommand{\pre}{\ensuremath{\textit{pre}}}
\newcommand{\toptextit}{\ensuremath{\textit{top}}}
\newcommand{\abs}{\ensuremath{\textit{abs}}}
\newcommand{\norm}{\ensuremath{\textit{norm}}}
\newcommand{\po}{\ensuremath{\textit{po}}}
\newcommand{\stextit}{\ensuremath{\textit{s}}}
\newcommand{\cc}{\ensuremath{\textit{cc}}}
\newcommand{\adm}{\ensuremath{\textit{adm}}}

%-------------------------------------------------------------------%
% Binary operators & symbols                                        %
%-------------------------------------------------------------------%

\newcommand{\cross}{\times}
\newcommand{\crosslimits}{\operatornamewithlimits{\cross}} % A limit version of × that puts subscripts underneath
\newcommand{\leftadjoint}{\shortlefttack}
\newcommand{\isomorphic}{\cong}
\newcommand{\orientedcup}{\mathbin{\overleftarrow{\cup}}}
\newcommand{\orientedtimes}{\mathbin{\overleftarrow{\times}}}
\newcommand{\of}{\circ}
\newcommand{\equivalent}{\simeq}
% \newcommand{\cop}{Ⓒ}
\newcommand{\paren}[1]{\left(#1\right)}
\newcommand{\leftnat}[1]{\vphantom{#1}_{\natural}\mskip-1mu{#1}}
\newcommand{\rightnat}[1]{{#1}^{\natural}}
\renewcommand{\rhd}{\smalltriangleright}
\newcommand{\angs}[1]{\langle #1 \rangle}
\newcommand{\ultimes}{\,\underline{\times}\,}
\newcommand{\horn}[2]{\Lambda_{#1}^{#2}}
\def\rnatural{{\mathpalette\reflectedop\natural}}
\def\reflectedop#1#2{\mathop{\reflectbox{$#1{#2}$}}}
\newcommand{\cosv}[1]{{\uparrow}{#1}}
\newcommand{\sv}[1]{{\downarrow}{#1}}

%-------------------------------------------------------------------%
% Categories                                                        %
%-------------------------------------------------------------------%

\newcommand{\poSet}{\po\Set}
\newcommand{\poSetfin}{\poSet^{\fin}}
\newcommand{\Setfin}{\Set^{\fin}}
\newcommand{\Preord}{\categ{Preord}}

\newcommand{\Comp}{\categ{Comp}}
\newcommand{\Stone}{\categ{Stone}}
\newcommand{\Vect}{\categ{Vect}}
\newcommand{\Proj}{\categ{Proj}}
\newcommand{\Cplx}{\categ{Cplx}}

\newcommand{\FC}{\categ{FC}}
\newcommand{\FEt}{\textbf{\textup{FÉt}}}
\newcommand{\Fib}{\categ{Fib}}
\newcommand{\Sch}{\categ{Sch}}
\newcommand{\Schft}{\Sch^{\,\ft}}

\newcommand{\Op}{\categ{Op}}
\newcommand{\Orbital}{\categ{Orb}}
\newcommand{\PreOp}{\categ{PreOp}}

\newcommand{\TSpc}{\categ{TSpc}}
\newcommand{\TSpcall}{\TSpc_{\all}}
\newcommand{\Num}{\TSpc}
\newcommand{\TSpcP}{\TSpc_{/P}}
\newcommand{\sSet}{\stextit\Set}
\newcommand{\sSetP}{\sSet_{/P}}

\newcommand{\LS}{\categ{LS}}
\newcommand{\Sh}{\categ{Sh}}
\newcommand{\Sheff}[1]{\Sh_{\eff}(#1)}

\newcommand{\preTop}{\pre\Top_{\infty}}
\newcommand{\preTopbdd}{\preTop^{\bdd}}
\newcommand{\Topcoh}{\Top_{\infty}^{\coh}}
\newcommand{\Topbc}{\Top_{\infty}^{\bc}}
\newcommand{\Topbcadm}{\Top_{\infty}^{\bc,\adm}}
\newcommand{\Toploc}{\Top_{\infty}^{\loc}}
\newcommand{\Toppt}{\Top_{\infty,\ast}}
\newcommand{\StrTop}{\categ{StrTop}}
\newcommand{\StrTopspec}{\StrTop^{\spec}}
\newcommand{\LocCocart}{\categ{LocCocart}}

\newcommand{\profincomp}{_{\pi}^{\wedge}}
\newcommand{\Spacefin}{\Space_{\pi}}
\newcommand{\Spaceprofin}{\Space_{\pi}^{\wedge}}
\newcommand{\Str}{\categ{Str}}
\newcommand{\Strat}{\categ{Str}}
\newcommand{\Strfin}{\Str_{\pi}}
\newcommand{\Stratfin}{\Str_{\pi}}
\newcommand{\Strprofin}{\Str_{\pi}^{\wedge}}
\newcommand{\Stratprofin}{\Str_{\pi}^{\wedge}}
\newcommand{\Lay}{\categ{Lay}}
\newcommand{\fet}{\ensuremath{\textit{fét}}}

\newcommand{\Aff}{\categ{Aff}}
\newcommand{\PSh}{\categ{PSh}}
\newcommand{\Constr}{\categ{Constr}}

%-------------------------------------------------------------------%
% Tildes                                                            %
%-------------------------------------------------------------------%

\newcommand{\xtilde}{\widetilde{x}}
\newcommand{\stilde}{\tilde{s}}
\newcommand{\Stilde}{\widetilde{S}}
\newcommand{\Ptilde}{\widetilde{P}}
\newcommand{\Mtilde}{\widetilde{M}}
\newcommand{\Utilde}{\widetilde{U}}
\newcommand{\Sigmatilde}{\widetilde{\Sigma}}

%-------------------------------------------------------------------%
% Math operators                                                    %
%-------------------------------------------------------------------%

\DeclareMathOperator{\Isom}{Isom}
\DeclareMathOperator{\Equival}{Equiv}
\DeclareMathOperator{\Flat}{Flat}
\DeclareMathOperator{\Gal}{Gal}
\newcommand{\GalDelta}{\Gal^{\Delta}}
\DeclareMathOperator{\Lin}{Lin}
\DeclareMathOperator{\mat}{mat}
\DeclareMathOperator{\Pairing}{Pair}
\DeclareMathOperator{\Pro}{Pro}
\DeclareMathOperator{\rk}{rk}
\DeclareMathOperator{\Sing}{Sing}
\DeclareMathOperator{\specpos}{sp}
\DeclareMathOperator{\Tot}{Tot}
\DeclareMathOperator{\Pt}{Pt}
\DeclareMathOperator{\Locop}{Loc}
\DeclareMathOperator{\Nbd}{Nbd}
\DeclareMathOperator{\Ran}{Ran}
\newcommand{\MOR}{\ensuremath{\textup{\textsc{Mor}}}}
\newcommand{\Funlowerstar}{\Fun_{\ast}}
\newcommand{\Funupperstar}{\Fun^{\ast}}
\newcommand{\sdop}{\sd^{\op}}

%-------------------------------------------------------------------%
% Coherent objects                                                  %
%-------------------------------------------------------------------%

\newcommand{\WWcoh}{\WW^{\coh}}
\newcommand{\XXcoh}{\XX^{\coh}}
\newcommand{\YYcoh}{\YY^{\coh}}
\newcommand{\ZZcoh}{\ZZ^{\coh}}
\newcommand{\WWcohbdd}{\WWcoh_{<\infty}}
\newcommand{\XXcohbdd}{\XXcoh_{<\infty}}
\newcommand{\YYcohbdd}{\YYcoh_{<\infty}}
\newcommand{\ZZcohbdd}{\ZZcoh_{<\infty}}

%-------------------------------------------------------------------%
% Other topoi                                                       %
%-------------------------------------------------------------------%

\newcommand{\XXbar}{\overline{\XX}}
\newcommand{\XXhyp}{\XX^{\hyp}}
\DeclareMathOperator{\Pathop}{Path}
\newcommand{\Path}[1]{\Pathop(#1)}

%-------------------------------------------------------------------%
% Lisse & constructible objects                                     %
%-------------------------------------------------------------------%

\newcommand{\lisse}{\ensuremath{\textit{lisse}}}
\newcommand{\locsys}{\ensuremath{\textit{locsys}}}
\newcommand{\oneconstr}{[1]\ensuremath{\textit{-constr}}}
\newcommand{\Pconstr}{P\ensuremath{\textit{-constr}}}
\newcommand{\Pprimeconstr}{P'\ensuremath{\textit{-constr}}}
\newcommand{\Sconstr}{S\ensuremath{\textit{-constr}}}
\newcommand{\Sprimeconstr}{S'\ensuremath{\textit{-constr}}}
\newcommand{\Mconstr}{M\ensuremath{\textit{-constr}}}
\newcommand{\Uconstr}{U\ensuremath{\textit{-constr}}}
\newcommand{\Pfconstr}{P\ensuremath{\textit{-fconstr}}}
\newcommand{\Sfconstr}{S\ensuremath{\textit{-fconstr}}}
\newcommand{\Pspec}{P\ensuremath{\textit{-spec}}}
\newcommand{\Sspec}{S\ensuremath{\textit{-spec}}}
\newcommand{\nspec}{[n]\ensuremath{\textit{-spec}}}

%-------------------------------------------------------------------%
% Décollages                                                        %
%-------------------------------------------------------------------%

\newcommand{\Dec}{\categ{Déc}}
\newcommand{\DecTop}[1]{\Dec_{#1}(\Topbc)}
\newcommand{\DecTopStone}[1]{\Dec_{#1}(\TopStone)}
\newcommand{\DecSpace}[1]{\Dec_{#1}(\Space)}
\newcommand{\DecSpacefin}[1]{\Dec_{#1}(\Spacefin)}
\newcommand{\DecSpaceprofin}[1]{\Dec_{#1}(\Spaceprofin)}

%-------------------------------------------------------------------%
% Oriented fibre products and pushouts                              %
%-------------------------------------------------------------------%

\newcommand{\orientedpull}[3]{#1 \orientedtimes_{#2} #3}
\newcommand{\orientedpush}[3]{#1 \orientedcup^{#2} #3}
\newcommand{\commacat}[3]{#1 \mathbin{\downarrow}_{#2} #3}
\newcommand{\Loc}[2]{{#1}_{(#2)}}

%-------------------------------------------------------------------%
%-------------------------------------------------------------------%
% Referencing                                                       %
%-------------------------------------------------------------------%
%-------------------------------------------------------------------%

% \newcommand{\enumref}[2]{(\ref{#1}.\ref{#1.#2})} % Allows for easy referencing of enumerated lists
\newcommand{\enumref}[2]{(\hyperref[#1.#2]{\ref*{#1}.\ref*{#1.#2}})}

\setcounter{secnumdepth}{2}
\numberwithin{equation}{subsection}

\hyphenation{mon-oid-al}
\hyphenation{Grun-dlehren}
\hyphenation{Birkhäus-er} % Comment out when using PDFLaTeX

% Does sectioning and addition of references to table of contents correctly.
\defbibheading{references}[\refname]{%
	\section*{#1}%
	\addcontentsline{toc}{part}{References} %
	\markboth{#1}{#1}
	}

\addbibresource{universal.bib} % To add the bibliography resource

%-------------------------------------------------------------------%
% Todos                                                             %
%-------------------------------------------------------------------%

\usepackage{todonotes}
\usepackage{xargs}  % To use more than one optional parameter in a new command
\newcommandx{\PHtodo}[2][1=]{\todo[linecolor=blue,backgroundcolor=blue!25,bordercolor=blue,#1]{#2}}

%-------------------------------------------------------------------%
%  Title data                                                       %
%-------------------------------------------------------------------%

\title{General pyknosis}

\author{MSRI}

\date{Spring 2020}

%-------------------------------------------------------------------%
%-------------------------------------------------------------------%
%-------------------------------------------------------------------%
%-------------------------------------------------------------------%

\begin{document}

\maketitle

\bookmark[page=1,level=1]{Contents}
\setcounter{tocdepth}{2}
\tableofcontents

\setcounter{section}{-1}

%!TEX root = rootfile.tex

%-------------------------------------------------------------------%
%-------------------------------------------------------------------%
\section*{Introduction}\addcontentsline{toc}{section}{Introduction}
%-------------------------------------------------------------------%
%-------------------------------------------------------------------%

These are notes for a series of talks at MSRI about the \emph{pyknotic formalism}.

The pyknotic formalism offers a way to coordinate `topological' and `derived' structures.
This formalism is only partially developed, but it’s already clear that there is a lot to explore, and a lot of interesting possible applications.
Many different points of view on pyknotic objects will be necessary to develop these applications.

\subsection*{What we're trying to achieve}

\begin{exm*}
	The study of field theories necessitates often the use of infinite-dimensional vector spaces, such as the space of distributions on a smooth manifold.
	Of course, these are locally convex topological spaces, and the functional analysis literature is full of techniques to deal with these structures.
	At the same time, modern tools like the Batalin--Vilkovisky formalism require the deployment of homological algebra.
	So a natural question arises: how does one \emph{do} homological algebra with the kinds of topological vector spaces that arise in these situations?

	The literature is full of examples in which one manages to work with complexes of locally convex topological vector spaces by contemplating the derived \category of a quasiabelian category.
	But for more sophisticated applications, one has to go further:
\end{exm*}

%!TEX root = rootfile.tex

%-------------------------------------------------------------------%
%-------------------------------------------------------------------%
\section{Elements of general topology}
%-------------------------------------------------------------------%
%-------------------------------------------------------------------%

%-------------------------------------------------------------------%
\subsection{Ultrafilters and compacta}
%-------------------------------------------------------------------%

\begin{ntn}
	Write $ \Set $ for the category of tiny finite sets.
	Write $ \Fin \subset \Set $ for the full subcategory of finite sets,
	and write $ i $ for the inclusion $ \Fin \inclusion \Set$.
\end{ntn}

\begin{dfn}
	For any tiny set $ S $, write $ h^S $ for the functor $ \Fin \to \Set $ given by $ I  \mapsto \Map(S, I) $.
	An \defn{ultrafilter} $ \mu $ on $ S $ is a natural transformation
	\[
		\int_S (\cdot) \ d\mu \colon h^S \to i \comma
	\]
	which for any finite set $I$ gives a map
	\[
		\begin{tikzcd}[column sep={1ex}, row sep={0ex}]
			\Map(S, I) \ar[r] & I \\
			f \ar[r, mapsto] & \int_S f \ d \mu
		\end{tikzcd}
	\]

	Write $ \beta (S) $ for the set of ultrafilters on $ S $.
	For any set $ S $, the set $ \beta(S) $ is the set
	\[
		\beta(S) = \lim_{I \in \Fin_{S/}} I \period
	\]
	The functor
	\[
		\beta \colon \Set \to \Set
	\]
	is thus the right Kan extension of the inclusion $ \Fin \inclusion \Set $ along itself.
\end{dfn}

\begin{exm}
	For any set $ S $ and any element $ s \in S $, there is a \defn{principal ultrafilter} $ \delta_s $, which is defined so that
	\[
		\int_S f \ d \delta_s = f(s) \period
	\]
\end{exm}

Every ultrafilter on a finite set is principal,
but infinite sets have ultrafilters that are not principal.
To prove the existence of these, let us look at a more traditional way of defining an ultrafilter on a set.

\begin{dfn}
	Let $ S $ be a set, $ T \subseteq S$, and $ \mu $ an ultrafilter on $ S $.
	There is a unique \defn{characteristic map} $ \chi_T \colon S \to \{ 0,1 \}$ such that $ \chi_T(s) = 1 $ if and only if $ s \in T $.
	Let us write
	\[
		\mu(T) \coloneq \int_S \chi_T \ d \mu \period
	\]
	
	We say that \defn{$ T $ is $ \mu $-thick} if and only if $\mu(T) = 1$.
	Otherwise (that is, if $ \mu(T) = 0 $), then we say that $ T $ is \defn{$ \mu $-thin}.

	For any $ s \in S$, the principal ultrafilter $ \delta_s $ is the unique ultrafilter relative to which $ \{ s \} $ is thick.
\end{dfn}

\begin{nul}
	If $ S $ is a set and $ \mu $ is an ultrafilter on $ S $, then we can observe the following facts about the collection of thick and thin subsets (relative to $ \mu $):
	\begin{enumerate}[(1)]
		\item The empty set is thin.
		\item Complements of thick sets are thin.
		\item Every subset is either thick or thin.
		\item Subsets of thin sets are thin.
		\item The intersection of two thick sets is thick.
	\end{enumerate}
	In other words, if $ S $ is a set, then an ultrafilter on $ S $ is tantamount to a Boolean algebra homomorphism $ \PP(S) \to \{0,1\} $.

	It is possible to define ultrafilters on more general posets, and if $ P $ is a Boolean algebra, then an ultrafilter is precisely a Boolean algebra homomorphism $ P \to \{0, 1\} $.
\end{nul}

\begin{nul}
	Ultrafilters are functorial in maps of sets.
	Let $ \phi \colon S \to T $ be a map, and let $ \mu $ be an ultrafilter on $ S $.
	The ultrafilter $ \phi_{\ast}\mu $ on $ T $ given by
	\[
		\int_T f \ d (\phi_{\ast}\mu) = \int_S (f \circ \phi) \ d \mu \period
	\]
	For any $ U \subseteq T$, one has in particular
	\[
		(\phi_{\ast} \mu)(U) = \mu (\phi^{-1}(U)) \period
	\]
	Thus $ U $ is $ \phi_{\ast} \mu $-thick if and only if $ \phi^{-1} U $ is $ \mu $-thick.
\end{nul}

\begin{dfn}
	A \defn{system of thick subsets} of $ S $ is a collection $ F \subseteq \PP(S) $ such that for any finite set $ I $ and any partition
	\[
		S = \coprod_{ i \in I } S_i \comma
	\]
	there is a unique $ i \in I $ such that $ S_i \in F $.
\end{dfn}

\begin{cnstr}
	We have seen that an ultrafilter $ \mu $ specifies the system $ F_{\mu} $ of $ \mu $-thick subsets.
	In the other direction, attached to any system $F$ of thick subsets is an ultrafilter $\mu_F$: for any finite set $ I $ and any map $ f \colon S \to I $, the element $ i = \int_S f \ d \mu \in I $ is the unique one such that $ S_i \in F$.

	The assignments $ \mu \mapsto F_{\mu} $ and $ F \mapsto \mu_F $ together define a bijection between ultrafilters on $S$ and systems of thick subsets.
\end{cnstr}

\begin{dfn}
	If $ S $ is a set, and if $ G \subseteq \PP(S) $, then an ultrafilter $ \mu $ is said to be \defn{supported on $ G $} if and only if every element of $G$ is $ \mu $-thick, that is, $ G \subseteq F_{\mu} $.
\end{dfn}

\begin{lem} \label{generateultrafilters}
	Let $ S $ be a set, and let $ G \subseteq \PP(S) $.
	Assume that no finite intersection of elements of $ G $ is empty.
	Then there exists an ultrafilter $ \mu $ on $ S $ supported on $ G $.
\end{lem}

\begin{proof}
	Consider all the families $ A \subseteq \PP(S) $ with the following properties:
	\begin{enumerate}[(1)]
		\item $ A $ contains $ G $;
		\item \label{FIP} no finite intersection of elements of $ A $ is empty.
	\end{enumerate}
	By Zorn's lemma there is a maximal such family, $ F $.

	We claim that $ F $ is a system of thick subsets.
	For this, let $ S = \coprod_{i \in I} S_i $ be a finite partition of $ S $.
	Condition \ref{FIP} ensures that at most one of the summands $ S_i $ can lie in $ F $.
	Now suppose that none of the summands $ S_i $ lies in $ F $.
	Consider, for each $ i \in I $, the family $ F \cup \{ S_i \} \subseteq \PP(S) $;
	the maximality of $ F $ implies that none of these families can satisfy Condition \ref{FIP}.
	Thus for each $ i \in I $, there is an empty finite intersection
	$S_i \cap \bigcap_{j = 1}^{n_i} T_{ij} = \varnothing $.
	But this implies that the intersection $ \bigcap_{i \in I}\bigcap_{j = 1}^{n_i} T_{ij} $ is empty, contradicting Condition \ref{FIP} for $ F $ itself.
	Hence at least one -- and thus exactly one -- of the summands $ S_i $ lies in $ F $.
	Thus $ F $ is a system of thick subsets of $ S $.
\end{proof}

\begin{nul}
	It is not quite accurate to say that the Axiom of Choice is \emph{necessary} to produce nonprincipal ultrafilters, but it is true that their existence is independent of Zermelo--Fraenkel set theory.
\end{nul}

\begin{nul}
	If $ \phi $ is a functor $ \Set \to \Set $, then a natural transformation $ \phi \to \beta $ is the same thing as a natural transformation $ \phi \circ i \to i $.
	Please observe that we have a canonical identification $ \beta \circ i = i $.

	It follows readily that the functor $ \beta $ is a monad: the unit $ \delta \colon \id \to \beta $ corresponds to the identification $ {\id} \circ i = i $, and the multiplication $ \mu \colon \beta^2 \to \beta $ corresponds to the identification $ \beta^2 \circ i = i $.

	The unit for the monad $\beta$ structure is the assignment $ s \mapsto \delta_s $ that picks out the principal ultrafilter at a point.

	To describe the multiplication $ \tau \mapsto \mu_{\tau} $, let us write $ T^{\dag} $ for the set of ultrafilters supported on $\{T\}$.
	Now if $ \tau $ is an ultrafilter on $ \beta(S) $, then $ \mu_{\tau} $ is the ultrafilter on $S$ such that
	\[
		\mu_{\tau} ( T ) = \tau ( T^{\dag} ) \period
	\]
\end{nul}

\begin{cnstr}
	Let $ \categ{Top} $ denote the category of tiny topological spaces.
	If $ S $ is a set, we can introduce a topology on $ \beta(S) $ simply by forming the inverse limit $ \lim_{I \in \Fin_{S/}} I $ in $ \categ{Top} $.
	That is, we endow $ \beta(S) $ with the coarsest topology such that all the projections $ \beta(S) \to I $ are continuous.
	We call this the \defn{Stone topology} on $\beta(S)$.
	By Tychonoff, this limit is a compact Hausdorff topological space.
	This lifts $ \beta $ to a functor $ \Set \to \Top $.
\end{cnstr}

\begin{nul}
	Let's be more explicit about the topology on $ \beta(S) $.
	The topology on $ \beta(S) $ is generated by the sets $ T^{\dag} $ (for $ T \subseteq S $).
	In fact, since the sets $ T^{\dag} $ are stable under finite intersections, they form a base for the Stone topology on $ \beta(S) $.
	Additionally, since the sets $ T^{\dag} $ are stable under the formation of complements, they even form a base of clopens of $ \beta(S) $.
\end{nul}

\begin{dfn} \label{compactaasbetaalgebras}
	A \defn{compactum} is an algebra for the monad $ \beta $.
	Hence a compactum consists of a set $ K $ and a map $ \lambda_K \colon \beta(K) \to K $, which is required to satisfy the usual identities:
	\[
		\lambda_K(\lambda_{K,\ast}\tau) = \lambda_K(\mu_{\tau}) \andeq{} \lambda_K(\delta_s) = s \comma
	\]
	for any ultrafilter $ \tau $ on $ \beta(S) $ and any point $ s \in S $.
	The image $ \lambda_K(\mu) $ will be called the \defn{limit} of the ultrafilter $\mu$.
	We write $ \Comp $ for the category of compacta, and write $\Compfree \subset \Comp $ for the full subcategory spanned by the \defn{free compacta} -- \emph{i.e.}, free algebras for $ \beta $.
\end{dfn}

\begin{cnstr} \label{turnacompactumintoatopspace}
	If $K$ is a compactum, then we use the limit map $ \lambda_K \colon \beta(K) \to K $ to topologise $ K $ as follows.
	For any subset $ T \subseteq K $, we define the closure of $ T $ as the image $ \lambda_K(T^{\dag}) $.

	A subset $ Z \subseteq K $ is thus closed if and only if the limit of any ultrafilter relative to which $ Z $ is thick lies in $ Z $.
	Dually, a subset $ U \subseteq K $ is open if and only if it is thick with respect to any ultrafilter whose limit lies in $ U $.

	We denote the resulting topological space $ K^{\textit{top}} $.
	The assignment $ K \mapsto K^{\textit{top}} $ defines a lift $ \Alg(\beta) \to \Top $ of the forgetful functor $ \Alg(\beta) \to \Set $.
\end{cnstr}

\begin{prp} \label{compactaarecompacta}
	The functor $ K \mapsto K^{\textit{top}} $ identifies the category of compacta with the category of compact Hausdorff topological spaces. 
\end{prp}

We will spend the remainder of this section proving this claim.
Please observe first that $ K \mapsto K^{\textit{top}} $ is faithful.
What we will do now is prove:
\begin{enumerate}[(1)]
	\item that for any compactum $ K $, the topological space $ K^{\textit{top}} $ is compact Hausdorff;
	\item that for any compact Hausdorff topological space $ X $, there is a $ \beta $-algebra structure $ K $ on the underlying set of $ X $ such that $ X \cong K^{\textit{top}} $; and
	\item that for any compacta $ K $ and $ L $, any continuous map $ K^{\textit{top}} \to L^{\textit{top}} $ lifts to a $ \beta $-algebra homomorphism $ K \to L$.
\end{enumerate}
To do this, it is convenient to describe a related idea: that of \emph{convergence} of ultrafilters on topological spaces.

\begin{dfn} \label{limitpointofultrafilter}
	Let $ X $ be a topological space, and let $ x \in X $.
	We say that $ x $ is a \defn{limit point} of an ultrafilter $ \mu $ on (the underlying set of) $ X $ if and only if every open neighbourhood of $ x $ is $ \mu $-thick.
	In other words, $ x $ is a limit point of $ \mu $ if and only if, for every open neighbourhood $ U $ of $ x $, one has $ \mu \in U^{\dag} $.
\end{dfn}

\begin{lem} \label{opensetsarethickwrtultrafilters}
	Let $ X $ be a topological space, and let $ U \subseteq X$ be a subset.
	Then $ U $ is open if and only if it is thick with respect to any ultrafilter with limit point in $ U $.
\end{lem}

\begin{proof}
	If $ U $ is open, then $ U $ is by definition thick with respect to any ultrafilter with limit point in $ U $.

	Conversely, assume that $ U $ is thick with respect to any ultrafilter with limit point in $ U $.
	Let $ u \in U $.
	Consider the set $ G \coloneq N(u) \cup \{ X \smallsetminus U \} $, where $N(u)$ is the collection of open neighbourhoods of $u$.
	If $ U $ does not contain any open neighbourhood of $u$, then no finite intersection of elements of $ G $ is empty.
	By \Cref{generateultrafilters} there is an ultrafilter $ \mu $ supported on the $ N(u) \cup \{ X \smallsetminus U \} $, whence $ u $ is a limit point of $ \mu $, but $ U $ is not $ \mu $-thick.
	This contradicts our assumption, and so we deduce that $ U $ contains an open neighbourhood of $ u $.
\end{proof}

\begin{lem} \label{continuityviaultrafilters}
	Let $ X $ and $ Y $ be topological spaces, and let $ \phi \colon X \to Y $ be a map.
	Then $ \phi $ is continuous if and only if, for any ultrafilter $ \mu $ on $ X $ with limit point $ x \in X $, the point $ \phi(x) $ is a limit point of $ \phi_{\ast}\mu $.
\end{lem}

\begin{proof}
	Assume that $ \phi $ is continuous, and let $ \mu $ be an ultrafilter on $ X $, and assume that $ x \in X $ is a limit point of  $ \mu $.
	Now assume that $ V $ is an open neighbourhood of $ \phi(x) $.
	Since $ \phi^{-1}V $ is an open neighbourhood of $ x $, so it is $ \mu $-thick, whence $ V $ is $\phi_{\ast}\mu$-thick.
	Thus $ \phi(x) $ is a limit point of $ \phi_{\ast}\mu $.

	Assume now that if $ x \in X $ is a limit point of an ultrafilter $ \mu $, then $ \phi(x) $ is a limit point of $ \phi_{\ast}\mu $.
	Let $ V \subseteq Y $ be an open set.
	Let $ x \in \phi^{-1}(V) $, and let $ \mu $ be an ultrafilter on $ X $ with limit point $ x $.
	Then $ \phi(x) $ is a limit point of $ \phi_{\ast}\mu $, so $V$ is $ \phi_{\ast}\mu $-thick, whence $ \phi^{-1}(V) $ is $ \mu $-thick.
	It follows from \Cref{opensetsarethickwrtultrafilters} that $\phi^{-1}(V)$ is open.
\end{proof}

\begin{lem} \label{quasicompactiffeveryultrafilterhasalimitpoint}
	Let $ X $ be a topological space.
	Then $ X $ is quasicompact if and only if every ultrafilter on $ X $ has at least one limit point.
\end{lem}

\begin{proof}
	Assume first that $ X $ is quasicompact.
	Let $ \mu $ be an ultrafilter on $ X $, and assume that $ \mu $ has no limit point.
	Select, for every point $ x \in X $, an open neighbourhood $ U_x $ thereof that is not $ \mu $-thick.
	Quasicompactness implies that there is a finite collection $ x_1, \dots, x_n \in X $ such that $ \left\{ U_{x_1}, \dots, U_{x_n} \right\} $ covers $ X $.
	But at least one of $ U_{x_1}, \dots, U_{x_n} $ must be $ \mu $-thick.
	This is a contradiction.

	Now assume that $ X $ is not quasicompact.
	Then there exists a collection $ G \subseteq \PP(X) $ of closed subsets of $ X $ such that the intersection all the elements of $ G $ is empty, but no finite intersection of elements of $ G $ is empty.
	In light of \Cref{generateultrafilters}, there is an ultrafilter $ \mu $ with the property that every element of $ G $ is thick.
	For any $ x \in X $, there is an element $ Z \in G $ such that $ x \in X \smallsetminus Z $.
	Since $ Z $ is $ \mu $-thick, $ X \smallsetminus Z $ is not.
	Thus $ \mu $ has no limit points.
\end{proof}

\begin{lem} \label{hausdorffiffeveryultrafilterhasatmostonelimitpoint}
	Let $ X $ be a topological space.
	Then $ X $ is Hausdorff if and only if every ultrafilter on $ X $ has at most one limit point.
\end{lem}

\begin{proof}
	Assume that $ \mu $ is an ultrafilter with two distinct limit points $ x_1 $ and $ x_2 $.
	Choose open neighbourhoods $ U_1 $ of $ x_1 $ and $ U_2 $ of $ x_2 $.
	Since they are both $ \mu $-thick, they cannot be disjoint;
	hence $ X $ is not Hausdorff.

	Conversely, assume that $ X $ is not Hausdorff.
	Select two points $ x_1 $ and $ x_2 $ such that every open neighbourhoods $ U_1 $ of $ x_1 $ and $ U_2 $ of $ x_2 $ intersect.
	Now the set $ G $ consisting of open neighbourhoods of either $ x_1 $ \emph{or} $ x_2 $ has the property that no finite intersection of elements of $ G $ is empty.
	In light of \Cref{generateultrafilters}, there is an ultrafilter $ \mu $ with the property that every element of $ G $ is thick.
	Thus $ x_1 $ and $ x_2 $ are limit points of $ \mu $.
\end{proof}

Let us now return to our functor $ K \mapsto K^{\textit{top}} $.

\begin{lem} \label{limitsarelimits}
	Let $ K $ be a compactum, and let $ \mu $ be an ultrafilter on $ K $.
	Then a point of $ K^{\textit{top}} $ is a limit point of $ \mu $ in the sense of \Cref{limitpointofultrafilter} if and only if it is the limit of $ \mu $ in the sense of \Cref{compactaasbetaalgebras}.
\end{lem}

\begin{proof}
	Let $ x \coloneq \lambda_K(\mu) $.
	The open neighbourhoods $ U $ of $ x $ are by definition thick (relative to $ \mu $), so certainly $ x $ is a limit point of $ \mu $.

	Now assume that $ y \in K^{\textit{top}} $ is a limit point of $ \mu $.
	To prove that the limit of $ \mu $ is $ y $, we shall build an ultrafilter $ \tau $ on $ \beta(K) $ with the following properties:
	\begin{enumerate}[(1)]
		\item under the multiplication $ \beta^2 \to \beta $, the ultrafilter $ \tau $ is sent to $ \mu $; and
		\item under the map $ \lambda_{\ast} \colon \beta^2 \to \beta  $, the ultrafilter $ \tau $ is sent to $\delta_y$.
	\end{enumerate}
	Once we have succeeded, it will follow that
	\[
		\lambda_K( \mu ) = \lambda_K( \mu_{\tau} ) = \lambda_K(\lambda_{K,\ast}\tau) = \lambda_K(\delta_y) = y \comma
	\]
	and the proof will be complete.

	Consider the family $ G' $ of subsets of $ \beta(K) $ of the form $ T^{\dag} $ for a $ \mu $-thick subset $ T \subseteq S $;
	since these are all nonempty and they are stable under finite intersections, it follows that no finite intersection of elements of $ G' $ is empty.

	Now consider the set $ G \coloneq G' \cup \{ \lambda_K^{-1}\{y\} \}$.
	If $ T $ is $ \mu $-thick, then we claim that there is an ultrafilter $ \nu \in \lambda_K^{-1}\{y\} \cap T^{\dag} $.
	Indeed, consider the set $ N(y) \cup \{T\} $, where $ N(y) $ is the collection of open neighbourhoods of $ y $.
	Since every open neighbourhood of $ y $ is $ \mu $-thick, no intersection of an open neighbourhood of $ y $ with $ T $ is empty.
	By \Cref{generateultrafilters} there is an ultrafilter supported on $ N(y) \cup \{T\} $, which implies that no finite intersection of elements of $ G $ is empty.

	Applying \Cref{generateultrafilters} again, we see that $ G $ supports an ultrafilter $ \tau $ on $ \beta(K) $.
	For any $ T \subseteq K $,
	\[
		\mu_{\tau}(T) = \tau(T^{\dag}) \comma
	\]
	so since $ \tau $ is supported on $ G' $, it follows that $ \mu_{\tau} = \mu $.
	At the same time, since $ \tau $ is supported on $ \{\lambda_K^{-1}\{y\}\} $, it follows that $ \{y\} $ is thick relative to $ \lambda_{K,\ast}\tau $, whence $ \lambda_{K,\ast}\tau = \delta_y $.
\end{proof}

\begin{proof}[Proof of \Cref{compactaarecompacta}]
	Let $ K $ be a compactum.
	Combine \Cref{hausdorffiffeveryultrafilterhasatmostonelimitpoint,quasicompactiffeveryultrafilterhasalimitpoint,limitsarelimits} to conclude that $ K^{\textit{top}} $ is a compact Hausdorff topological space.

	Let $ X $ be a compact Hausdorff topological space with underlying set $ K $.
	Define a map $ \lambda_K \colon \beta(K) \to K $ by carrying an ultrafilter $ \mu $ to its unique limit point in $ X $.
	This is a $ \beta $-algebra structure on $ X $, and it follows from \Cref{limitsarelimits} and the definition of the topology together imply that $ X \cong K^{\textit{top}}$.

	Finally, let $ K $ and $ L $ be compacta, and let $ \phi \colon K^{\textit{top}} \to L^{\textit{top}} $ be a continuous map.
	To prove that $ \phi $ is a $ \beta $-algebra homomorphism, it suffices to confirm that if $ \mu $ is an ultrafilter on $ K $, then
	\[
		\lambda_L (\phi_{\ast} \mu) = \phi (\lambda_K(\mu)) \comma
	\]
	but this follows exactly from \Cref{continuityviaultrafilters}.
\end{proof}

\begin{nul}
	We opted in \Cref{turnacompactumintoatopspace} to define the topology on a compactum $ K $ in very explicit terms, but note that the map $ \lambda_K \colon \beta(K) \to K^{\textit{top}} $ is a continuous surjection between compact Hausdorff topological spaces.
	Thus $ K^{\textit{top}} $ is endowed with the quotient topology relative to $ \lambda_K $.
\end{nul}

%-------------------------------------------------------------------%
\subsection{Stone spaces and projective compacta}
%-------------------------------------------------------------------%

\begin{dfn}
	Let $ X $ be a topological space.
	One says that $ X $ is \defn{totally separated} if and only if, for any two distinct points $ x, y \in X $, there exists a clopen subset $ V \subseteq X $ that contains $ x $ but not $ y $.
\end{dfn}

\begin{lem}
	A compactum $ K $ is totally separated if and only if it admits a base consisting of clopen sets.
\end{lem}

\begin{proof}
	Assume that $ K $ is totally separated.
	Let $ U \subseteq K $ be an open subset.
	It suffices to show that for any point $ x \in U $, there is a clopen neighbourhood of $ x $ that is contained in $ U $.
	For any $ y \notin U $, let $ V_y \subseteq X $ be a clopen that contains $ y $ but not $ x $;
	now $ \{ V_y \}_{y \in X \smallsetminus U} $ covers $ X \smallsetminus U $.
	Since $ X \smallsetminus U $ is a closed subset of a compactum, it too is compact, whence there exist finitely many points $ y_1, \dots, y_n \in X \smallsetminus U $ such that $ \{ V_{y_1}, \dots, V_{y_n} \} $ cover $ X \smallsetminus U $.
	Now the complement 
	\[
		X \smallsetminus (V_{y_1} \cup \dots \cup V_{y_n})
	\]
	is a clopen neighbourhood of $ x $ contained in $ U $.

	Conversely, assume that $ X $ admits a base of clopen subsets, and let $ x, y \in X $ be distinct points of $ X $.
	By Hausdorffness, there exists an open neighbourhood $ U $ of $ x $ that does not contain $ y $. 
	Since $ X $ admits a base of clopen subsets, there is a clopen neighbourhood of $ x $ that is contained in $ U $, which therefore does not contain $ y $.
\end{proof}

\begin{dfn}
	A compactum is a \defn{Stone space} if and only if it is totally separated.
	Let us write $ \CompStone \subseteq \Comp $ for the full subcategory spanned by the Stone spaces.
\end{dfn}

\begin{exm}
	Clearly any finite set is a Stone space.

	More generally, let $ I \colon A^{\op} \to \Fin $ be a diagram of finite sets.
	If we form the limit $ K = \lim_{\alpha \in A^{\op}} I_{\alpha}^{\disc}$ in $ \Top $ or $ \Comp $, then $ K $ is a Stone space.
	Indeed, $ K $ is clearly Hausdorff and compact by Tychonoff;
	since it admits a base consisting of the inverse images of opens from the discrete spaces $ I_{\alpha}^{\disc} $, it follows that it admits a base of clopens.

	In particular, if $ S $ is any set, then $ \beta(S) = \lim_{I \in \Fin_{S/}} I $ is a Stone space.
\end{exm}

\begin{lem}
	Any Stone space is the inverse limit of its finite discrete quotients.
\end{lem}

\begin{proof}
	Let $ K $ be a Stone space.
	The category $ \Fin_{K/} $ of finite sets to which $ K $ maps (in $ \Comp $ is an \defn{ inverse } category -- i.e., the opposite of a filtered category.
	Limit-cofinal in $ \Fin_{K/} $ is the full subcategory spanned by the finite discrete quotients.
	Hence we aim to show that the natural continuous map
	\[
		p \colon K \to \lim_{I \in \Fin_{K/}} I
	\]
	is a homeomorphism.
	Since both source and target are compact Hausdorff topological spaces, it suffices to prove that $ p $ is a bijection.
	For this, let $ x = \{ x_I \}_{I \in \Fin_{K/}} $ be a point of the limit.
	For any finite discrete quotient $p_I \colon K \to I $, let $ W_I $ be the clopen set $ p_I^{-1}(x_I) $;
	each of these is clopen, and the claim now is that the intersection
	\[
		W \coloneq \bigcap_I W_I
	\]
	consists of exactly one point of $ K $.
	Since $ K $ is quasicompact, it follows that $ W $ is nonempty.
	Since $ K $ is totally disconnected and Hausdorff, it follows that if $ x \neq y $, there exists a continuous map to $\{ 0,1 \}^{\disc}$ such that $ x \mapsto 0 $ and $ y \mapsto 1 $;
	hence $ W $ contains at most one point.
\end{proof}

\begin{nul}
	In particular, $ \CompStone $ is the smallest full subcategory of $ \Top $ that contains $ \Fin $ and is closed under inverse limits.
\end{nul}

Inverse limits of compacta are exceptionally well behaved.
A key lemma that demonstrates this is the following.

\begin{lem} \label{emptyinverselimitofcompacta}
	Let $ \{ K_{\alpha} \}_{\alpha \in A^{\op}} $ be an inverse system of compacta, and assume that the inverse limit is empty.
	Then one of the $ K_{\alpha} $ is empty as well.
\end{lem}

\begin{proof}
	Let $ K $ be the product $\prod_{\alpha \in A} K_{\alpha}$;
	by Tychonoff it is compact.
	For any $ \beta \in A $, consider the subset
	\[
		Z_{\beta} \coloneq \left\{ (x_{\alpha})_{\alpha \in A} \in K : \forall \beta \to \alpha,\ \phi_{\alpha\beta} (x_{\alpha}) = x_{\beta} \right\} \period
	\]
	The subsets $Z_{\beta} \subseteq X$ are closed by Hausdorffness, and the intersection $\bigcap_{\beta \in A} Z_{\beta}$ is the limit of the $ K_{\alpha} $, which is empty.
	By compactness and the filteredness of $ A $, there exists an index $ \beta $ for which $ Z_{\beta} $ is empty.

	On the other hand, $Z_{\beta}$ is in bijection with $ K_{\beta} \times L_{\beta} $, where $ L_{\beta} $ is the product of $ K_{\gamma} $ over those $ \gamma \in A $ such that $ \gamma $ does not receive a map from $ \beta $.
	Thus one of these copies of $ K_{\alpha} $ is empty.
\end{proof}

\begin{exm}
	The compactness condition is necessary in the previous lemma.
	For instance, consider the inverse system
	\begin{equation*}
		\begin{tikzcd}[sep=1.5em]
			\cdots \arrow[r, "s", hooked] & \NN^{\disc} \arrow[r, "s", hooked] &  \NN^{\disc} \arrow[r, "s", hooked] & \NN^{\disc}
		\end{tikzcd}
	\end{equation*}
	where $ s \colon \NN \inclusion \NN $ is the successor function.
	Its limit is empty.
\end{exm}

\begin{lem}
	Any finite discrete set is cocompact as an object of $ \Comp $.
	Consequently, the fully faithful functor $ \Fin \inclusion \Comp $ extends to a limit-preserving fully faithful functor $ \Pro(\Fin) \inclusion \Comp $ whose essential image is $ \CompStone $.
\end{lem}

\begin{proof}
	Let $ \{ K_{\alpha} \}_{\alpha \in A^{\op}} $ be an inverse system of compacta, and let $ I $ be a finite set.
	Write $ K \coloneq \lim_{\alpha \in A^{\op}} K_{\alpha} $;
	the claim is that the map $ \colim_{\alpha \in A^{\op}} \Map(K_{\alpha}, I^{\disc}) \to \Map(K, I^{\disc}) $ is a bijection.

	For any topological space $ X $, a continuous map $ X \to I^{\disc} $ is the same thing as a partition of $ X $ into clopens indexed by the elements of $ I $.
	Hence by induction, it suffices to show:
	\begin{enumerate}[(1)]
		\item that every clopen $ V \subseteq K $ into two complementary clopens is the inverse image of some clopen of $ V_{\alpha} \subseteq K_{\alpha} $, and
		\item that if clopens $ V_{\alpha} \subseteq K_{\alpha} $ and $ V_{\beta} \subseteq V_{\beta} $ pull back to the same $ V \subseteq K $, then there are maps $ \alpha \to \gamma $ and $ \beta \to \gamma $ in $ A $ such that $ V_{\alpha} $ and $ V_{\beta} $ pull back to the same subset of $ K_{\gamma} $.
	\end{enumerate}

	For the first claim, consider a clopen $ V \subseteq K $.
	Since $ K $ has the inverse limit topology, $ V $ is a union of open sets of the form $ V_{\gamma} $, where $ V_{\gamma} $ is pulled back from an open $ K_{\gamma} $.
	But since $ V $ is also closed, it is quasicompact, and therefore by the filteredness of $ A $ there is a single $ \gamma \in A $ such that $ V $ is pulled back from $ V_{\gamma} $.
	The same analysis of the complement of $ V $ exhibits it as the pullback from some $ K_{\beta} $;
	now letting $ \alpha \in A $ be an object that receives maps from both $ \beta $ and $ \gamma $ completes the proof.

	For the second claim, suppose that clopens $ V_{\alpha} \subseteq K_{\alpha} $ and $ V_{\beta} \subseteq V_{\beta} $ pull back to the same $ V \subseteq K $.
	For any object $ \gamma \in A $ that receives morphisms from both $ \alpha $ and $ \beta $, let $ V_{\alpha\gamma} \subseteq K_{\gamma} $ denote the inverse image of $ V_{\alpha} $, and let $ V_{\beta\gamma} \subseteq K_{\gamma} $ denote the inverse image of $ V_{\beta} $.
	Let $D_{\gamma} \subseteq K_{\gamma} $ be the symmetric difference of $ V_{\alpha\gamma} $ and $ V_{\beta\gamma} $;
	its inverse image in $ K $ is empty.
	\Cref{emptyinverselimitofcompacta} now implies that for some index $\gamma$, the set $ D_{\gamma} $ is empty, whence $ V_{\alpha\gamma} = V_{\beta\gamma}$.
\end{proof}

\begin{dfn}
	By a \defn{projective compactum} we mean a projective object in compacta.
	That is, a compactum $ K $ is projective if and only if, for any surjection $ Y \surjection Z $, the map $ \Map(K, Y) \to \Map(K, Z) $ is also a surjection.
	We write $ \Compproj \subset \Comp $ for the full subcategory spanned by the projective compacta.
\end{dfn}

\begin{lem}
	The following are equivalent for a compactum $ K $.
	\begin{itemize}
		\item $ K $ is a retract of a free compactum $ \beta(S) $.
		\item $ K $ is projective.
	\end{itemize}
\end{lem}

\begin{proof}
	Since a set $ S $ is a projective object of $ \Set $, it follows that the free compactum $ \beta(S) $ is a projective compactum,
	and since surjections are stable under retracts, it follows that projective compacta are stable under retracts.

	Conversely, let $ K $ be a projective compactum, and let $ \lambda_K \colon \beta(K) \surjection K $ be the structure map.
	Since $ K $ is projective, $ \lambda_K $ admits a section, whence $ K $ is a retract of a free compactum.
\end{proof}

\begin{nul}
	In particular, please note that any projective compactum is a Stone space.
\end{nul}

\begin{dfn}
	A topological space $ X $ is \defn{extremally disconnected} if and only if any closure of an open subset is open.
\end{dfn}

\begin{nul}
	Taking complements, we see that a topological space $ X $ is extremally disconnected if and only if the interior of a closed subset is closed.
\end{nul}

\begin{nul}
	It is quite difficult to construct interesting examples of extremally disconnected topological spaces.
	Any metric space that is extremally disconnected is in fact discrete.
	The next lemma provides the main source of these examples.
\end{nul}

\begin{lem}
	If $ X $ is an extremally disconnected topological space, then if $ \{ Z_1, \dots, Z_n \} $ is a finite family of closed subsets that cover $ X $, then the interiors $ \iota(Z_1), \dots, \iota(Z_n) $ cover $ X $ as well.
\end{lem}

\begin{proof}
	Assume that $ 1 \leq i \leq n $ and that $ \{ \iota(Z_1), \dots, \iota(Z_{i-1}), Z_i, \dots, Z_n \} $ cover $ X $.
	Then since $ \iota(Z_1) \cup \dots \cup \iota(Z_{i-1}) \cup Z_{i+1} \cup \dots \cup Z_n $ is closed, it follows that $ \{ \iota(Z_1), \dots, \iota(Z_i), Z_{i+1}, \dots, Z_n \} $ cover $ X $ as well.
\end{proof}

\begin{prp}
	The following are equivalent for a compactum $ K $.
	\begin{itemize}
		\item $ K $ is projective.
		\item $ K $ is extremally disconnected as a topological space.
	\end{itemize}
\end{prp}

\begin{proof}
	Assume that $ K $ is projective, and let $ U \subseteq K $ be an open subset.
	Let $ Z $ be the complement of $ U $, and let $ V $ be its closure.
	The composite $ \phi $ of the inclusion $ Z \sqcup V \inclusion K \sqcup K $ followed by the fold map $ \nabla \colon K \sqcup K \to K $ is a surjection, so since $ K $ is projective, it admits a section $ \sigma \colon K \to Z \sqcup V $.
	For any $ x \in U $, one has $ \sigma(x) = x $, and by continuity the same holds for any $ x \in V $.
	Thus $ \sigma^{-1}(V) = V $, so since $ V $ is open in $ Z \sqcup V $, it follows that $ V $ is open in $ K $.

	Conversely, assume that $ K $ is extremally disconnected, assume that $ X \surjection Y $ is a surjection between compacta, and assume that $ f \colon K \to Y $ is a continuous map.
	A lift of $ f $ is the same thing as a section of the projection $ p \colon P \coloneq X \times_Y K \surjection K $.
	In other words, it suffices to prove the existence of a closed subset $ W \subseteq P $ such that $ p $ restricts to a homeomorphism $ W \equivalence K $.
	Consider the set of closed subsets $ W' \subseteq P $ such that $ p(W') = K $;
	Zorn's lemma ensures that this collection contains a minimal element $ W $.
	To show that $ p $ restricts to a homeomorphism on $ W $, it suffices to show that $ p $ restricts to an injection.

	Let $ x \neq y $ be distinct points of $ W $.
	Choose closed subsets $ E $ and $ F $ that cover $ W $ such that $ x \notin F $, and $ y \notin E $.
	The sets $ p(E) $ and $ p(F) $ cover $ K $.
	Since $ K $ is extremally disconnected, it follows that the interiors $ \iota(p(E)) $ and $ \iota(p(F))) $ also cover $ K $.

	So to prove that $ p(x) \neq p(y) $, we shall show that $ p(x) \notin \iota(p(F)) $, and that $ p(y) \notin \iota(p(E)) $.
	Without loss of generality it suffices to prove the first claim.
	Suppose that $ B \subseteq K $ is a closed subset such that $ B \cup p(F) = K $;
	we aim to show that $ p(x) \in B $.
	Indeed, $ p(p^{-1}(B) \cup F) = K $, so the minimality of $ W $ implies that $ p^{-1}(B) \cup F = W $, whence $ x \in p^{-1}(B) $.
\end{proof}
	

%-------------------------------------------------------------------%
\subsection{Compactly generated topological spaces}
%-------------------------------------------------------------------%

\begin{dfn}
	Let $ X $ be a topological space, and let $ \KK $ be a class of nonempty topological spaces.
	A \defn{$ \KK $-test map} is a continuous map $ f \colon K \to X $, where $ K \in \KK $.
	We say that a subset $ U \subseteq X $ is \defn{$ \KK $-open} if and only if for any $ \KK $-test map $ \phi \colon K \to X $, the subset $ f^{-1}(U) \subseteq K $ is open.
\end{dfn}

\begin{nul}
	The set of $ \KK $-open subsets of $ X $ is a topology -- the \defn{$ \KK $-topology}.
	We write $ X_{\KK} $ for this topological space.
	Clearly any open subset of $ X $ is $ \KK $-open, so the $ \KK $-topology is at least as fine as the originally topology.
\end{nul}

\begin{dfn}
	Let $ \KK $ be a class of nonempty topological spaces.
	A topological space $ X $ is said to be \defn{$ \KK $-generated} if and only if every $ \KK $-open subset of $ X $ is open -- that is, if and only if its topology coincides with the $ \KK $-topology, so that the continuous map $ X_{\KK} \to X $ is a homeomorphism.
\end{dfn}

\begin{nul}
	If $ \KK $ is a class of nonempty topological spaces, then the category $ \Top_{\KK} $ of $ \KK $-generated topological spaces is the smallest full subcategory of $ \Top $ containing $ \KK $ and stable under colimits.
	The inclusion $ \Top_{\KK} \inclusion \Top $ admits a right adjoint, which carries a topological space $ X $ to the topological space $ X_{\KK} $ with the $ \KK $-topology.
\end{nul}

\begin{exm}
	Since any compactum is a quotient of a free compactum, it follows that the $ \KK $-generated topological spaces coincide for the following classes $ \KK $:
	\begin{itemize}
		\item the collection of free compacta,
		\item the collection of projective compacta,
		\item the collection of Stone spaces, and
		\item the collection of all compacta.
	\end{itemize}
	In any of these cases, we say \defn{compactly generated} (or sometimes \defn{Kelley} or \defn{k}, but be aware that some authors reserve this word for compactly generated weak Hausdorff topological spaces!) as a synonym for $ \KK $-generated, and we write $ \Topcg $ for the full subcategory of $ \Top $ spanned by the compactly generated topological spaces.
	The right adjoint $ \Top \to \Topcg $ will be denoted $ k $;
	it is sometimes called \defn{kaonisation} or \defn{Kelleyification}.
\end{exm}

\begin{exm}
	The $ \KK $-generated topological spaces coincide for the following classes $ \KK $:
	\begin{itemize}
		\item the singleton consisting only of the interval $ [0,1] $,
		\item the singleton consisting only of the line $ \RR $,
		\item the collection of all cubes $ [0, 1]^n $,
		\item the collection of all Euclidean spaces $ \RR^n $, and
		\item the collection of all topological simplices $ |\Delta^n| $.
	\end{itemize}
	In any of these cases, one says \defn{$ \mbfDelta $-generated}, \defn{$I$-generated}, or \defn{numerically generated} as a synonym for $ \KK $-generated.
	We won't spend any significant quantity of time with numerically generated topological spaces, but they are a relatively well-adapted category for homotopy theory.
	(For example, the natural map $ X \to \pi_0 X $ is always continuous if $ X $ is numerically generated.)
\end{exm}

\begin{exm}
	First countable topological spaces are all compactly generated.
	Indeed, a first countable topological space is $ \{ \NN^{\ast} \} $-generated, where $ \NN^{\ast} $ is the one-point compactification of $ \NN $.
	(More generally, $\{ \NN^{\ast} \}$-generated topological spaces are sometimes called \defn{sequential}.)
	In particular, metrisable topological spaces are all compactly generated.
\end{exm}

\begin{exm}
	Every locally compact topological space is compactly generated.
\end{exm}

\begin{wrn}
	A subspace $ A $ of a compactly generated topological space $ X $ may not be compactly generated.
	The space $ A $ is compactly generated if it is either open or closed in $ X $, but in general, we shall refer to $ kA $ as the \defn{compactly generated subspace topology}.
\end{wrn}

\begin{cnstr}
	To form the limit of a diagram $ X \colon A^{\op} \to \Topcg $ of compactly generated topological spaces, one may first form the limit $ \lim^0 X $ in $ \Top $.
	This won't generally be compactly generated, so it is necessary to kaonise it:
	\[
		\lim X \cong k({\lim}^0\ X) \period
	\]
\end{cnstr}

\begin{exm}
	If $ X $ and $ Y $ are both first countable, then so is the product $ X \times^0 Y $, whence it is the product in $ \Topcg $.
\end{exm}

\begin{prp}
	If $ X $ is locally compact Hausdorff and $ Y $ is compactly generated, then the product topology $ X \times^0 Y $ is compactly generated, so it is the product in $ \Topcg $.
\end{prp}

\begin{proof}
	Let $ Z \subseteq X \times Y $ be a closed subset.
	We aim to show that $ Z $ is closed as a subset of $ X \times^0 Y $.
	Let $ (x_0, y_0) \in  X \times Y \smallsetminus Z $.
	Use the local compactness of $ X $ to select an open neighbourhood $ V $ of $ x_0 $ such that the closure $ \overline{ V } $ is compact, and $ \overline{ V } \times \{ y_0 \} $ does not meet $ Z $.
	Now let $ W $ be the set of all points $ y \in Y $ such that $ \overline{V} \times \{ y \} $ does not meet $ Z $;
	clearly $ y_0 \in W $, so it will suffice for us to prove that $ W $ is open.

	For this, we use that $ Y $ is compactly generated.
	Hence let $ f \colon K \to Y $ be a test map.
	The inverse image $ Z' $ of $ Z $ under $ \overline{ V } \times K \to X \times^0 Y $ is closed and thus compact and its image $ \pr_2(Z') $ under the projection $ \pi_2 \colon \overline{ V } \times K \to K $ is compact and thus closed.
	Observe that the complement of $ \pr_2(Z') $ is exactly the inverse image $ f^{-1}(W) $, which is open.
\end{proof}

\begin{cnstr}
	Let $ X $ and $ Y $ be compactly generated topological spaces.
	Write $ \Map(X, Y) $ for the set of continuous maps $ X \to Y $;
	we shall endow this with a compactly generated topology. 
	For any test map $ f \colon K \to X $ and any open set $ V \subseteq Y $, define
	\[
		U(f, V) \coloneq \{ g \in \Map(X, Y) : f(K) \subseteq g^{-1}(V) \}
	\]
	The \defn{compact-open topology} on $ \Map(X, Y) $ is the coarsest topology such that all the subsets $ W(f, V) $ are open;
	we write $ \Map^0(X, Y) $ for the set $ \Map(X, Y) $ with the compact-open topology.
	We now kaonise this topological space:
	\[
		\Map(X, Y) \coloneq k\Map^0(X, Y) \period
	\]

	If $ W $ is compactly generated, then for any continuous map $ g \colon W \to X $, the assignment $ f \mapsto f \circ g $ is a continuous map $ \Map( X, Y ) \to \Map( W, Y ) $;
	dually, for any continuous map $ h \colon Y \to W $, the assignment $ f \mapsto h \circ f $ is a continuous map $ \Map( X, Y ) \to \Map( X, W ) $.
\end{cnstr}

\begin{lem}
	Let $ X $ and $ Y $ be compactly generated topological spaces.
	Then the natural maps $ \varepsilon \colon X \times \Map( X, Y ) \to Y $ and $ p \colon Y \to \Map( X, X \times Y ) $ are continuous.
\end{lem}

\begin{proof}
	We first show that $\varepsilon$ is continuous.
	Let $ ( f, g ) \colon K \to X \times \Map( X, Y ) $ be a test map, and let $ V \subseteq Y $ be an open subset.
	We are interested in the inverse image $ W \coloneq ( f, g )^{-1}\varepsilon^{-1} V $.
	For any $ k \in W $, the map $ g(k) \circ f \colon K \to Y $ is continuous, so select a compact neighbourhood $ L $ of $ k $ such that $ f(L) \subseteq g(k)^{-1}(V) $.
	One has $ g(k) \in U(f|_L, V) $.
	Now the neighbourhood $ L \cap g^{-1}U(f|_L, V) $ of $ k $ lies in $ W $, so $ W $ is open.

	Now let us show that $ p $ is continuous.
	Let $ U \subseteq X \times Y $ be an open subset, and let $ f \colon K \to X $ be a test map.
	We aim to show that $ p^{-1} W(f, U) \subseteq $ is open.
	So let $ g \colon L \to Y $ be a test map;
	we form the inverse image
	\[
		g^{-1}(p^{-1}(W(f, U))) = \{ v \in L : K \times \{v\} \subseteq (f, g)^{-1}(U) \} \period
	\]
	This is now open by the `Tube Lemma' for compacta.
\end{proof}

\begin{cor}
	For any compactly generated topological spaces $ X $, $ Y $, and $ Z $, the natural map
	\[
		\Map( X, \Map(Y, Z) ) \to \Map( X \times Y, Z )
	\]
	is a bijection and even a homeomorphism.
	Thus $ \Map(Y, -) $ is right adjoint to the product $ - \times Y $.
\end{cor}

\begin{cor}
	The product of two quotient maps in $ \Topcg $ is again a quotient map.
\end{cor}

Let us briefly discuss filtered colimits of compactly generated topological spaces.
First, we note a peculiarity.

\begin{wrn}
	Every compact metrisable topological space $ X $ is the filtered colimit of its compact countable subspaces.

	Compacta are not compact in either $ \Top $ or $ \Topcg $.
	Indeed, the identity $ [0, 1] \to [0, 1] $ does not factor through any countable subspace, even though the interval can be exhibited as the colimit of its countable compact subspaces.
\end{wrn}

\begin{dfn}
	A filtered diagram $ X \colon A \to \Topcg $ is said to be \defn{admissible} if and only if the following conditions obtain:
	\begin{itemize}
		\item For any morphism $ \alpha \to \beta $ of $ A $, the map $ X_{\alpha} \to X_{\beta} $ is a closed inclusion.
		\item Every compact subset of the colimit $ \colim X $ lies in the image of some $ X_{\alpha} $.
	\end{itemize}
\end{dfn}

\begin{exm}
	Every sequence of closed inclusions of compactly generated topological spaces is an admissible filtered diagram.
\end{exm}

\begin{lem}
	For any compactum $ K $, the functor $ \Map(K, -) $ preserves admissible filtered diagrams.
\end{lem}

We won't give the (relatively routine) proof here, but it goes some way to expressing the compactness properties of compacta.

\begin{dfn}
	A compactly generated topological space $ X $ is \defn{weak Hausdorff} if and only if the diagonal $ \Delta_X \colon X \to X \times X $ is a closed inclusion.
	We write $ \Topcgwh \subset \Topcg $ for the full subcategory spanned by the weak Hausdorff topological spaces.
\end{dfn}

\begin{lem}
	A compactly generated topological space $ X $ is weak Hausdorff if and only if the image of any test map $ K \to X $ is closed.
\end{lem}

\begin{proof}
	Assume that $ X $ is weak Hausdorff, and let $ f \colon K \to X $ be a test map.
	If $ g \colon L \to X $ is a test map, then $ g^{-1}(f(K)) $ is equal to the image $ \pr_2((f, g)^{-1}(K)) $, which is the image of closed and thus compact subset of $ K \times L $.

	Conversely, assume that the image of any test map is closed.
	In particular, any point of $ X $ is closed.
	Let $ (f, g) \colon K \to X \times X $ be a test map;
	we aim to show that $ X \times_{X \times X} K \subseteq K $ is closed.
	Let $ k $ be an element of the complement $ K \smallsetminus ( X \times_{X \times X} K ) $;
	Let $ U \coloneq f^{-1}( X \smallsetminus \{ g(k) \} ) $;
	this set is an open neighbourhood of $ k $.
	Select a closed neighbourhood $ V $ of $ k $ that is contained in $ U $.
	Now the set $ g^{-1}(X \smallsetminus f(V)) $ is open, by weak Hausdorffness, and the intersection $ \iota(V) \cap W $ is an open neighbourhood of $ k $ that does not meet $ X \times_{X \times X} K $, as desired.
\end{proof}

\begin{lem}
	Let $ X $ be a compactly generated topological space, and let $ R \subseteq X \times X $ be an equivalence relation.
	Then the quotient topological space $ X/R $ is weak Hausdorff if and only if $ R $ is closed in $ X \times X $.
\end{lem}

\begin{proof}
	Since $ X \times X \to (X/R) \times (X/R) $ is a quotient map, it follows that $ \Delta_{X/R} $ is a closed immersion if and only if
	\[
		R \cong (X \times X) \times_{(X/R) \times (X/R)} X/R \to X \times X
	\]
	is a closed immersion.
\end{proof}

\begin{cor}
	For any compactly generated topological space $ X $, there exists a smallest closed equivalence relation $ H \subseteq X \times X $ such that $ X \to X/H $ is the universal compactly generated weak Hausdorff topological space to which $ X $ maps.
	In particular, the assignment $ X \mapsto hX \coloneq X/H $ defines a left adjoint to the inclusion $ \Topcgwh \inclusion \Topcg $.
\end{cor}







%!TEX root = rootfile.tex

%-------------------------------------------------------------------%
%-------------------------------------------------------------------%
\section{Pyknotic sets}
%-------------------------------------------------------------------%
%-------------------------------------------------------------------%

%-------------------------------------------------------------------%
\subsection{Definitions}
%-------------------------------------------------------------------%

\begin{cnstr}
	A continuous map of compacta $ K \to L $ is a surjection if and only if
	it is a quotient map.
	It is in that case an \emph{effective epimorphism};
	that is, the sequence
	\[
		K \times_L K \parallelto K \to L
	\]
	is a coequaliser.

	A pullback of a surjection in $ \Comp $ is again a surjection.
	The coproduct of a finite collection of surjections in $ \Comp $ is a surjection.
	Consequently, there is a Grothendieck topology on $ \Comp $ in which the covering sieves are families generated by surjections $ K \to L $.
	We call this the \defn{effective epimorphism topology}.
\end{cnstr}

\begin{nul}
	A sheaf (of sets) for the effective epimorphism topology is a functor $ F \colon \Comp^{\op} \to \Set $ such that for any surjection $ K \to L $, the sequence
	\[
		F(L) \to F(K) \parallelto F(K \times_L K)
	\]
	is an equaliser.
\end{nul}

\begin{wrn}
	As written, the object sets of all these categories is a proper class.
	Consequently, to deal honestly with categories of sheaves on these sites requires a little set-theoretic care.

	We work within \textsc{zfcu} -- Zermelo--Fraenkel set theory along with Grothendieck's \defn{Axiom of Universes}.
	This is the requirement that any cardinal is dominated by a strongly inaccessible cardinal (which we always take to be uncountable).
	For any strongly inaccessible cardinal $ \delta $, a set that is in bijection with one whose rank is strictly less than $ \delta $ is said to be \defn{$ \delta $-small}.
	More generally, we shall be slightly indolent in our usage, and we shall say that an abelian group, space, spectrum, category, \category, \emph{etc}., is \defn{$\delta $-small} if and only if it is \emph{equivalent} (in whatever sense is appropriate) to one whose underlying set is $ \delta $-small.

	The strongly inaccessible cardinals are linearly ordered:
	\[
		\delta_0 < \delta_1 < \cdots \period
	\]
	We will say that a set is \defn{tiny} if and only if it is in bijection with a $ \delta_0 $-small set;
	we will say that it is \defn{small} if and only if it is $ \delta_1 $-small;
	and we will say that it is \defn{large} if and only if it is $ \delta_2 $-small.%
	\footnote{This terminology isn't really ideal;
	if someone else has an alternative proposal, please tell me! --- CB}

	We will also introduce some shorthand notations:
	\begin{itemize}
		\item $ \Comp $, $ \CompStone $, $ \Compproj $, and $ \Compfree $ will be taken to mean the categories of tiny compacta, Stone spaces, projective compact, and free compacta (respectively).
		\item $ \Set $ will be taken to mean the category of small sets.
		\item $ \Top $, $ \Topcg $, and $ \Topcgwh $ will be taken to mean the categories of small topological spaces, compactly generated topological spaces, and compactly generated weak Hausdorff topological spaces (respectively).
	\end{itemize}
\end{wrn}

\begin{dfn}
	A \defn{pyknotic set} is a sheaf $ F \colon \Comp^{\op} \to \Set $ for the effective epimorphism topology.
	Thus the category $ \Pyk(\Set) $ of pyknotic sets is the full subcategory of $ \Fun(\Comp^{\op}, \Set) $ spanned by the sheaves for the effective epimorphism topology.
	Observe that $ \Pyk(\Set) $ is a $ \delta_1 $-presentable category.
\end{dfn}

\begin{lem}
	The full subcategories
	\[
		\Compfree \subseteq \Compproj \subseteq \CompStone \subseteq \Comp
	\]
	are bases for the effective epimorphism topology.
\end{lem}

\begin{proof}
	This follows from the fact that any compactum $ K $ is a quotient of the free compactum $ \beta(K^{\disc}) $.
\end{proof}

\begin{nul}
	Thus a pyknotic set may be described as a sheaf for the effective epimorphism topology on $ \CompStone $, $ \Compproj $, or $ \Compfree $.
	This fact has a number of useful consequences.

	First, the identification $ \CompStone \simeq \Pro(\Fin) $ identifies the effective epimorphism site of $ \CompStone $ with Bhatt and Scholze's proétale topology on a geometric point.
	Thus the category of pyknotic sets is the proétale topos of a geometric point.

	In $ \Compproj $, every surjection is split.
	Consequently, a functor $ F \colon \Compproj^{\op} \to \Set $ is a sheaf for the effective epimorphism topology if and only if it carries finite coproducts of projective compacta to products:
	\[
		F( K \sqcup L ) \equivalence F ( K ) \times F ( L ) \period
	\]
	The projective compacta are thus a collection of compact projective generators for the category $ \Pyk(\Set) $.

	Finally, $ \Compfree $ is the Kleisli category for the ultrafilter monad $ \beta $.
	Consequently, a pyknotic set $ F \colon \Compfree^{\op} \to \Set $ can be regarded as an algebra $ F_{\beta} $ for the `infinitary Lawvere theory' defined by $ \beta $:
	it carries any tiny set to a set $ F_{\beta}(S) $, and it carries any map $ S \to \beta(T) $ to a map $ F_{\beta}(T) \to F_{\beta}(S) $, all in such a way as to ensure that $ F_{\beta}(S \sqcup T) \cong F_{\beta}(S) \times F_{\beta}(T) $.
\end{nul}

\begin{exm}
	Any topological space $ X $ defines, via Yoneda, a pykontic set, which we will also denote $ X $:
	\[
		X(K) \coloneq \Map(K, X) \comma
	\]
	for any compactum $ K $.
\end{exm}

\begin{prp}
	The assignment above defines a fully faithful functor $ \Topcg \inclusion \Pyk(Set) $.
\end{prp}

\begin{proof}
	That the functor is fully faithful is the very definition of compact generation.
\end{proof}

\begin{cnstr}
	Let $ Y $ be a pyknotic set.
	Then $ Y $ can be exhibited as the canonical colimit of the compacta that map to it:
	\[
		Y \simeq \colim_{K \in \Comp_{/Y}} K \period
	\]
	We define a topological space $ Y^{\textit{top}} $ by forming this same colimit in the category $ \Top $.
	Since it is a colimit of compacta, $ Y^{\textit{top}} $ is a compactly generated topological space.
	The topological space $ Y^{\textit{top}} $ is thus the set $ Y(\ast) $ equipped with the quotient topology from the surjection
	\[
		\coprod_{\beta(S) \in \Compfree_{/Y}} \beta(S) \surjection Y(\ast)
	\]
	This colimit defines a left adjoint $ Y \mapsto Y^{\textit{top}} $ to the fully faithful functor $ \Topcg \inclusion \Pyk(Set) $.

	Thus compactly generated topological spaces form a locaisation of the category of pyknotic sets.
	Consequently, limits formed in $ \Topcg $ are the same as those formed in $ \Pyk(\Set) $.
	
	The inclusion $ \Topcgwh \inclusion \Pyk(\Set) $ also admits a left adjoint that carries $ Y \mapsto h(Y^{\textit{top}}) $.
	Thus limits formed in $ \Topcgwh $ are also the same as those formed in $ \Pyk(\Set) $.
\end{cnstr}

\begin{nul}
	If $ X \colon A \to \Pyk(\Set) $ is a filtered diagram of pyknotic sets, then the colimit is given objectwise:
	\[
		(\colim_{\alpha \in A} X_{\alpha})(K) \cong \colim_{\alpha \in A} X_{\alpha}(K) \period
	\]
	Note that the filtered colimit of compactly generated topological spaces is not preserved by the inclusion into pyknotic sets;
	in particular, compacta are compact objects in $ \Pyk(\Set) $, but not in $ \Topcg $.
\end{nul}

\begin{exm}
	For any set $ S $, there is both a discrete and an indiscrete pyknotic set attached to $ S $.
	The indiscrete pyknotic set attached to $ S $ is given by the formula
	\[
		S^{\indisc}(K) \cong \Map(K, S^{\indisc}) \cong \prod_{k \in |K|} S \comma
	\]
	where $ |K|$ denotes the set underlying $ K $. 
	
	The discrete pyknotic set attached to $ S $ is given by the formula
	\[
		S^{\disc}(K) \cong S^K \coloneq \colim_{\alpha \in A} \prod_{k \in K_{\alpha}} S \comma
	\]
	where $ K = \lim_{\alpha \in A^{\op}} K_{\alpha} $ is a Stone space, exhibited as an inverse limit of finite sets.
	Please observe that this is the right Kan extension along $ \{ \ast \} \inclusion \Fin^{\op} $ followed by a left Kan extension along $ \Fin^{\op} \inclusion \CompStone^{\op} $.
	Thus the discrete pyknotic sets $ Y $ are exactly those that are left Kan extended from a functor $ \Fin^{\op} \to \Set $ that is itself right Kan extended from the point.
\end{exm}

\begin{dfn}[Clausen \& Scholze]
	A pyknotic set $ X $ is a \defn{condensed set} if its values are all tiny, and the functor
	\[
		X \colon \Ind(\Fin^{\op}) \simeq \Pro(\Fin)^{\op} \to \Set_{\delta_0}
	\]
	is accessible (relative to the tiny universe).
\end{dfn}

\begin{exm}
	The Sierpinski space $ \{s, \eta \} $, in which $ \{\eta\} $ is open but $ \{ s \} $ is not, is not a condensed set.
	In a similar vein, indiscrete pyknotic sets on sets with more than two points are never condensed.
\end{exm}

\begin{cnstr}
	Since $ \Pyk (\Set) $ is a topos, it follows readily that there is an internal Hom between two pyknotic sets;
	that is, the functor $ - \times Y $ admits a right adjoint $ \Map( Y, - ) $, where $ \Map(Y, Z) $ is given by the assignment
	\[
		P \mapsto \Map(Y \times P, Z) \cong \Map(Y|P, Z|P) \comma
	\]
	where $ Y|P $ and $ Z|P $ are the restrictions of $ Y $ and $ Z $ to $ \Compproj_{/P} $.
\end{cnstr}

For compactly generated topological spaces, this presents a potential conflict of notation, but in fact all is well:
\begin{prp}
	Let $ X $ and $ Y $ be compactly generated topological spaces.
	Then the pyknotic set $ \Map(X, Y) $ is the one attached to the compactly generated topological space $ \Map(X, Y) $.
\end{prp}

%-------------------------------------------------------------------%
\subsection{Quasicompact and quasiseparated}
%-------------------------------------------------------------------%

\begin{nul}
	The category $ \Comp $ is a pretopos, and the topos $ \Pyk(\Set) $ is the corresponding topos.
	In particular, that means that compacta are exactly the same as the \emph{coherent} pyknotic sets -- that is, those that are quasicompact and quasiseparated.
	Though this is a complete proof, it is actually advantageous to be a little more direct about this.
\end{nul}

\begin{nul}
	In a topos $ \XX $, a \defn{covering} $ {\{U_i\}}_{i \in I} $ of an object $ X $ is a family of objects of $ \XX_{/X} $ such that the morphism
	\[
		\coprod_{i \in I} U_i \surjection X
	\]
	is an effective epimorphism.
	We say that $ {\{U_i\}}_{i \in I} $ \defn{covers} $ X $.

	The object $ X $ is \defn{quasicompact} if and only if, for every covering $ {\{ U_i \}}_{i \in I} $, there exists a finite subset $ I_0 \subseteq I $ such that $ {\{ U_i \}}_{i \in I_0} $ covers as well.

	A morphism $ X \to Y $ is \defn{quasicompact} if and only if, for any quasicompact object $ L $ any any morphism $ L \to Y $, the pullback $ X \times_Y L $ is quasicompact as well.

	The object $ X $ is \defn{quasiseparated} if and only if, for any two quasicompact objects $ K $ and $ K' $, the fibre product $ K \times_X K' $ is quasicompact as well.

	Finally, we say that $ X $ is \defn{coherent} if and only if it is both quasicompact and quasiseparated.
\end{nul}

\begin{nul}
	The 

	If $ K $ is quasicompact, and if $ K \surjection L $ is an effective epimorphism, then $ L $ is quasicompact as well.
	The proof familiar from topology adapts readily to this situation.
\end{nul}

\begin{lem}
	A pyknotic set $ Y $ is quasicompact if and only if, for any covering $ {\{ P_i \}}_{i \in I} $ of $ Y $ by projective compacta, there exists a finite subset $ I_0 \subseteq I $ such that $ {\{P_i\}}_{i \in I_0} $ covers as well.
\end{lem}

\begin{proof}
	It is clear that any quasicompact object enjoys this property.
	Conversely, suppose that $ Y $ enjoys this property, and suppose that $ {\{ U_i \}}_{i \in I} $ is a covering of $ Y $. 
	Choose, for each $ i \in I $, a covering $ {\{ P_j \}}_{j \in J_i} $ comprised of projective compacta.
	
	Write $ J = \coprod_{i \in I} J_i $, and let $ f \colon J \to I $ be the map that carries $ j $ to the element $ i \in I $ such that $ j \in J_i $.
	Hence the objects $ {\{ P_j \}}_{j \in J} $ cover $ Y $, 
	and by assumption, there exists a finite subset $ J_0 \subseteq J $ such that $ {\{ P_j \}}_{j \in J_0} $ cover $ Y $. 
	Now let $ I_0 \coloneq f(J_0) $;
	then $ {\{ U_i \}}_{i \in I_0} $ covers $ Y $, and the proof is complete.
\end{proof}

\begin{cor}
	Any compactum is quasicompact as a pyknotic set.
\end{cor}

\begin{lem}
	A pyknotic set $ Y $ is quasiseparated if and only if, for any projective compacta $ P $ and $ P' $ and any morphism $ P \to Y $ and $ P' \to Y $, the pullback $ P \times_Y P' $ is quasicompact.
\end{lem}

\begin{proof}
	It is clear that any quasiseparated pyknotic set enjoys this property.
	Conversely, assume that $ Y $ enjoys this property, let $ K $ and $ K' $ be quasicompact objects, and let $ K \to Y $ and $ K' \to Y $ be morphisms.
	Using quasicompacness, we may select effective epimorphisms $ P \surjection K $ and $ P' \surjection K' $.
	By assumption, $ P \times_Y P' $ is quasicompact, and the map $ P \times_Y P' \surjection K \times_Y K' $ is an effective epimorphism, so $ K \times_Y K' $ is quasicompact as well.
\end{proof}

\begin{cor}
	Any compactum is quasiseparated as a pyknotic set.
\end{cor}

\begin{prp}
	Any coherent pyknotic set is a compactum.
\end{prp}

\begin{proof}
	Let $ Y $ be a coherent pyknotic set.
	By quasicompacness there exists an effective epimorphism $ K \surjection Y $, and by quasiseparatedness the fibre product $ K \times_Y K $ is quasicompact.
	By the definition of the effective epimorphism topology, if $ K \times_Y K $ is represented by a closed subspace of $ K \times K $, then the quotient $ Y $ will be the quotient of $ K $ as computed in $ \Comp $.
	We therefore aim to show that $ K \times_Y K $ is represented by a closed subspace of $ K \times K $.

\end{proof}




%-------------------------------------------------------------------%
%-------------------------------------------------------------------%
%-------------------------------------------------------------------%
%  References                                                       %
%-------------------------------------------------------------------%
%-------------------------------------------------------------------%
%-------------------------------------------------------------------%

\newpage

% Separates bibliography into two parts: the specially labeled references followed by the numbered references. The specially labeled references appear with just the label; numbered references appear as "n."
\DeclareFieldFormat{labelnumberwidth}{#1}
\printbibliography[keyword=alph, heading=references]
% \addcontentsline{toc}{part}{References} % Adds References to table of contents
\DeclareFieldFormat{labelnumberwidth}{{#1\adddot\midsentence}}
\printbibliography[heading=none, notkeyword=alph]

%-------------------------------------------------------------------%
%-------------------------------------------------------------------%
%-------------------------------------------------------------------%
%  Indices                                                          %
%-------------------------------------------------------------------%
%-------------------------------------------------------------------%
%-------------------------------------------------------------------%

\newpage
% \addcontentsline{toc}{part}{Index of Notation}
\printindex[notation]

\newpage
% \addcontentsline{toc}{part}{Glossary of Terminology}
\printindex[terminology]

\end{document}